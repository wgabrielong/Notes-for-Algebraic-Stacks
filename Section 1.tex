\part*{Grothendieck Topologies, Sites, and Fibered Categories}
\section{Grothendieck Topologies}
Let us recall the following definitions. 
\begin{definition}[Presheaf]\label{def: presheaf on topological space}
    Let $X$ be a topological space. A presheaf of sets $\Fcal$ on $X$ is a functor $(X^{\Opens})^{\Opp}\to\Sets$. 
\end{definition}
A presheaf is a sheaf if it satisfies additional gluing axioms. 
\begin{definition}[Sheaf]\label{def: topological sheaf}
    Let $X$ be a topological space. A sheaf of sets $\Fcal$ on $X$ is a functor $(X^{\Opens})^{\Opp}\to\Sets$ such that the sequence 
    $$% https://q.uiver.app/#q=WzAsMyxbMCwwLCJcXEZjYWwoWCkiXSxbMiwwLCJcXHByb2Rfe2l9XFxGY2FsKFhfe2l9KSJdLFs0LDAsIlxccHJvZF97aSxqfVxcRmNhbChYX3tpfVxcY2FwIFhfe2p9KSJdLFswLDFdLFsxLDIsIiIsMCx7Im9mZnNldCI6LTF9XSxbMSwyLCIiLDAseyJvZmZzZXQiOjF9XV0=
    \begin{tikzcd}
        {\Fcal(X)} && {\prod_{i}\Fcal(X_{i})} && {\prod_{i,j}\Fcal(X_{i}\cap X_{j})}
        \arrow[from=1-1, to=1-3]
        \arrow[shift left, from=1-3, to=1-5]
        \arrow[shift right, from=1-3, to=1-5]
    \end{tikzcd}$$
    is an equalizer for $\{X_{i}\}$ an open cover of $X$.  
\end{definition}
In this way, we say that a sheaf on $X$ is a presheaf on $X$ satisfying descent. Note here that we implicitly used the fact that $X^{\Opens}$ can be naturally endowed with the strucutre of a category with objects open sets of the topological space $X$ and morphisms inclusions of such open sets. This begs the question if we can replace $X^{\Opens}$ with some other category $\Csf$, allowing us to define a sheaf on an arbitrary category $\Csf$. We do this via the construction of a Grothendieck topology by replacing open sets of a topological space with maps into this space. 
\begin{definition}[Grothendieck Topology]\label{def: Grothendieck topology}
    Let $\Csf$ be a category. A Grothendieck topology on $\Csf$ is the data of a set $\{X_{i}\to X\}$ for each object $X\in\Obj(\Csf)$ known as a covering of $X$ such that the following conditions hold:
    \begin{enumerate}[label=(\alph*)]
        \item If $Y\to X$ is an isomorphism then $\{Y\to X\}$ is a covering.  
        \item If $\{X_{i}\to X\}$ is a covering and $Y\to X$ any morphism then $\{X_{i}\times_{X}Y\}$ exist and $\{X_{i}\times_{X}Y\to X\}$ is a covering. 
        \item If $\{X_{i}\to X\}$ is a covering and for each $i$ $\{X_{ij}\to X_{i}\}$ is a covering then the composites $\{X_{ij}\to X_{i}\to X\}$ is a covering. 
    \end{enumerate}
\end{definition}
This allows us to define a site. 
\begin{definition}[Site]\label{def: site}
    A site on a category $\Csf$ is the category $\Csf$ endowed with a Grothendieck topology. 
\end{definition}
Let us see some examples.
\begin{example}[Site of a Topological Space]
    Let $X$ be a topological space and $X^{\Opens}$ the category of open sets of $X$ with morphisms inclusions. One can endow $X^{\Opens}$ with a Grothendieck topology by associating to each $U\subseteq X$ open, the set of open covers of $U$. The fiber product is given by intersection of open sets, which agrees with our previous discussion of sheaves and presheaves. 
\end{example}
\begin{example}
    Consider the category of topological spaces $\Top$. The category $\Top$ can be endowed with a Grothendieck topology by taking covers of a topological space $X\in\Obj(\Top)$ to open, continuous, injective maps $X_{i}\to X$. 
\end{example}
\begin{example}[\'{E}tale Site of a Scheme]
    Let $S$ be a scheme. The \'{e}tale site of $S$, denoted $X_{\et}$, is the full subcategory of $\Sch_{S}$ where covers are \'{e}tale morphisms. 
\end{example}
It is sometimes useful to consider coverings that are not finitely presented. We do this by using the fpqc topology. 
\begin{definition}[fpqc Morphism]\label{def: fpqc morphism}
    Let $f:X\to Y$ be a surjective morphism of schemes. $f$ is an fpqc morphism if for every affine open $V\subseteq Y$, $f^{-1}(V)$ is quasicompact in $X$.
\end{definition}
\begin{remark}
    The notation fpqc arises from the French, fid\`{e}lement plat et quasi-compact. 
\end{remark}
This allows us to define the fpqc topology on the category of schemes over a fixed scheme $S$. 
\begin{definition}[fpqc Site]\label{def: fpqc site}
    Let $S$ be a scheme. The fpqc site on $S$-schemes $(\Sch_{S})_{\fpqc}$ is the data of coverings $\{X_{i}\to X\}$ for each $X\in\Obj(\Sch_{S})$ such that $\coprod_{i}X_{i}\to X$ is fpqc. 
\end{definition}
We can thus show the following proposition. 
\begin{proposition}
    Let $X\to Y$ be a morphism of schemes and $\{Y_{i}\to Y\}$ an fpqc covering of $Y$. Suppose that for each $i$, the induced map $Y_{i}\times_{Y}X\to Y_{i}$ has one of the following properties: 
    \begin{enumerate}[label=(\alph*)]
        \item separated, 
        \item quasicompact, 
        \item locally of finite presentation, 
        \item proper, 
        \item affine, 
        \item finite, 
        \item flat, 
        \item smooth, 
        \item unramified, 
        \item \'{e}tale,
        \item is an embedding, 
        \item or is a closed embedding
    \end{enumerate}
    then so does $X\to Y$. 
\end{proposition}
\begin{proof}
    All these properties and being fpqc are affine-local on target and the condition on the induced map $Y_{i}\times_{Y}X\to Y$ implies the condition on the map $X\to Y$ by \cite[Prop. 2.7.1]{EGAIVpt1}. 
\end{proof}
Having defined a topology, we can now define a sheaf theory on a site. Recall from earlier that sheaves on a topological space $X$ is a presheaf that satisfies additional axioms. The sheaf condition can be generalized to any site, replacing intersections of open sets with fibered products. 
\begin{definition}[Separated Presheaf on a Site]\label{def: sep presheaf on site}
    Let $\Csf$ be a site and $F:\Csf^{\Opp}\to\Sets$ a presheaf of sets on $\Csf$. $F$ is a separated functor if there are $a,b\in F(X)$ whose pullbacks to $F(X_{i})$ coincide then $a=b$ on $F(U)$. 
\end{definition}
\begin{definition}[Sheaf on a Site]\label{def: sheaf on site}
    Let $\Csf$ be a site and $F:\Csf^{\Opp}\to\Sets$ a presheaf of sets on $\Csf$. $F$ is a sheaf if for all coverings $\{X_{i}\to X\}$ the sequence 
    $$% https://q.uiver.app/#q=WzAsMyxbMCwwLCJGKFgpIl0sWzIsMCwiXFxwcm9kX3tpfUYoWF97aX0pIl0sWzQsMCwiXFxwcm9kX3tpLGp9RihYX3tpfVxcdGltZXNfe1h9WF97an0pIl0sWzAsMV0sWzEsMiwiXFxwcl97MX1eeyp9IiwwLHsib2Zmc2V0IjotMX1dLFsxLDIsIlxccHJfezJ9XnsqfSIsMix7Im9mZnNldCI6MX1dXQ==
    \begin{tikzcd}
        {F(X)} && {\prod_{i}F(X_{i})} && {\prod_{i,j}F(X_{i}\times_{X}X_{j})}
        \arrow[from=1-1, to=1-3]
        \arrow["{\pr_{1}^{*}}", shift left, from=1-3, to=1-5]
        \arrow["{\pr_{2}^{*}}"', shift right, from=1-3, to=1-5]
    \end{tikzcd}$$
    is an equalizer. 
\end{definition}
\begin{remark}
    As in the case of sheaves on a topological space, the condition on equalizers is equivalent to the gluing of sections. Given a covering $\{X_{i}\to X\}$ function of sets $F(X)\to \prod_{i}F(X_{i})$ is induced along restrictions and the natural projection maps $$\pr_{1}:X_{i}\times_{X}X_{j}\to X_{i},\pr_{2}:X_{i}\times_{X}X_{j}\to X_{j}$$ sends $a_{i}\in F(X_{i})$ to $\pr_{1}^{*}a_{i}$ with $\pr_{2}^{*}$ defined similarly. The equalizer condition states that if pullbacks of sections $\pr_{1}^{*}a_{i}=\pr_{2}^{*}a_{j}$ as elements of $F(X_{i}\times_{X}X_{j})$ then they are pulled back from some $a$ on $F(X)$ restricting to the $a_{i}$ on $F(X_{i})$. 
\end{remark}
Given an object $X\in\Obj(\Csf)$ we can define a sieve on $X$ as follows. 
\begin{definition}[Sieve]
    Let $X\in\Obj(\Csf)$. A sieve on $X$ is a subfunctor of the Yoneda functor $h_{X}:\Csf^{\Opp}\to\Sets$.
\end{definition}
Given a subfunctor $S$ of $h_{X}$, we get a collection $\Scal$ of arrows $T\to X$ by taking $\cup_{T\in\Obj(\Csf)}S(T)$ such that for $T\to X$ in $\Scal$, every composite $T'\to T\to X$ is in $\Scal$. Conversely, given $\Scal$ we can restrict the Yoneda functor appropriately to get $S:\Csf^{\Opp}\to\Sets$. 
\\\\
For $\{X_{i}\to X\}$ a a collection of morphisms and $X\in\Obj(\Csf)$ we can similarly define a subfunctor $h_{\{X_{i}\to X\}}:\Csf^{\Opp}\to\Sets$ where $T\to X$ if and only if there exists a factorization $T\to X_{i}\to X$ for some $i$. In the case of the functor $h_{\{X_{i}\to X\}}$, the metaphor of the sieve becomes much more clear: the $X_{i}$ are the holes of the sieve, and $T\to X$ is in $h_{\{X_{i}\to X\}}(T)$ if and only if it ``goes through one of the holes''. 
\\\\
For a collection $\{X_{i}\to X\}$ and $F:\Csf^{\Opp}\to\Sets$ a functor, we can define $F(\{X_{i}\to X\})$ to be the set of elements of $\prod_{i}F(X_{i})$ whose images in $\prod_{i,j}F(X_{i}\times_{X}X_{j})$ are equal. Sections of $F(X)$ evidently give rise to sections of $\prod_{i}F(X_{i})$ that agree on $\prod_{i,j}F(X_{i}\times_{X}X_{j})$ inducing a map $F(X)\to F(\{X_{i}\to X\})$. Note here that if $F$ is a sheaf the descent condition implies the map $F(X)\to F(\{X_{i}\to X\})$ is a bijection for all coverings. 
\\\\
We show that the set $F(\{X_{i}\to X\})$ can be defined in terms of sieves. 
\begin{proposition}\label{prop: sieve of cover}
    There is a canonical bijection of sets $$R:\NatTrans(h_{\{X_{i}\to X\}},F)\to F(\{X_{i}\to X\})$$ such that the diagram 
    $$% https://q.uiver.app/#q=WzAsNCxbMCwwLCJcXE5hdFRyYW5zKGhfe1h9LEYpIl0sWzIsMCwiRihYKSJdLFswLDEsIlxcTmF0VHJhbnMoaF97XFx7WF97aX1cXHRvIFhcXH19LCBGKSJdLFsyLDEsIkYoXFx7WF97aX1cXHRvIFhcXH0pIl0sWzIsMywiUiIsMl0sWzEsM10sWzAsMV0sWzAsMl1d
    \begin{tikzcd}
        {\NatTrans(h_{X},F)} && {F(X)} \\
        {\NatTrans(h_{\{X_{i}\to X\}}, F)} && {F(\{X_{i}\to X\})}
        \arrow["R"', from=2-1, to=2-3]
        \arrow[from=1-3, to=2-3]
        \arrow[from=1-1, to=1-3]
        \arrow[from=1-1, to=2-1]
    \end{tikzcd}$$
    commutes and $R$ is universal with respect to this property. 
\end{proposition}