\documentclass{amsart}
\usepackage[margin=1.1in]{geometry} 
\usepackage{amsmath}
\usepackage{tcolorbox}
\usepackage{amssymb}
\usepackage{amsthm}
\usepackage{lastpage}
\usepackage{fancyhdr}
\usepackage{accents}
\usepackage{hyperref}
\usepackage{xcolor}
\usepackage{color}
\input{shortcuts.tex}
\setlength{\headheight}{40pt}


\newenvironment{solution}
  {\renewcommand\qedsymbol{$\blacksquare$}
  \begin{proof}[Solution]}
  {\end{proof}}
\renewcommand\qedsymbol{$\blacksquare$}

\usepackage{amsmath, amssymb, tikz, amsthm, csquotes, multicol, footnote, tablefootnote, biblatex, wrapfig, float, quiver, mathrsfs, cleveref, enumitem, upgreek,stmaryrd, bm}
\addbibresource{refs.bib}
\theoremstyle{definition}
\newtheorem{theorem}{Theorem}[section]
\newtheorem{lemma}[theorem]{Lemma}
\newtheorem{corollary}[theorem]{Corollary}
\newtheorem{exercise}[theorem]{Exercise}
\newtheorem{question}[theorem]{Question}
\newtheorem{example}[theorem]{Example}
\newtheorem{proposition}[theorem]{Proposition}
\newtheorem{conjecture}[theorem]{Conjecture}
\newtheorem*{remark}{Remark}
\newtheorem{definition}[theorem]{Definition}
\newtheorem*{nntheorem}{Theorem}
\numberwithin{equation}{section}
\setcounter{tocdepth}{1}
\begin{document}
\large
\title[Algebraic Stacks]{Algebraic Stacks}
\author{Wern Juin Gabriel Ong}
\address{Bowdoin College, Brunswick, Maine 04011}
\email{gong@bowdoin.edu}
\urladdr{https://wgabrielong.github.io/}
\maketitle
\section*{Overview}
These notes roughly correspond to an attempt to learn stack theory that began in the winter of 2023. We will begin with the categorical preliminaries as laid out in \cite{Vistoli} before developing the basic theory per the text of Olsson \cite{Olsson} and conclude with a sampling of the more advanced topics in \cite[Part 7]{stacks-project}. The standard texts are \cite{LaumonMoret-Bailly} and \cite{Olsson}. The compendium \cite{stacks-project} is encyclopedic. 
\tableofcontents
\newpage
\part*{Grothendieck Topologies, Sites, and Fibered Categories}
\section{Grothendieck Topologies}
\section{Fibered Categories}\label{sec: fibered categories}
Fix a base category $\Ssf$. We consider categories over $\Ssf$. That is, those categories $\Fsf$ with a functor $p:\Fsf\to\Ssf$. More explicitly, we have the following definition. 
\begin{definition}[Category Over]\label{def: category over}
    We say that $\Fsf$ is a category over $\Ssf$ if there exists a functor $p:\Fsf\to\Ssf$. 
\end{definition}
The relationship between the categories $\Fsf$ and $\Ssf$ are given by ``lying over'' in the following sense. 
\begin{definition}[Lying Over]
    Consider $\Fsf$ over $\Ssf$ with $p:\Fsf\to\Ssf$. 
    \begin{enumerate}[label=(\alph*)]
        \item (Objects) An object $\alpha\in\Obj(\Fsf)$ lies over $A\in\Obj(\Ssf)$ if $p(\alpha)=A$. 
        \item (Morphisms) An morphism $\phi:\alpha\to\beta$ in $\Fsf$ lies over $f:A\to B$ in $\Ssf$ if the diagram 
        $$% https://q.uiver.app/#q=WzAsNCxbMCwwLCJcXGFscGhhIl0sWzIsMCwiXFxiZXRhIl0sWzAsMSwiQSJdLFsyLDEsIkIiXSxbMCwxLCIoXFxhbHBoYVxcdG9cXGJldGEpIl0sWzEsMywicCgtKSJdLFswLDIsInAoLSkiLDJdLFsyLDMsInAoXFxhbHBoYVxcdG9cXGJldGEpIiwyXV0=
        \begin{tikzcd}
            \alpha && \beta \\
            A && B
            \arrow["{\phi}", from=1-1, to=1-3]
            \arrow["{p(-)}", from=1-3, to=2-3]
            \arrow["{p(-)}"', from=1-1, to=2-1]
            \arrow["{p(\phi)=f}"', from=2-1, to=2-3]
        \end{tikzcd}$$
        commutes. 
    \end{enumerate}
\end{definition}
This allows us to define the notion of a Cartesian morphism in $\Fsf$. 
\begin{definition}[Cartesian Morphism]\label{def: cartesian morphism}
    Let $\Fsf$ be a category over $\Ssf$. A morphism $\phi:\alpha\to\beta$ in $\Fsf$ is Cartesian if for any $\psi:\eta\to\beta$ in $\Fsf$ and $g:p(\eta)\to p(\alpha)$ in $\Ssf$ with $p(\phi)\circ g= p(\psi)$ in $\Ssf$ there exists a unique $\rho:\eta\to\alpha$ lying over $g$ making the diagram 
    $$% https://q.uiver.app/#q=WzAsNixbMCwwLCJcXGV0YSJdLFsyLDEsIlxcYWxwaGEiXSxbMiwyLCJwKFxcYWxwaGEpIl0sWzAsMSwicChcXGV0YSkiXSxbNCwxLCJcXGJldGEiXSxbNCwyLCJwKFxcYmV0YSkiXSxbMCw0LCJcXHBzaSIsMV0sWzAsMSwiXFxleGlzdHMhXFxyaG8iLDEseyJzdHlsZSI6eyJib2R5Ijp7Im5hbWUiOiJkYXNoZWQifX19XSxbMCwzXSxbMywyLCJnIiwxXSxbMSwyXSxbMyw1XSxbMSw0LCJcXHBoaSIsMV0sWzIsNV0sWzQsNV1d
    \begin{tikzcd}
        \eta \\
        {p(\eta)} && \alpha && \beta \\
        && {p(\alpha)} && {p(\beta)}
        \arrow["\psi"{description}, from=1-1, to=2-5]
        \arrow["{\exists!\rho}"{description}, dashed, from=1-1, to=2-3]
        \arrow[from=1-1, to=2-1]
        \arrow["g"{description}, from=2-1, to=3-3]
        \arrow[from=2-3, to=3-3]
        \arrow[from=2-1, to=3-5]
        \arrow["\phi"{description}, from=2-3, to=2-5]
        \arrow[from=3-3, to=3-5]
        \arrow[from=2-5, to=3-5]
    \end{tikzcd}$$
    commute, and universal with respect to that property. 
\end{definition}
One can show that Cartesian morphisms satisfy the following properties. 
\begin{proposition}
    Let $\Fsf$ be a category over $\Ssf$. 
    \begin{enumerate}[label=(\alph*)]
        \item The composite of Cartesian arrows in $\Fsf$ is Cartesian. 
        \item If $\alpha\to\beta$ and $\beta\to\eta$ are morphisms in $\Fsf$ and $\beta\to\eta$ is Cartesian then $\alpha\to\beta$ is Cartesian if and only if the composite $\alpha\to\beta\to\eta$ is Cartesian. 
        \item A morphism $\phi$ in $\Fsf$ such that $p(\phi)$ is an isomorphism in $\Ssf$ is Cartesian if and only if $\phi$ is an isomorphism in $\Fsf$. 
        \item Let $p:\Csf\to\Ssf$ and $F:\Fsf\to\Csf$ be functors and $\phi:\alpha\to\beta$ a morphism in $\Fsf$. If $\phi$ is Cartesian over $F(\phi):F(\alpha)\to F(\beta)$ in $\Csf$ and $F(\phi)$ is Cartesian over $p(F(\phi)):p(F(\alpha))\to p(F(\beta))$ in $\Ssf$ then $\phi$ is Cartesian over $p(F(\phi)):p(F(\alpha))\to p(F(\beta))$ in $\Ssf$. 
    \end{enumerate}
\end{proposition}
This in turn allows us to define fibered categories. 
\begin{definition}[Fibered Category]\label{def: fibered category}
    Let $\Fsf$ be a category over $\Ssf$. $\Fsf$ is a fibered category over $\Ssf$ if for all $f:A\to B$ in $\Ssf$ and $\beta$ lying over $B$ there is a Cartesian morphism $\phi:\alpha\to\beta$ in $\Fsf$ lying over $f$ in $\Ssf$. 
\end{definition}
This means that for $p:\Fsf\to\Ssf$ a fibered category, objects of $\Fsf$ can be pulled back along any morphism in $\Ssf$ and that these pullbacks are unique up to unique isomorphism. Naturally we want these fibered functors to amalgamate into the data of a (possibly higher) category. To that end, we define morphisms of fibered categories as follows. 
\begin{definition}[Morphism of Fibered Categories]\label{def: morphism of fibered categories}
    Let $p:\Fsf\to\Ssf, q:\Gsf\to\Ssf$ be categories fibered over $\Ssf$. A morphism of fibered categories is a functor $F:\Fsf\to\Gsf$ such that $q\circ F=p$ and $F$ takes Cartesian morphisms in $\Fsf$ to Cartesian morphisms in $\Gsf$. 
\end{definition}
The language of fibered categories should be quite suggestive of analogous concepts in algebraic topology. Given a topological space $B$ and $E\to B$ a fibration, one can consider $E_{b}$ the fiber over $b\in B$. In the context of fibered categories, we define the following. 
\begin{definition}[Fiber]\label{def: fiber}
    Let $p:\Fsf\to\Ssf$ be a fibered category. For $X\in\Obj(\Ssf)$ the fiber $\Fsf_{X}$ is the subcategory of $\Fsf$ whose objects are those lying over $X$ and whose morphisms are those lying over $\id_{X}$. 
\end{definition}
For a morphism of fibered categories $F$ as in \Cref{def: morphism of fibered categories}, the functor $F$ sends $\Fsf_{X}$ to $\Gsf_{X}$ and hence restricts to a subfunctor $F_{X}:\Fsf_{X}\to\Gsf_{X}$. Let us consider how this definition of a fiber is reasonably compatible with the pullback structure we have previously discussed. 
\\\\
Let $\Fsf$ be fibered over $\Ssf$ and $f:A\to B$ a morphism in $\Ssf$. For each $\beta$ lying over $B$ in $\Fsf$ we choose a pullback $\phi_{\beta}:f^{*}\beta\to\beta$ with $\phi_{\beta}$ lying over $f$ and $f^{*}\beta$ lying over $A$ we can define a functor on the fiber categories $f^{*}:\Fsf_{B}\to\Fsf_{A}$ by defining a map on objects $\beta\mapsto f^{*}\beta$ and for $\tau:\beta\to\beta'$ over $\id_{B}$ in $\Fsf_{B}$ the unique map $f^{*}\tau:f^{*}\beta\to f^{*}\beta'$ that makes the diagram 
$$% https://q.uiver.app/#q=WzAsNCxbMCwwLCJmXnsqfVxcYmV0YSJdLFswLDEsImZeeyp9XFxiZXRhJyJdLFsyLDAsIlxcYmV0YSJdLFsyLDEsIlxcYmV0YSciXSxbMCwyLCJcXHBoaV97XFxiZXRhfSJdLFsxLDMsIlxccGhpX3tcXGJldGEnfSIsMl0sWzAsMSwiZl57Kn1cXHRhdSIsMix7InN0eWxlIjp7ImJvZHkiOnsibmFtZSI6ImRhc2hlZCJ9fX1dLFsyLDMsIlxcdGF1Il1d
\begin{tikzcd}
	{f^{*}\beta} && \beta \\
	{f^{*}\beta'} && {\beta'}
	\arrow["{\phi_{\beta}}", from=1-1, to=1-3]
	\arrow["{\phi_{\beta'}}"', from=2-1, to=2-3]
	\arrow["{f^{*}\tau}"', dashed, from=1-1, to=2-1]
	\arrow["\tau", from=1-3, to=2-3]
\end{tikzcd}$$
commute. 
\begin{definition}[Cleavage]\label{def: cleavage}
    Let $p:\Fsf\to\Ssf$ be a fibered category. A cleavage of $\Fsf$ is a collection of Cartesian morphisms $K$ of $\Fsf$ such that for each $f:A\to B$ in $\Ssf$ and $\beta\in\Obj(\Fsf_{B})$ there is $\alpha\in\Fsf_{A}$ such that $p(\alpha)=A$ and a unique Cartesian morphism $\phi:\alpha\to\beta$ in $K$ lying over $f$. 
\end{definition}
Visually we have the following diagram 
$$% https://q.uiver.app/#q=WzAsNCxbMCwwLCJcXGFscGhhIl0sWzIsMCwiXFxiZXRhIl0sWzAsMSwiQSJdLFsyLDEsIkIiXSxbMiwzLCJcXHBoaSIsMl0sWzAsMSwiZiJdLFswLDIsInAiLDJdLFsxLDMsInAiXV0=
\begin{tikzcd}
	\alpha && \beta \\
	A && B
	\arrow["f"', from=2-1, to=2-3]
	\arrow["\phi", from=1-1, to=1-3]
	\arrow["p"', from=1-1, to=2-1]
	\arrow["p", from=1-3, to=2-3]
\end{tikzcd}$$
where $\phi$ is a Cartesian morphism in $K$. 
\subsection{Higher Functors and the 2-Category $\Cat$}\label{subsec: higher functors}
By the axiom of choice, every fibered category has a cleavage. For $p:\Fsf\to\Ssf$ a fibered category and each $f:A\to B$ there is a functor $f^{*}:\Fsf_{B}\to\Fsf_{A}$ between the fibers. However, there are several fundamental issues. Firstly, the pullback along identities $\id_{A}^{*}:\Fsf_{A}\to\Fsf_{A}$ need not be identities; secondly, the category of categories does not form a category itself, but instead a 2-category $\Cat$ -- the 2-category $\Cat$ then would have objects categories, morphisms functors between these categories, and 2-morphisms natural transformations between functors. Instead of a functor, we get a lax 2-functor, which we now define. 
\newpage
\begin{definition}[Lax 2-Functor]\label{def: lax 2 functor}
    Let $\Ssf$ be a category. A lax 2-functor $\Phi$ on $\Ssf$ consists of the following data. 
    \begin{enumerate}[label=(\alph*)]
        \item For each $X\in\Obj(\Ssf)$ a category $\Phi\left(X\right)$.
        \item For each $(f:X\to Y)\in\Mor_{\Ssf}$ a functor $f^{*}:\Phi\left(Y\right)\to\Phi\left(X\right)$. 
        \item For each $X\in\Obj(\Csf)$ a natural isomorphism $\varepsilon_{X}:\id_{X}^{*}\Longrightarrow \id_{\Phi\left(X\right)}$ between functors $\Phi\left(X\right)\to\Phi\left(X\right)$. 
        \item For $(f:X\to Y), (g:Y\to Z)\in\Mor_{\Ssf}$ a natural isomorphism $\alpha_{f,g}:f^{*}g^{*}\Longrightarrow (fg)^{*}$ between functors $\Phi\left(X\right)\to\Phi\left(Z\right)$ such that the following conditions hold:
        \begin{enumerate}[label=(\roman*)]
            \item For $\beta\in\Obj\left(\Phi\left(Y\right)\right)$ we have 
            $$\alpha_{\id_{X},f}(\beta)=\varepsilon_{X}\left(f^{*}\beta\right):\id_{X}^{*}f^{*}\beta\to f^{*}\beta$$
            and
            $$\alpha_{f,\id_{Y}}(\beta)=f^{*}\varepsilon_{Y}:f^{*}\id_{Y}^{*}\beta\to f^{*}\beta.$$
            \item If also we have $(h:Z\to W)\in\Mor_{\Ssf}$ and $\gamma\in\Obj\left(\Phi\left(W\right)\right)$ the diagram 
            $$% https://q.uiver.app/#q=WzAsNCxbMCwwLCJmXnsqfWdeeyp9aF57Kn1cXGdhbW1hIl0sWzIsMCwiKGdmKV57Kn1oXnsqfVxcZ2FtbWEiXSxbMCwxLCJmXnsqfShnaCleeyp9XFxnYW1tYSJdLFsyLDEsIihoZ2YpXnsqfVxcZ2FtbWEiXSxbMCwyLCJmXnsqfVxcYWxwaGFfe2csaH0oXFxnYW1tYSkiLDJdLFsxLDMsIlxcYWxwaGFfe2dmLGh9KFxcZ2FtbWEpIl0sWzAsMSwiXFxhbHBoYV97ZixnfShoXnsqfVxcZ2FtbWEpIl0sWzIsMywiXFxhbHBoYV97ZixoZ30oXFxnYW1tYSkiLDJdXQ==
            \begin{tikzcd}
                {f^{*}g^{*}h^{*}\gamma} && {(gf)^{*}h^{*}\gamma} \\
                {f^{*}(gh)^{*}\gamma} && {(hgf)^{*}\gamma}
                \arrow["{f^{*}\alpha_{g,h}(\gamma)}"', from=1-1, to=2-1]
                \arrow["{\alpha_{gf,h}(\gamma)}", from=1-3, to=2-3]
                \arrow["{\alpha_{f,g}(h^{*}\gamma)}", from=1-1, to=1-3]
                \arrow["{\alpha_{f,hg}(\gamma)}"', from=2-1, to=2-3]
            \end{tikzcd}$$
            commutes. 
        \end{enumerate}
    \end{enumerate}
\end{definition}
This construction of a lax 2-functor rigidifies the construction of a (higher) functor into $\Cat$, the category of categories. Indeed, \emph{a priori}, even if it is natural to do so, there is neither a reason why the functor induced by pullback along $(\id_{X}:X\to X)\in\Mor_{\Ssf}$ induces an equivalence of categories $\id_{\Phi\left(X\right)}:\Phi\left(X\right)\to\Phi\left(X\right)$ nor is there a reason why for $(f:X\to Y),(g:Y\to Z)\in\Mor_{\Ssf}$ the functors 
$$% https://q.uiver.app/#q=WzAsMyxbMCwwLCJcXFBoaVxcbGVmdChaXFxyaWdodCkiXSxbMiwwLCJcXFBoaVxcbGVmdChZXFxyaWdodCkiXSxbNCwwLCJcXFBoaVxcbGVmdChaXFxyaWdodCkiXSxbMCwxLCJnXnsqfSJdLFsxLDIsImZeeyp9Il1d
\begin{tikzcd}
	{\Phi\left(Z\right)} && {\Phi\left(Y\right)} && {\Phi\left(Z\right)}
	\arrow["{g^{*}}", from=1-1, to=1-3]
	\arrow["{f^{*}}", from=1-3, to=1-5]
\end{tikzcd}$$
necessarily agrees with the composition
$$% https://q.uiver.app/#q=WzAsMixbMCwwLCJcXFBoaVxcbGVmdChaXFxyaWdodCkiXSxbNCwwLCJcXFBoaVxcbGVmdChaXFxyaWdodCkiXSxbMCwxLCIoZmcpXnsqfSJdXQ==
\begin{tikzcd}
	{\Phi\left(Z\right)} &&&& {\Phi\left(Z\right)}.
	\arrow["{(fg)^{*}}", from=1-1, to=1-5]
\end{tikzcd}$$
This makes the diagram
$$% https://q.uiver.app/#q=WzAsMyxbMCwwLCJcXFBoaVxcbGVmdChYXFxyaWdodCkiXSxbMiwwLCJcXFBoaVxcbGVmdChaXFxyaWdodCkiXSxbMSwxLCJcXFBoaVxcbGVmdChZXFxyaWdodCkiXSxbMSwyLCJnXnsqfSJdLFsyLDAsImZeeyp9Il0sWzEsMCwiKGZnKV57Kn0iLDJdLFs1LDIsIlxcYWxwaGFfe2YsZ30iLDEseyJsYWJlbF9wb3NpdGlvbiI6NjAsInNob3J0ZW4iOnsic291cmNlIjoyMH0sInN0eWxlIjp7InRhaWwiOnsibmFtZSI6ImFycm93aGVhZCJ9LCJoZWFkIjp7Im5hbWUiOiJub25lIn19fV1d
\begin{tikzcd}
	{\Phi\left(X\right)} && {\Phi\left(Z\right)} \\
	& {\Phi\left(Y\right)}
	\arrow["{g^{*}}", from=1-3, to=2-2]
	\arrow["{f^{*}}", from=2-2, to=1-1]
	\arrow[""{name=0, anchor=center, inner sep=0}, "{(fg)^{*}}"', from=1-3, to=1-1]
	\arrow["{\alpha_{f,g}}"{description, pos=0.6}, shorten <=3pt, Rightarrow, 2tail reversed, no head, from=0, to=2-2]
\end{tikzcd}$$
rigid in the sense that the composition is unique. 
\begin{remark}
    The astute would note this is the definition of a contravariant lax 2-functor. We omit the more general definitions for ease of exposition. 
\end{remark}
\begin{remark}
    Those familiar with higher categorical language will note that the exposition above is equivalent to the 2-category of categories $\Cat$ being a strict 2-category, in the sense that all coherence data above the level of 2-morphisms are rigid. 
\end{remark}
One would expect a fibered category $p:\Fsf\to\Ssf$ equipped with a cleavage gives rise to a lax 2-functor -- the cleavage here gives uniqueness of the functor up to a choice of preimage. This is indeed the case as we now show. 
\newpage
\begin{lemma}\label{lem: fibered cat with cleavage is lax2}
    Let $p:\Fsf\to\Ssf$ be a fibered category. $p$ defines a lax 2-functor to the category of categories associating to each object $A\in\Obj(\Ssf)$ a category
    $$A\mapsto \Fsf_{A}$$
    and to each morphism a functor
    $$(f:A\to B)\mapsto (f^{*}:\Fsf_{B}\to\Fsf_{A}).$$
\end{lemma}
\begin{proof}
    Evidently (a) and (b) are fulfilled. (c) follows from the fact that pullbacks are unique up to unique isomorphism so $\id_{A}^{*}:\Fsf_{A}\to\Fsf_{A}$ is naturally isomorphic to the identity functor. For (d) any functor completing the solid diagram 
    $$% https://q.uiver.app/#q=WzAsMyxbMCwwLCJcXEZzZl97WH0iXSxbMiwwLCJcXEZzZl97Wn0iXSxbMSwxLCJcXEZzZl97WX0iXSxbMSwyLCJnXnsqfSJdLFsyLDAsImZeeyp9Il0sWzEsMCwiIiwyLHsic3R5bGUiOnsiYm9keSI6eyJuYW1lIjoiZGFzaGVkIn19fV1d
    \begin{tikzcd}
        {\Fsf_{A}} && {\Fsf_{C}} \\
        & {\Fsf_{B}}
        \arrow["{g^{*}}", from=1-3, to=2-2]
        \arrow["{f^{*}}", from=2-2, to=1-1]
        \arrow[dashed, from=1-3, to=1-1]
    \end{tikzcd}$$
    it is unique since $\id_{\Fsf_{A}}:\Fsf_{A}\to\Fsf_{A}$ is Cartesian. 
    \\\\
    (d)(i) is given by the commutativity of the following diagram 
    $$% https://q.uiver.app/#q=WzAsMTIsWzAsMCwiXFxpZF97QX1eeyp9Zl57Kn1cXGJldGEiXSxbMCwxLCJBIl0sWzIsMiwiZl57Kn1cXGJldGEiXSxbMiwzLCJBIl0sWzQsMSwiXFxiZXRhIl0sWzQsMiwiQiJdLFs2LDAsImZeeyp9XFxpZF97Qn1eeyp9XFxiZXRhIl0sWzYsMSwiQSJdLFs4LDIsIlxcaWRfe0J9XnsqfVxcYmV0YSJdLFs4LDMsIkIiXSxbMTAsMSwiXFxiZXRhIl0sWzEwLDIsIkIiXSxbNCw1XSxbMiwzXSxbMyw1LCJmIiwxXSxbMSw1LCJmIiwxLHsibGFiZWxfcG9zaXRpb24iOjYwLCJjdXJ2ZSI6LTJ9XSxbMSwzLCJcXGlkX3tBfSIsMV0sWzIsMCwiXFxpZF97QX1eeyp9IiwxXSxbMCwxXSxbNCwwLCJcXGlkX3tcXEZzZl97QX19Zl57Kn0iLDEseyJjdXJ2ZSI6Mn1dLFs0LDIsImZeeyp9IiwxXSxbNyw5LCJmIiwxXSxbOCw2LCJmXnsqfSIsMSx7ImxhYmVsX3Bvc2l0aW9uIjozMH1dLFs2LDddLFs4LDldLFs5LDExLCJcXGlkX3tCfSIsMV0sWzEwLDgsIlxcaWRfe0J9XnsqfSIsMSx7ImxhYmVsX3Bvc2l0aW9uIjo0MH1dLFsxMCw2LCJmXnsqfVxcaWRfe1xcRnNmX3tCfX0iLDEseyJjdXJ2ZSI6Mn1dLFsxMCwxMV0sWzcsMTEsImYiLDEseyJjdXJ2ZSI6LTJ9XSxbMjcsMjIsIiIsMSx7InNob3J0ZW4iOnsic291cmNlIjoyMCwidGFyZ2V0IjoyMH19XSxbMTksMTcsIiIsMSx7InNob3J0ZW4iOnsic291cmNlIjoyMCwidGFyZ2V0IjoyMH19XV0=
    \begin{tikzcd}
        {\id_{A}^{*}f^{*}\beta} &&&&&& {f^{*}\id_{B}^{*}\beta} \\
        A &&&& \beta && A &&&& \beta \\
        && {f^{*}\beta} && B &&&& {\id_{B}^{*}\beta} && B \\
        && A &&&&&& B
        \arrow[from=2-5, to=3-5]
        \arrow[from=3-3, to=4-3]
        \arrow["f"{description}, from=4-3, to=3-5]
        \arrow["f"{description, pos=0.6}, curve={height=-12pt}, from=2-1, to=3-5]
        \arrow["{\id_{A}}"{description}, from=2-1, to=4-3]
        \arrow[""{name=0, anchor=center, inner sep=0}, "{\id_{A}^{*}\cong\id_{\Fsf_{A}}}"{description}, from=3-3, to=1-1]
        \arrow[from=1-1, to=2-1]
        \arrow[""{name=1, anchor=center, inner sep=0}, "{\id_{A}^{*}f^{*}}"{description}, curve={height=12pt}, from=2-5, to=1-1]
        \arrow["{f^{*}}"{description}, from=2-5, to=3-3]
        \arrow["f"{description}, from=2-7, to=4-9]
        \arrow[""{name=2, anchor=center, inner sep=0}, "{f^{*}}"{description, pos=0.3}, from=3-9, to=1-7]
        \arrow[from=1-7, to=2-7]
        \arrow[from=3-9, to=4-9]
        \arrow["{\id_{B}}"{description}, from=4-9, to=3-11]
        \arrow["{\id_{B}^{*}\cong\id_{\Fsf_{B}}}"{description, pos=0.4}, from=2-11, to=3-9]
        \arrow[""{name=3, anchor=center, inner sep=0}, "{f^{*}\id_{B}^{*}}"{description}, curve={height=12pt}, from=2-11, to=1-7]
        \arrow[from=2-11, to=3-11]
        \arrow["f"{description}, curve={height=-12pt}, from=2-7, to=3-11]
        \arrow[shorten <=8pt, shorten >=8pt, Rightarrow, from=3, to=2]
        \arrow[shorten <=8pt, shorten >=8pt, Rightarrow, from=1, to=0]
    \end{tikzcd}$$
    \\\\
    To verify (d)(ii) with the setup $f:A\to B, g:B\to C, h:C\to D$ we have $f^{*}g^{*}h^{*}\delta$ and $(hgf)^{*}\delta$ both pullbacks of some $\delta\in\Obj\left(\Fsf_{D}\right)$ but there is a unique morphism in $\Fsf_{A}$ lying over $\id_{A}$ such that the diagram 
    $$% https://q.uiver.app/#q=WzAsNCxbMCwwLCJmXnsqfWdeeyp9aF57Kn1cXGRlbHRhIl0sWzAsMSwiQSJdLFsyLDEsIkEiXSxbMiwwLCIoaGdmKV57Kn1cXGRlbHRhIl0sWzEsMiwiXFxpZF97QX0iLDJdLFswLDNdLFswLDEsInAiLDJdLFszLDIsInAiXV0=
    \begin{tikzcd}
        {f^{*}g^{*}h^{*}\delta} && {(hgf)^{*}\delta} \\
        A && A
        \arrow["{\id_{A}}"', from=2-1, to=2-3]
        \arrow[from=1-1, to=1-3]
        \arrow["p"', from=1-1, to=2-1]
        \arrow["p", from=1-3, to=2-3]
    \end{tikzcd}$$
    commutes by the definition of Cartesian arrows and the natural isomorphisms in the following diagram
    $$% https://q.uiver.app/#q=WzAsNSxbMCwwLCJmXnsqfWdeeyp9aF57Kn1cXGRlbHRhIl0sWzAsMiwiZl57Kn0oaGcpXnsqfVxcZGVsdGEiXSxbMywwLCIoZ2YpXnsqfWheeyp9XFxkZWx0YSJdLFszLDIsIihoZ2YpXnsqfVxcZGVsdGEiXSxbMSwyXSxbMCwyLCJcXGFscGhhX3tmLGd9KGheeyp9XFxkZWx0YSkiXSxbMiwzLCJcXGFscGhhX3tnZixofShcXGRlbHRhKSJdLFsxLDMsIlxcYWxwaGFfe2YsaGd9KFxcZGVsdGEpIiwyXSxbMCwxLCJmXnsqfVxcYWxwaGFfe2csaH0oXFxkZWx0YSkiLDJdLFswLDNdLFsyLDksIiIsMSx7InNob3J0ZW4iOnsidGFyZ2V0IjoyMH19XSxbMSw5LCIiLDEseyJzaG9ydGVuIjp7InRhcmdldCI6MjB9fV1d
    \begin{tikzcd}
        {f^{*}g^{*}h^{*}\delta} &&& {(gf)^{*}h^{*}\delta} \\
        \\
        {f^{*}(hg)^{*}\delta} & {} && {(hgf)^{*}\delta}
        \arrow["{\alpha_{f,g}(h^{*}\delta)}", from=1-1, to=1-4]
        \arrow["{\alpha_{gf,h}(\delta)}", from=1-4, to=3-4]
        \arrow["{\alpha_{f,hg}(\delta)}"', from=3-1, to=3-4]
        \arrow["{f^{*}\alpha_{g,h}(\delta)}"', from=1-1, to=3-1]
        \arrow[""{name=0, anchor=center, inner sep=0}, from=1-1, to=3-4]
        \arrow[shorten >=8pt, Rightarrow, from=1-4, to=0]
        \arrow[shorten >=8pt, Rightarrow, from=3-1, to=0]
    \end{tikzcd}$$
    as desired. 
\end{proof}
Under special circumstances, a lax 2-functor from a category $\Ssf$ to the category of categories can be made into a functor. 
\begin{definition}[Splitting Cleavage]\label{def: split cleavage}
    Let $p:\Fsf\to\Ssf$ be a fibered category with cleavage $K$. The cleavage $K$ is splitting if it contains all identities and is closed under composition. 
\end{definition}
One then shows the following. 
\begin{lemma}\label{lem: lax2 is functor iff split cleavage}
    The lax 2-functor associated to the cleavage is a functor if and only if the cleavage is splitting. 
\end{lemma}
\begin{proof}
    If the cleavage is splitting then the composition law is rigid with unitality and associativity by \Cref{def: lax 2 functor}.
\end{proof}
\begin{remark}
    Generally, fibered categories do not admit a splitting. 
\end{remark}
The converse to \Cref{lem: fibered cat with cleavage is lax2} also holds. That is, given a fibered category with a lax 2-functor, one can construct a cleavage.
\begin{lemma}\label{lem: lax2 fibered cat has cleavage}
   Suppose $p:\Fsf\to\Ssf$ is a fibered category and $\Phi$ some lax 2-functor on $\Ssf$ given on objects by 
   $$A\mapsto\Fsf_{A}$$
   and on morphisms by 
   $$(f:A\to B)\mapsto (f^{*}:\Fsf_{B}\to\Fsf_{A}).$$
   There exists a cleavage $K$ realizing $\Phi$.  
\end{lemma} 
Evidently this results in the following theorem. 
\begin{theorem}\label{thm: lax2 iff cleavage}
    Let $p:\Fsf\to\Csf$ be a fibered category. The following are equivalent. 
    \begin{enumerate}[label=(\alph*)]
        \item There is a lax 2-functor $\Phi:\Ssf\to\Cat$.
        \item The category $\Ssf$ has a cleavage $K$.
    \end{enumerate}
\end{theorem}
\begin{proof}
    Immediate from \Cref{lem: lax2 fibered cat has cleavage} and \Cref{lem: fibered cat with cleavage is lax2}.
\end{proof}
We list an additional property of fibered categories.
\begin{definition}[Stable Property of Morphisms]\label{def: stable property of morphisms}
    A collection of morphisms $\Psf\subseteq\Mor_{\Csf}$ of a category $\Csf$ is stable if the following two conditions hold:
    \begin{enumerate}[label=(\alph*)]
        \item If $f:A\to B$ is in $\Psf$ and $\phi:A'\to A,\psi:B'\to B$ are isomorphisms then the composition $(\psi\circ f\circ\phi):A'\to B'$ is in $\Psf$ as well. 
        \item Given $A\to B$ in $\Psf$ and any map $C\to B$ the fibered product 
        $$% https://q.uiver.app/#q=WzAsNCxbMCwwLCJBXFx0aW1lc197Qn1DIl0sWzAsMSwiQSJdLFsyLDEsIkIiXSxbMiwwLCJDIl0sWzMsMl0sWzEsMl0sWzAsM10sWzAsMV1d
        \begin{tikzcd}
            {A\times_{B}C} && C \\
            A && B
            \arrow[from=1-3, to=2-3]
            \arrow[from=2-1, to=2-3]
            \arrow[from=1-1, to=1-3]
            \arrow[from=1-1, to=2-1]
        \end{tikzcd}$$
        exists and $A\times_{B}C\to C$ is in $\Psf$. 
    \end{enumerate}
\end{definition}
\subsection{Categories Fibered in Groupoids}\label{subsec: categories fibered in groupoids}
We now turn to a very important example of fibered categories. Categories fibered in groupoids. Recall here the following definition from category theory. 
\begin{definition}[Groupoid]\label{def: groupoid}
    Let $\Csf$ be a category. $\Csf$ is a groupoid if and only if every morphism in $\Csf$ is an isomorphism. 
\end{definition}
Naturally, we define a category fibered in groupoids as follows. 
\begin{definition}[Category Fibered in Groupoids]\label{def: category fibered in groupoids}
    Let $p:\Fsf\to\Ssf$ be a fibered category. $p:\Fsf\to\Ssf$ is a category fibered in groupoids if for all $A\in\Obj(\Ssf)$ the category $\Fsf_{A}$ is a groupoid. 
\end{definition}
One can alternatively characterize categories fibered in groupoids as follows. 
\begin{lemma}\label{lem: cfg iff cartesian plus}
    Let $p:\Fsf\to\Ssf$ be a fibered category. $p:\Fsf\to\Ssf$ is fibered in groupoids if and only if the following two properties hold:
    \begin{enumerate}[label=(\alph*)]
        \item Every arrow in $\Fsf$ is Cartesian. 
        \item Given $\beta\in\Obj(\Fsf)$ and $f:A\to p(\beta)$ in $\Ssf$, there exists $\phi:\alpha\to\beta$ in $\Fsf$ such that $p(\phi)=f$. 
    \end{enumerate}
\end{lemma}
\begin{proof}
    $(\Longrightarrow)$ Suppose $p:\Fsf\to\Ssf$ is a category fibered in groupoids. Lifts exist and are unique by $\id_{\Fsf_{B}}$ Cartesian satisfying (ii). For (i), let $\phi:\alpha\to\beta$ lying over $f:A\to B$ and $\phi':\alpha'\to\beta$ a pullback of $\beta\in\Obj(\Fsf_{B})$ to $\Fsf_{A}$, there is a map $\eta:\alpha\to\alpha'$ in $\Fsf_{A}$ which is an isomorphism since $p:\Fsf\to\Ssf$ is a category fibered in groupoids. But isomorphisms are unique so $\phi$ is Cartesian. 
    \\\\
    $(\Longleftarrow)$ Suppose conditions (a) and (b) hold and let $\phi:\alpha'\to\alpha$ be a morphism $\Fsf_{A}$ over $A\in\Obj(\Ssf)$. Since this morphism $\phi$ is Cartesian there is $\psi:\alpha\to\alpha'$ filling in the following diagram. 
    $$% https://q.uiver.app/#q=WzAsNixbMiwxLCJcXGFscGhhJyJdLFsyLDIsIkEiXSxbNCwxLCJcXGFscGhhIl0sWzQsMiwiQSJdLFswLDAsIlxcYWxwaGEiXSxbMCwxLCJBIl0sWzUsMSwiXFxpZF97QX0iLDFdLFs0LDAsIlxccHNpIiwxLHsic3R5bGUiOnsiYm9keSI6eyJuYW1lIjoiZGFzaGVkIn19fV0sWzAsMV0sWzQsMiwiXFxpZF97XFxhbHBoYX0iLDFdLFs1LDMsIlxcaWRfe0F9IiwxLHsibGFiZWxfcG9zaXRpb24iOjYwfV0sWzQsNV0sWzIsM10sWzAsMiwiXFxwaGkiLDFdLFsxLDMsIlxcaWRfe0F9IiwxXV0=
    \begin{tikzcd}
        \alpha \\
        A && {\alpha'} && \alpha \\
        && A && A
        \arrow["{\id_{A}}"{description}, from=2-1, to=3-3]
        \arrow["\psi"{description}, dashed, from=1-1, to=2-3]
        \arrow[from=2-3, to=3-3]
        \arrow["{\id_{\alpha}}"{description}, from=1-1, to=2-5]
        \arrow["{\id_{A}}"{description, pos=0.6}, from=2-1, to=3-5]
        \arrow[from=1-1, to=2-1]
        \arrow[from=2-5, to=3-5]
        \arrow["\phi"{description}, from=2-3, to=2-5]
        \arrow["{\id_{A}}"{description}, from=3-3, to=3-5]
    \end{tikzcd}$$
    By the commutativity of the upper triangle, $\phi\circ\psi\Longrightarrow\id_{\alpha}$ so $\psi$ is a right inverse of $\phi$ and by the diagram 
    $$% https://q.uiver.app/#q=WzAsNixbMiwxLCJcXGFscGhhIl0sWzIsMiwiQSJdLFs0LDEsIlxcYWxwaGEnIl0sWzQsMiwiQSJdLFswLDAsIlxcYWxwaGEnIl0sWzAsMSwiQSJdLFs1LDEsIlxcaWRfe0F9IiwxXSxbNCwwLCJcXHBoaSIsMSx7InN0eWxlIjp7ImJvZHkiOnsibmFtZSI6ImRhc2hlZCJ9fX1dLFswLDFdLFs0LDIsIlxcaWRfe1xcYWxwaGEnfSIsMV0sWzUsMywiXFxpZF97QX0iLDEseyJsYWJlbF9wb3NpdGlvbiI6NjB9XSxbNCw1XSxbMiwzXSxbMCwyLCJcXHBzaSIsMV0sWzEsMywiXFxpZF97QX0iLDFdXQ==
    \begin{tikzcd}
        {\alpha'} \\
        A && \alpha && {\alpha'} \\
        && A && A
        \arrow["{\id_{A}}"{description}, from=2-1, to=3-3]
        \arrow["\phi"{description}, dashed, from=1-1, to=2-3]
        \arrow[from=2-3, to=3-3]
        \arrow["{\id_{\alpha'}}"{description}, from=1-1, to=2-5]
        \arrow["{\id_{A}}"{description, pos=0.6}, from=2-1, to=3-5]
        \arrow[from=1-1, to=2-1]
        \arrow[from=2-5, to=3-5]
        \arrow["\psi"{description}, from=2-3, to=2-5]
        \arrow["{\id_{A}}"{description}, from=3-3, to=3-5]
    \end{tikzcd}$$
    so too is $\phi$ the right inverse of $\psi$, showing $\phi$ is an isomorphism. 
\end{proof}
We now consider the higher categorical variant of Yoneda's lemma. Recall that we have seen how a category $\Csf$ can be embedded into the functor category $\Fun(\Csf^{\Opp},\Sets)$, equivalently described as the category of set-valued presheaves on $\Csf$. One can similarly define for a fibered category $p:\Fsf\to\Ssf$ an embedding $\Fun(\Ssf^{\Opp},\Sets)$ into the 2-category of fibered categories over $\Ssf$ denoted $\FibCat(\Ssf)$ by taking for $F\in\Obj\left(\Fun(\Ssf^{\Opp},\Sets)\right)$ defining the fiber to be the collection of objects
$$\Fsf_{A}=\left\{(A,\alpha)\in\Obj(\Ssf)\times F(A)\right\}$$
and morphisms $f^{*}:\Fsf_{B}\to\Fsf_{A}$ such that 
$$f^{*}\left((B,F(f(\alpha)))\right)=(A,\alpha).$$
By composing these embeddings, we have an embedding of a category $\Csf$ to the 2-category of categories fibered over $\Csf$, $\FibCat(\Csf)$. For a category $\Csf$, the map on objects takes $Z\in\Obj(\Csf)$ to the slice category $\Csf_{(-/Z)}$ whose objects are morphisms $X\to Z$ and whose morphisms are commutative triangles 
$$% https://q.uiver.app/#q=WzAsMyxbMCwwLCJYIl0sWzIsMCwiWSJdLFsxLDEsIloiXSxbMCwyXSxbMSwyXSxbMCwxLCJmIl1d
\begin{tikzcd}
	X && Y \\
	& Z
	\arrow[from=1-1, to=2-2]
	\arrow[from=1-3, to=2-2]
	\arrow["f", from=1-1, to=1-3]
\end{tikzcd}$$
which we denote $f$, taking the morphisms to $Z$, known as the structure morphisms, as implicit. One yields a map $\Csf_{(-/Z)}\to\Csf$ by $(X\to Z)\mapsto X$, compatible with morphisms in the obvious way. 
\begin{definition}[Representable Fibered Category]\label{def: representable fibered category}
   A fibered category over $\Csf$ is representable if it is equivalent as a category to a slice category $\Csf_{(-/Z)}$ for some $Z\in\Obj(\Csf)$. 
\end{definition}
We now state and prove the 2-categorical Yoneda lemma. 
\begin{lemma}[2-Categorical Yoneda]\label{lem: 2cat Yoneda}
    Let $p:\Fsf\to\Csf$ be a fibered category and $Z\in\Obj(\Csf)$. There is an equivalence of categories 
    $$\Fun\left(\Csf_{(-/Z)},\Fsf\right)\to\Fsf_{Z}.$$
\end{lemma}
\section{Descent}\label{sec: descent}
Descent generalizes the identity and gluing axioms for sheaves on topological spaces to the setting of sites. We have already seen that for $\Ssf$ a site, we should think of a fibered category over $\Ssf$ as a lax 2-functor in the sense of \Cref{def: lax 2 functor}, that is, as a presheaf of categories on $\Ssf$. A stack, which we will soon encounter, is a sheaf of categories over $\Ssf$. 
\subsection{Descent in Fibered Categories}\label{subsec: descent in fibered categories}
Let $p:\Fsf\to\Ssf$ be a category fibered over a site $\Ssf$ with a fixed cleavage $K$. For a covering $\{X_{i}\to X\}$, denote $X_{ij}=X_{i}\times_{X}X_{j}$ and $X_{ijk}=X_{i}\times_{X}X_{j}\times_{X}X_{k}$. We now define an object with descent data. 
\begin{definition}[Object With Descent Data]\label{def: object w descent data}
    Let $\{X_{i}\to X\}$ be a covering in $\Ssf$. An object with descent data on $\{X_{i}\to X\}$ is the tuple $(\{\xi_{i}\},\{\phi_{ij}\})$ with objects $\xi_{i}\in\Fsf_{X_{i}}$ and isomorphisms $\phi_{ij}:\pr_{2}^{*}\xi_{j}\to\pr_{1}^{*}\xi_{i}$ in $\Fsf_{X_{ij}}$ such that 
    $$\pr_{1,3}^{*}\phi_{ik}=\pr_{1,2}^{*}\phi_{ij}\circ\pr_{2,3}^{*}\phi_{jk}:\pr_{3}^{*}\xi_{k}\to\pr_{1}^{*}\xi_{i}.$$
\end{definition}
\begin{remark}
    The maps $\phi_{ij}$ are known as transition isomorphisms of the object $X$ with descent data. 
\end{remark}
One naturally defines morphisms of objects with descent data as follows. 
\begin{definition}[Morphisms of Objects with Descent Data]
    A morphism between objects with descent data on $\{X_{i}\to X\}$, $(\{\xi_{i}\},\{\phi_{ij}\})$ and $(\{\upsilon_{i}\},\{\psi_{i,j}\})$, is a tuple $\{\alpha_{i}\}$ with $\alpha_{i}:\Fsf_{X_{i}}\to\Fsf_{X_{i}}$ where $\xi_{i}\mapsto\upsilon_{i}$ such that for each pair $i,j$ the diagram 
    $$% https://q.uiver.app/#q=WzAsNCxbMCwwLCJcXHByX3syfV57Kn1cXHhpX3tqfSJdLFswLDEsIlxccHJfezF9XnsqfVxceGlfe2l9Il0sWzIsMCwiXFxwcl97Mn1eeyp9XFx1cHNpbG9uX3tqfSJdLFsyLDEsIlxccHJfezF9XnsqfVxcdXBzaWxvbl97aX0iXSxbMSwzLCJcXHByX3sxfV57Kn1cXGFscGhhX3tpfSIsMl0sWzIsMywiXFxwc2lfe2lqfSJdLFswLDEsIlxccGhpX3tpan0iLDJdLFswLDIsIlxccHJfezJ9XnsqfVxcYWxwaGFfe2p9Il1d
    \begin{tikzcd}
        {\pr_{2}^{*}\xi_{j}} && {\pr_{2}^{*}\upsilon_{j}} \\
        {\pr_{1}^{*}\xi_{i}} && {\pr_{1}^{*}\upsilon_{i}}
        \arrow["{\pr_{1}^{*}\alpha_{i}}"', from=2-1, to=2-3]
        \arrow["{\psi_{ij}}", from=1-3, to=2-3]
        \arrow["{\phi_{ij}}"', from=1-1, to=2-1]
        \arrow["{\pr_{2}^{*}\alpha_{j}}", from=1-1, to=1-3]
    \end{tikzcd}$$
    commutes. 
\end{definition}
The morphisms $\alpha_{i}$ compose in the obvious way making objects with descent data the objects of a category $\Fsf_{\{X_{i}\to X\}}$. 
\\\\
More explicitly, let $\xi\in\Fsf_{X}$ and $\{\sigma_{i}:X_{i}\to X\}$ a covering. We can construct an object with descent data as follows. Set $\xi_{i}=\sigma_{i}^{*}\xi$ and transition isomorphisms the identity after identifying $\pr_{2}^{*}\sigma_{j}^{*}\xi$ and $\pr_{1}^{*}\sigma_{i}^{*}\xi$, the pullbacks of $\xi$ to $X_{ij}$. For some $\alpha:\xi\to\upsilon$ in $\Fsf_{X}$ we get $\alpha_{i}=\sigma_{i}^{*}\alpha:\sigma_{i}^{*}\xi\to\sigma_{i}^{*}\upsilon$, yielding a morphism of objects with descent data $\{\alpha_{i}\}$ from $(\{\xi_{i}\},\{\phi_{ij}\})$ to $(\{\upsilon_{i}\},\{\psi_{ij}\})$. Evidently this is a functor $\Fsf_{X}\to\Fsf_{\{X_{i}\to X\}}$ where on objects is given by 
$$\xi\mapsto(\{\xi_{i}\},\{\phi_{ij}\})$$
and on morphisms by 
$$(\alpha:\xi\to\upsilon)\mapsto \{\alpha_{i}:\xi_{i}\to\upsilon_{i}\}$$
which are compatible with the transition isomorphisms $\phi_{ij},\psi_{ij}$. 
\begin{remark}
    This construction does not depend on the cleavage in the sense that the categories $\Fsf_{\{X_{i}\to X\}}$ are equivalent regardless of the choice of cleavage.
\end{remark}
We can now define some long-awaited notions. 
\begin{definition}[Prestack]\label{def: categorical prestack}
    Let $p:\Fsf\to\Ssf$ be a category fibered over a site. $\Fsf$ is a prestack over $\Ssf$ if for each covering $\{X_{i}\to X\}$ in $\Ssf$ the functor $\Fsf_{X}\to\Fsf_{\{X_{i}\to X\}}$ is fully faithful. 
\end{definition}
\begin{definition}[Stack]\label{def: categorical stack}
    Let $p:\Fsf\to\Ssf$ be a category fibered over a site. $\Fsf$ is a stack over $\Ssf$ if for each covering $\{X_{i}\to X\}$ in $\Ssf$ the functor $\Fsf_{X}\to\Fsf_{\{X_{i}\to X\}}$ is an equivalence of categories. 
\end{definition}
Unpacking the definition, to be a prestack means that for $X\in\Obj(\Ssf)$; $\xi,\upsilon\in\Obj(\Fsf_{X})$; $\{X_{i}\to X\}$ a covering; $\xi_{i},\upsilon_{i}$ pullbacks of $\xi,\upsilon$ to the $X_{i}$; $\xi_{ij},\upsilon_{ij}$ pullbacks to $X_{ij}$ and there was some $a_{i}:\xi_{i}\to\upsilon_{i}$ in $\Fsf_{X_{i}}$ such that $\pr_{1}^{*}\alpha_{i}=\pr_{2}^{*}\alpha_{j}:\xi_{ij}\to\upsilon_{ij}$ then there is a unique $\alpha:\xi\to\upsilon$ in $\Fsf_{X}$ whose pullback is $\alpha_{i}:\xi_{i}\to\upsilon_{i}$ for all $i$. Clarifying what happens if $p:\Fsf\to\Ssf$ is a stack will require introducing the following notion. 
\begin{definition}[Effective Descent Data]\label{def: effective descent data}
    An object with descent data $(\{\xi_{i}\},\{\phi_{ij}\})$ in $\Fsf_{\{X_{i}\to X\}}$ is effective if it is isomorphic to the image of an object in $\Fsf_{X}$. 
\end{definition} 
In other words an object with descent data $(\{\xi_{i}\},\{\phi_{ij}\})$ in $\Fsf_{\{X_{i}\to X\}}$ is effective if there exists $\xi\in\Obj(\Fsf_{X})$ and Cartesian arrows $\xi_{i}\to\xi$ over $\sigma_{i}:X_{i}\to X$ such that the pentagonal diagram 
$$% https://q.uiver.app/#q=WzAsNixbMiwwLCJcXHhpIl0sWzEsMV0sWzAsMSwiXFx4aV97aX0iXSxbNCwxLCJcXHhpX3tqfSJdLFsxLDMsIlxccHJfezF9XnsqfVxceGlfe2l9Il0sWzMsMywiXFxwcl97Mn1eeyp9XFx4aV97an0iXSxbMiwwXSxbMywwXSxbNCwyXSxbNSwzXSxbNSw0LCJcXHBoaV97aWp9Il1d
\begin{tikzcd}
	&& \xi \\
	{\xi_{i}} & {} &&& {\xi_{j}} \\
	\\
	& {\pr_{1}^{*}\xi_{i}} && {\pr_{2}^{*}\xi_{j}}
	\arrow[from=2-1, to=1-3]
	\arrow[from=2-5, to=1-3]
	\arrow[from=4-2, to=2-1]
	\arrow[from=4-4, to=2-5]
	\arrow["{\phi_{ij}}", from=4-4, to=4-2]
\end{tikzcd}$$
commutes for all $i,j$.  
\\\\
If $p:\Fsf\to\Ssf$ is a stack as in \Cref{def: categorical stack}, we know the functor $\Fsf_{X}\to\Fsf_{\{X_{i}\to X\}}$ is fully faithful and essentially surjective. So all $(\{\xi_{i}\},\{\phi_{ij}\})$ are isomorphic to the image of some object in $\Fsf_{X}$, that is, all descent data is effective. 
\subsection{Descent for Torsors}\label{subsec: descent for torsors}
Torsors are closely related to principal $G$-bundles in algebraic topology, and are central objects in the study of both algebraic and arithmetic geometry. 
\begin{definition}[Torsor]\label{def: torsor}
    Let $\Ssf$ be a site and $\Bmu$ a sheaf of groups on $\Ssf$. A $\Bmu$-torsor on $\Ssf$ is a pair $(\Pcal,\rho)$ where $\Pcal$ is a sheaf on $\Ssf$ with a left action 
    $$\rho:\Bmu\times\Pcal\longrightarrow\Pcal$$
    such that the following conditions hold:
    \begin{enumerate}
        \item For all $X\in\Obj(\Ssf)$ there is a covering $\{X_{i}\to X\}_{i\in I}$ such that $\Pcal(X_{i})\neq\emptyset$ for all $i\in I$.
        \item The morphism of sheaves
        $$\Bmu\times\Pcal\to\Pcal\times\Pcal; (g,p)\mapsto(p,g\cdot p)$$
        is an isomorphism. 
    \end{enumerate}
\end{definition}
\begin{remark}
    The sheaf of groups $\Bmu$ need not be a sheaf of Abelian groups. 
\end{remark}
\begin{remark}
    The condition (b) in \Cref{def: torsor} says that for $\Pcal(X)$ the action of the group $\Bmu(X)$ on $\Pcal(X)$ is simply transitive: that it is both free -- the identity is the only element fixing any point -- and for any two sections $p,p'\in\Pcal(X)$ there is a unique $g\in \mu(X)$ such that $g\cdot p=p'$. 
\end{remark}
\begin{definition}[Trivial Torsor]\label{def: trivial torsor}
    Let $\Ssf$ be a site, $\Bmu$ a sheaf of groups on $\Ssf$, and $(\Pcal,\rho)$ a $\Bmu$-torsor on $\Ssf$ if $\Pcal$ has a global section. 
\end{definition}
If $(\Pcal,\rho)$ is a trivial $\Bmu$-torsor on a site $\Ssf$, then for a global section $p\in\Pcal(X)$ we have an isomorphism 
$$\Bmu\to\Pcal; g\mapsto g\cdot p$$
identifying $\Pcal$ with $\Bmu$ and the action $\rho$ with the endomorphism by left multiplication. Morphisms of torsors are defined in the natural way as follows. 
\begin{definition}[Morphism of Torsors]\label{def: morphism of torosrs}
    Let $\Ssf$ be a site, $\Bmu$ a sheaf of groups on $\Ssf$, and $(\Pcal,\rho),(\Pcal,\rho')$ two $\Bmu$-torsors on $\Ssf$. A morphism of torsors $f:(\Pcal',\rho')\to(\Pcal,\rho)$ is the data of a morphism of sheaves $f:\Pcal'\to\Pcal$ such that the diagram 
    $$% https://q.uiver.app/#q=WzAsNCxbMCwwLCJcXEJtdVxcdGltZXNcXFBjYWwnIl0sWzIsMCwiXFxCbXVcXHRpbWVzXFxQY2FsIl0sWzAsMSwiXFxQY2FsJyJdLFsyLDEsIlxcUGNhbCJdLFsyLDMsImYiLDJdLFswLDEsIlxcaWRfe1xcQm11fVxcdGltZXMgZiJdLFswLDIsIlxccmhvJyIsMl0sWzEsMywiXFxyaG8iXV0=
    \begin{tikzcd}
        {\Bmu\times\Pcal'} && \Bmu\times\Pcal \\
        {\Pcal'} && \Pcal
        \arrow["f"', from=2-1, to=2-3]
        \arrow["{\id_{\Bmu}\times f}", from=1-1, to=1-3]
        \arrow["{\rho'}"', from=1-1, to=2-1]
        \arrow["\rho", from=1-3, to=2-3]
    \end{tikzcd}$$
    commutes. 
\end{definition}
Torsors are intimately connected to principal $G$-bundles in algebraic geometry and algebraic topology. 
\begin{definition}[Principal $G$-Bundle]\label{def: principal G bundle}
    Let $(\Sch_{X})_{\fppf}$ denote the fppf site of $X$-schemes for $X$ a base scheme and $\Bmu$ a sheaf of groups on $(\Sch_{X})_{\fppf}$ representable by a flat locally finitely presented $X$-group scheme $G$. A principal $G$-bundle on $X$ is a pair $(\pi:P\to X,\rho)$ where $\pi:P\to X$ is a flat, locally finitely presented, surjective morphism of schemes and 
    $$\rho:G\times_{X}P\to P$$
    a morphism such that the following conditions hold:
    \begin{enumerate}[label=(\alph*)]
        \item The diagram 
        $$% https://q.uiver.app/#q=WzAsNCxbMCwwLCJHXFx0aW1lc197WH1HXFx0aW1lc197WH1QIl0sWzIsMCwiR1xcdGltZXNfe1h9UCJdLFswLDEsIkdcXHRpbWVzX3tYfVAiXSxbMiwxLCJQIl0sWzIsMywiXFxyaG8iLDJdLFsxLDMsIlxccmhvIl0sWzAsMSwiXFxpZF97R31cXHRpbWVzXFxyaG8iXSxbMCwyLCJtXFx0aW1lc1xcaWRfe1B9IiwyXV0=
        \begin{tikzcd}
            {G\times_{X}G\times_{X}P} && {G\times_{X}P} \\
            {G\times_{X}P} && P
            \arrow["\rho"', from=2-1, to=2-3]
            \arrow["\rho", from=1-3, to=2-3]
            \arrow["{\id_{G}\times\rho}", from=1-1, to=1-3]
            \arrow["{m\times\id_{P}}"', from=1-1, to=2-1]
        \end{tikzcd}$$
        commutes, here denoting $m:G\times_{X}G\to G$ the group operation on the group scheme $G$. 
        \item If $e:X\to G$ is the identity action, the composition 
        $$% https://q.uiver.app/#q=WzAsMyxbMCwwLCJQIl0sWzIsMCwiR1xcdGltZXNfe1h9UCJdLFs0LDAsIlAiXSxbMCwxLCJlXFxjaXJjXFxwaVxcdGltZXNcXGlkX3tQfSJdLFsxLDIsIlxccmhvIl1d
        \begin{tikzcd}
            P && {G\times_{X}P} && P
            \arrow["{e\circ\pi\times\id_{P}}", from=1-1, to=1-3]
            \arrow["\rho", from=1-3, to=1-5]
        \end{tikzcd}$$
        is the identity on $P$. 
        \item The map $(\rho,\pr_{2}):G\times_{X}P\to P\times_{X}P$ is an isomorphism 
    \end{enumerate}
\end{definition}
The morphisms of principal $G$-bundles are given in the obvious way. 
\begin{definition}[Morphisms of Principal $G$-Bundles]\label{def: morphism of principal G bundles}
    Let $(\Sch_{X})_{\fppf}$ denote the fppf site of $X$-schemes for $X$ a base scheme and $\Bmu$ a sheaf of groups on $(\Sch_{X})_{\fppf}$ representable by a flat locally finitely presented $X$-group scheme $G$. A morphism of principal $G$-bundles 
    $$f:(\pi':P'\to X,\rho')\to(\pi:P\to X,\rho)$$
    is a morphism of $X$-schemes $f:P'\to P$ such that the diagram 
    $$% https://q.uiver.app/#q=WzAsNCxbMCwwLCJHXFx0aW1lc197WH1QJyJdLFswLDEsIlAnIl0sWzIsMCwiR1xcdGltZXNfe1h9UCJdLFsyLDEsIlAiXSxbMSwzLCJmIiwyXSxbMCwyLCJcXGlkX3tHfVxcdGltZXMgZiJdLFswLDEsIlxccmhvJyIsMl0sWzIsMywiXFxyaG8iXV0=
    \begin{tikzcd}
        {G\times_{X}P'} && {G\times_{X}P} \\
        {P'} && P
        \arrow["f"', from=2-1, to=2-3]
        \arrow["{\id_{G}\times f}", from=1-1, to=1-3]
        \arrow["{\rho'}"', from=1-1, to=2-1]
        \arrow["\rho", from=1-3, to=2-3]
    \end{tikzcd}$$
    commutes. 
\end{definition}
Given a principal $G$-bundle $(\pi:P\to X,\rho)$ on $X$, we get a $\mu$-torsor bytaking $\Pcal$ to be the sheaf on $(\Sch_{X})_{\fppf}$ represented by the scheme $P$ with action that induced by $\rho$. The conditions (a) and (b) of \Cref{def: principal G bundle} imply that the map $\Bmu\times\Pcal\to\Pcal$ is an action while condition (c) imposes that the action is simply transitive. Furthermore, since $\pi:P\to X$ is flat, locally finitely presented, and surjective as a map of schemes, there exists an fppf cover $\{X_{i}\to X\}$ such that $\Pcal(X_{i})\neq\emptyset$ for all $i\in I$. 
\\\\
Furthermore, one can observe that these constructions are functorial, yielding a categories of principal $G$-bundles on a scheme $X$ and $\Bmu$-torsors on the fppf site of $X$-schemes $(\Sch_{X})_{\fppf}$. Indeed, Yoneda's lemma tells us that this functor is fully faithful that is essentially surjective on the condition that the structure morphism of the algebraic group $G$ is affine. 
\begin{proposition}\label{prop: if affine then G bundles are mu torsors}
    Let $(\Sch_{X})_{\fppf}$ denote the fppf site of $X$-schemes for $X$ a base scheme and $\Bmu$ a sheaf of groups on $(\Sch_{X})_{\fppf}$ representable by a flat locally finitely presented $X$-group scheme $G$. If the structure morphism $G\to X$ is an affine morphism then there is a equivalence between the category of principal $G$-bundles on $X$ and the category $\Bmu$-torsors on the fppf site of $X$-schemes $(\Sch_{X})_{\fppf}$.
\end{proposition}
In some simple cases, we can describe the category of torsors in a relatively concrete manner. 
\\\\
Consider the scheme $X$ and $n$ an integer invertible on the scheme $X$. Denote $\Bmu_{n}$ the group scheme such that 
$$\Bmu_{n}(S)=\left\{f\in\Ocal_{X}^{\times}:f^{n}=1\right\}.$$
The category of $\Bmu_{n}$-torsors $\mathsf{Tors}(\Bmu_{n})$ on the small \'{e}tale site of $X$ $X_{\et}$ can be described as follows: consider $\Sigma_{n}$ the category with objects pairs $(L,\sigma)$ where $L$ is an invertible sheaf on the scheme $X$ and $\sigma:L^{\otimes n}\to\Ocal_{X}$ a trivialization of the $n$-th power of $L$, considering $L$ as a sheaf in the \'{e}tale topology. The morphisms in $\Sigma_{n}$ between two objects $(L',\sigma')$ and $(L,\sigma)$ are morphisms of line bundles $\rho:L'\to L$ such that the diagram 
$$% https://q.uiver.app/#q=WzAsMyxbMCwwLCJMJ157XFxvdGltZXMgbn0iXSxbMiwwLCJMXntcXG90aW1lcyBufSJdLFsxLDEsIlxcT2NhbF97WH0iXSxbMCwyLCJcXHNpZ21hJyIsMl0sWzEsMiwiXFxzaWdtYSJdLFswLDEsIlxccmhvXntcXG90aW1lcyBufSJdXQ==
\begin{tikzcd}
	{L'^{\otimes n}} && {L^{\otimes n}} \\
	& {\Ocal_{X}}
	\arrow["{\sigma'}"', from=1-1, to=2-2]
	\arrow["\sigma", from=1-3, to=2-2]
	\arrow["{\rho^{\otimes n}}", from=1-1, to=1-3]
\end{tikzcd}$$
commutes. One can construct a functor 
$$F:\Sigma_{n}\longrightarrow\mathsf{Tors}(\Bmu_{n})$$
by associating to each $(L,\sigma)\in\Sigma_{n}$ a sheaf $\Pcal_{(L,\sigma)}$ a sheaf on the \'{e}tale site $X_{\et}$ such that for any $U\to X$ \'{e}tale $\Pcal_{(L,\sigma)}(U)$ is the set of trivializations $\lambda:\Ocal_{U}\to L|_{U}$ such that the composite 
$$% https://q.uiver.app/#q=WzAsMyxbMCwwLCJcXE9jYWxfe1V9Il0sWzIsMCwiTF57XFxvdGltZXMgbn18X3tVfSJdLFs0LDAsIlxcT2NhbF97VX0iXSxbMSwyLCJcXHNpZ21hfF97VX0iXSxbMCwxLCJcXGxhbWJkYV57XFxvdGltZXMgbn0iXV0=
\begin{tikzcd}
	{\Ocal_{U}} && {L^{\otimes n}|_{U}} && {\Ocal_{U}}
	\arrow["{\sigma|_{U}}", from=1-3, to=1-5]
	\arrow["{\lambda^{\otimes n}}", from=1-1, to=1-3]
\end{tikzcd}$$
is the identity on the sheaf $\Ocal_{U}$. There is an action of $\Bmu_{n}(U)$ on $\Pcal_{(L,\sigma)}(U)$ for which $\zeta\in\Bmu_{n}(U)$ acts by $\lambda\mapsto \zeta\cdot\lambda$ which is simply transitive, endowing $\Pcal$ with the structure of a $\Bmu_{n}$-torsor restricting on fibers in the previously described way. 
\begin{remark}
    It is necessary to work in the \'{e}tale topology here, since it is not always possible to find a trivialization of the line bundle Zariski-locally. 
\end{remark}
\section{Descent for Quasicoherent Sheaves}\label{sec: descent for quasicoherent sheaves}
\section{Quotients in Algebraic Spaces}
\section{Quotients in Algebraic Spaces}\label{sec: quotients in algebraic spaces}
In the category of affine schemes and $G$ a finite group acting on an affine scheme $\spec A$, the quotient of the affine scheme by the group action is the affine scheme $\spec A^{G}$, where $A^{G}$ is the ring of $G$-invariant elements of $A$. We consider more general quotients using the tools of algebraic spaces. 

\subsection{Quotients by Finite Flat Groupoids}
Consider a subcategory of the category of $S$-schemes that is a groupoid and $s,t:X_{1}\to X_{0}$. Taking the fibered product, we have a map $m:X_{1}\times_{s,X_{0},t}\to X_{1}$ as follows
$$% https://q.uiver.app/#q=WzAsNCxbMCwwLCJYX3sxfVxcdGltZXNfe3MsWF97MH0sdH1YX3sxfSJdLFsyLDAsIlhfezF9Il0sWzAsMSwiWF97MX0iXSxbMiwxLCJYX3swfSJdLFsxLDMsInMiXSxbMiwzLCJ0IiwyXSxbMCwyXSxbMCwxXV0=
\begin{tikzcd}
	{X_{1}\times_{s,X_{0},t}X_{1}} && {X_{1}} \\
	{X_{1}} && {X_{0}}
	\arrow["s", from=1-3, to=2-3]
	\arrow["t"', from=2-1, to=2-3]
	\arrow[from=1-1, to=2-1]
	\arrow[from=1-1, to=1-3]
\end{tikzcd}$$
There are special maps $f:X_{0}\to T$ known as invariant maps which are defined as follows. 
\begin{definition}[Invariant Morphism]\label{def: invariant morphism}
    Let $s,t:X_{0}\to X_{1}$ be isomorphisms of schemes and $T$ be an algebraic space. A morphism $f:X_{0}\to T$ is invariant if $f\circ s=f\circ t$. 
\end{definition}
We prove the following theorem. 
\begin{theorem}\label{thm: invariant morphisms extending finite flat morphisms}
    If $s,t:X_{1}\to X_{0}$ are finite flat isomorphisms of schemes such that for all $x\in X_{0}$, $s(t^{-1}(x))$ is contained in an affine open subscheme of $X_{0}$ then there exists an invariant morphism of schemes $\pi:X_{0}\to Y$ and is universal with respect to this property. 
\end{theorem}
\begin{remark}
    In other words, $\pi$ is universal for invariant morphisms to schemes: for any other invariant morphism $g:X_{0}\to Z$, there is a unique morphism $h:Y\to Z$ such that $g=h\circ\pi$ such given by the commutativity of the following diagram. 
    $$% https://q.uiver.app/#q=WzAsNCxbMCwwLCJYX3sxfSJdLFsyLDAsIlhfezB9Il0sWzQsMCwiWSJdLFs0LDEsIloiXSxbMiwzLCJcXGV4aXN0cyFoIiwwLHsic3R5bGUiOnsiYm9keSI6eyJuYW1lIjoiZGFzaGVkIn19fV0sWzEsMywiZyIsMl0sWzEsMiwiXFxwaSJdLFswLDMsIiIsMCx7ImN1cnZlIjoyfV0sWzAsMSwicyIsMCx7Im9mZnNldCI6LTF9XSxbMCwxLCJ0IiwyLHsib2Zmc2V0IjoxfV1d
    \begin{tikzcd}
        {X_{1}} && {X_{0}} && Y \\
        &&&& Z
        \arrow["{\exists!h}", dashed, from=1-5, to=2-5]
        \arrow["g"', from=1-3, to=2-5]
        \arrow["\pi", from=1-3, to=1-5]
        \arrow[curve={height=12pt}, from=1-1, to=2-5]
        \arrow["s", shift left, from=1-1, to=1-3]
        \arrow["t"', shift right, from=1-1, to=1-3]
    \end{tikzcd}$$
\end{remark}
\subsection{Topological Properties of Algebraic Spaces}
Recall the definition of quasiseparatedness \Cref{def: quasiseparated space}. We show the following lemma. 
\begin{lemma}\label{lem: space is a point}
    If $X$ is a quasiseparated $S$-algebraic space such that there is an epimorphism $g:\spec(K)\to X$ for some field $K$ then $X$ is the spectrum of a field. 
\end{lemma}
Just as we can forget the structure sheaf of a scheme to consider its underlying topological space, we can do the same for algebraic spaces as follows. 
\begin{definition}[Topological Space of Algebraic Space]\label{def: topological space of algebraic space}
    Let $X$ be an $S$-algebraic space. The topological space associated to $X$ is 
    $$|X|=\Mor_{\Spaces_{S}}\left(\spec(k), X\right)/\sim$$
    where $(\spec(k)\to X)\sim(\spec(k')\to X)$ if and only if there is an isomorphism of affine schemes $\sigma:\spec(k)\to\spec(k')$ such that the diagram 
    $$% https://q.uiver.app/#q=WzAsMyxbMCwwLCJcXHNwZWMoaykiXSxbMiwwLCJcXHNwZWMoaycpIl0sWzEsMSwiWCJdLFswLDJdLFswLDEsIlxcc2lnbWEiXSxbMSwyXV0=
    \begin{tikzcd}
        {\spec(k)} && {\spec(k')} \\
        & X
        \arrow[from=1-1, to=2-2]
        \arrow["\sigma", from=1-1, to=1-3]
        \arrow[from=1-3, to=2-2]
    \end{tikzcd}$$
    commutes. 
\end{definition}
We naturally define a closed subspace of the topological space $|X|$ as those arising as the topological realization of a closed sub-algebraic space of the algebraic space $X$. 
\begin{proposition}
    If $X$ is a quasiseparated $S$-algebraic spae and $f:\spec(k)\to X$ a morphism of algebraic spaces for a field $k$ then there exists a point $\iota:\spec(k')\to X$ and a factorization of $f$ as 
    $$% https://q.uiver.app/#q=WzAsMyxbMCwwLCJcXHNwZWMoaykiXSxbMiwwLCJcXHNwZWMoaycpIl0sWzQsMCwiWCJdLFsxLDIsIlxcaW90YSJdLFswLDEsImciXV0=
    \begin{tikzcd}
        {\spec(k)} && {\spec(k')} && X.
        \arrow["\iota", from=1-3, to=1-5]
        \arrow["g", from=1-1, to=1-3]
    \end{tikzcd}$$
\end{proposition}
Note that the topological realization functor $|-|:\Spaces_{S}\to\Top$ is functorial since for $f:X\to Y$ a morphism of $S$-algebraic spaces, we can define the points of $Y$ by pre-composing maps $\spec(k)\to X$ with $f$. For $Z\subseteq Y$ an open sub-algebraic space -- the complement of a closed sub-algebraic space -- its preimage $f^{-1}(Z)\subseteq X$ is necessarily an open sub-algebraic space giving an open subspace $|f^{-1}(Z)|\subseteq|X|$ as its topological realization, showing $|f|$ is a continuous map, that is, a morphism in the category of topological spaces $\Top$. 
\\\\
Closedness and properness of morphisms of algebraic spaces are defined in terms of the underlying topological spaces as follows. 
\begin{definition}[Closed Morphism of Algebraic Spaces]\label{def: closed morphism of algebraic spaces}
    Let $f:X\to Y$ be a morphism of quasiseparated $S$-algebraic spaces. $f$ is a closed morphism of $S$-algebraic spaces if $|f|:|X|\to|Y|$ is a closed map of topological spaces. 
\end{definition}
\begin{definition}[Universally Closed Morphism of Algebraic Spaces]\label{def: universally closed morphism of algebraic spaces}
    Let $f:X\to Y$ be a morphism of quasiseparated $S$-algebraic spaces. $f$ is a universally closed morphism if for all maps of algebraic spaces $Z\to Y$ with $Z$ a quasiseparated algebraic space the morphism of algebraic spaces induced by base change $X\times_{Y}Z\to Y$ is closed. 
\end{definition}
\begin{remark}
    In \Cref{def: universally closed morphism of algebraic spaces} we are evidently considering the following pullback diagram. 
    $$% https://q.uiver.app/#q=WzAsNCxbMCwwLCJYXFx0aW1lc197WX1aIl0sWzIsMCwiWCJdLFsyLDEsIlkiXSxbMCwxLCJaIl0sWzEsMiwiZiJdLFswLDNdLFszLDJdLFswLDFdXQ==
    \begin{tikzcd}
        {X\times_{Y}Z} && X \\
        Z && Y
        \arrow["f", from=1-3, to=2-3]
        \arrow[from=1-1, to=2-1]
        \arrow[from=2-1, to=2-3]
        \arrow[from=1-1, to=1-3]
    \end{tikzcd}$$
\end{remark}

\begin{definition}[Proper Morphism of Algebraic Spaces]\label{def: proper morphism of algebraic spaces}
    Let $f:X\to Y$ be a morphism of algebraic spaces. $f$ is a proper morphism of algebraic spaces if it is separated, of finite type, and universally closed. 
\end{definition}
\begin{remark}
    Just as in the case of schemes, proper morphisms of stacks are taken to be separated (\Cref{def: separated morphism of spaces}), of finite type, and universally closed (\Cref{def: universally closed morphism of algebraic spaces}). Finite-typeness here is to be taken in the sense of \Cref{def: properties of morphisms of spaces via schemes} where there exist schemes $U,V$ and surjective \'{e}tale morphisms $U\to X, V\to Y$ such that the morphism of schemes $V\times_{Y}U\to V$ is of finite type. 
\end{remark}
\subsection{Open Subschemes of Algebraic Spaces}
We show that every algebraic space contains a scheme as a dense open set. 
\begin{theorem}\label{thm: every algebraic space has a dense open subscheme}
    If $X$ is a quasiseparated $S$-algebraic space then there is a scheme $U$ and a dense open embedding $U\hookrightarrow X$. 
\end{theorem}
\part*{Algebraic Stacks}
\section{Algebraic Stacks}
\section{Quasicoherent Sheaves on Algebraic Stacks}
\part*{The Geometry of Stacks}
\section{Geometric Properties of Stacks}
\part*{The Geometry of Stacks}\label{part: geometry of stacks}
\section{Geometric Properties of Stacks}\label{sec: geometry of stacks}
We now consider some further properties of morphisms of stacks. 
\begin{definition}[Embedding]\label{def: embedding of stacks}
    Let $f:\Zcal\to\Xcal$ be a morphism of $S$-algebraic stacks. $f$ is an embedding if $f$ is a representable morphism of stacks and for all $Y\to\Ycal$ for $Y$ an algebraic space the map $Y\times_{\Ycal}\Xcal\to\Ycal$ is an embedding of algebraic spaces. 
\end{definition}
\begin{definition}[Open Embedding]\label{def: open embedding of stacks}
    Let $f:\Zcal\to\Xcal$ be a morphism of $S$-algebraic stacks. $f$ is an open embedding if $f$ is a representable morphism of stacks and for all $Y\to\Ycal$ for $Y$ an algebraic space the map $Y\times_{\Ycal}\Xcal\to\Ycal$ is an embedding of algebraic spaces. 
\end{definition}
\begin{definition}[Closed Embedding]\label{def: closed embedding of stacks}
    Let $f:\Zcal\to\Xcal$ be a morphism of $S$-algebraic stacks. $f$ is a closed embedding if $f$ is a representable morphism of stacks and for all $Y\to\Ycal$ for $Y$ an algebraic space the map $Y\times_{\Ycal}\Xcal\to\Ycal$ is an embedding of algebraic spaces. 
\end{definition}
Here we used \Cref{def: property of stack morphism via spaces} and the following Cartesian diagram
$$% https://q.uiver.app/#q=WzAsNCxbMCwwLCJZXFx0aW1lc197XFxZY2FsfVxcWGNhbCJdLFswLDEsIlxcWGNhbCJdLFsyLDAsIlkiXSxbMiwxLCJcXFljYWwiXSxbMSwzLCJmIiwyXSxbMiwzXSxbMCwxXSxbMCwyXV0=
    \begin{tikzcd}
        {Y\times_{\Ycal}\Xcal} && Y \\
        \Xcal && \Ycal
        \arrow["f"', from=2-1, to=2-3]
        \arrow[from=1-3, to=2-3]
        \arrow[from=1-1, to=2-1]
        \arrow[from=1-1, to=1-3]
    \end{tikzcd}$$
for $Y\to\Ycal$ a map from an algebraic space $Y$ to the stack $\Ycal$. 
\begin{definition}[Closed Substack]\label{def: closed substack}
    Let $\Xcal$ be an $S$-algebraic stack. A closed substack of $\Xcal$ is an equivalence class of closed embeddings $\Zcal\to\Xcal$ such that 
    $$\left(f_{1}:\Zcal_{1}\to\Xcal\right)\sim\left(f_{2}:\Zcal\to\Xcal\right)$$
    if and only if there is a morphism of stacks $g:\Zcal_{1}\to\Zcal_{2}$ and a $S$-natural isomorphism of stack morphisms $\sigma:f_{2}\circ g\to f_{1}$. 
\end{definition}
\begin{remark}
    In the setup of \Cref{def: closed embedding of stacks} that $f$ is a representable morphism of stacks so in the definition of closed substacks (\ref{def: closed substack}) the pair $(g,\sigma)$ is unique up to unique isomorphism. 
\end{remark}
We now define the property closedness and universal closedness of morphisms of stacks. 
\begin{definition}[Closed Morphism of Stacks]\label{def: closed morphism of stacks}
    Let $\Xcal$ be an $S$-algebraic stack and $f:\Xcal\to Y$ a morphism from $\Xcal$ to a scheme $Y$. $f$ is a closed morphism of stacks if for all closed substacks $\Zcal\subseteq\Xcal$, $f(\Zcal)\subseteq Y$ is a closed subscheme. 
\end{definition}
\begin{definition}[Universally Closed Morphism of Stacks]\label{def: universally closed morphism of stacks}
    Let $f:\Xcal\to\Ycal$ be a morphism of $S$-algebraic stacks. $f$ is a universally closed morphism of stacks if for all schemes $Y$ and $Y\to\Ycal$ 
    $$% https://q.uiver.app/#q=WzAsNCxbMCwwLCJZXFx0aW1lc197XFxZY2FsfVxcWGNhbCJdLFswLDEsIlxcWGNhbCJdLFsyLDAsIlkiXSxbMiwxLCJcXFljYWwiXSxbMSwzLCJmIiwyXSxbMiwzXSxbMCwxXSxbMCwyXV0=
    \begin{tikzcd}
        {Y\times_{\Ycal}\Xcal} && Y \\
        \Xcal && \Ycal
        \arrow["f"', from=2-1, to=2-3]
        \arrow[from=1-3, to=2-3]
        \arrow[from=1-1, to=2-1]
        \arrow[from=1-1, to=1-3]
    \end{tikzcd}$$
    the morphism $Y\times_{\Ycal}\Xcal\to Y$ is a closed morphism of stacks. 
\end{definition}
This leads us to the definition of a proper morphism of algebraic stacks, building on the notion of separatedness of stack morphisms as introduced in \Cref{def: separated stack morphism}.
\begin{definition}[Proper Morphism of Stacks]\label{def: proper morphism of stacks}
    Let $f:\Xcal\to\Ycal$ be a morphism of $S$-algebraic stacks. $f$ is a proper morphism if it is separated, of finite type, and universally closed. 
\end{definition}
We can characterize universal closedness for representable separated morphisms of finite type in terms of properness.
\begin{proposition}\label{prop: separated finite type is universally closed iff proper}
    Let $f:\Xcal\to\Ycal$ be a representable separated morphism of finite type of $S$-algebraic stacks. $f$ is universally closed if and only if $f$ is proper. 
\end{proposition}
\subsection{The Functors $\relSpec$ and $\relProj$}
Once again drawing parallels to the constructions in the category of schemes, we can construct stacks from sheaves using relative spec $\relSpec$ and relative proj $\relProj$. 
\begin{definition}[Relative Spec for Stacks]
    Let $\Xcal$ be an $S$-algebraic stack and 
    $$\Acal:\LisEt\left(\Xcal\right)^{\Opp}\longrightarrow\Alg_{\Ocal_{\Xcal}}$$
    be a quasicoherent sheaf of algebras on $\Xcal$. The stack $\relSpec_{\Xcal}\left(\Acal\right)$ has 
    \begin{enumerate}[label=(\alph*)]
        \item objects triples $(T,x,\rho)$ where $T$ is an $S$-scheme, $x\in\Xcal(T)$, and $\rho:x^{*}\Acal\to\Ocal_{T}$ is a morphism of sheaves of algebras on the scheme $T$;
        \item morphisms $(g,g^{b}):(T',x',\rho')\to(T,x,\rho)$ such that $g:T'\to T$ a morphism of $S$-schemes and $g^{b}:x'\to x$ a morphism in $\Xcal$ over $g$ such that the diagram 
        $$% https://q.uiver.app/#q=WzAsMyxbMCwwLCJ4J157Kn1cXEFjYWwiXSxbMiwwLCJnXnsqfXheeyp9XFxBY2FsIl0sWzEsMSwiXFxPY2FsX3tUJ30iXSxbMCwxLCJnXntifSJdLFswLDIsIlxccmhvJyIsMl0sWzEsMiwiZ157Kn1cXHJobyJdXQ==
        \begin{tikzcd}
            {x'^{*}\Acal} && {g^{*}x^{*}\Acal} \\
            & {\Ocal_{T'}}
            \arrow["{g^{b}}", from=1-1, to=1-3]
            \arrow["{\rho'}"', from=1-1, to=2-2]
            \arrow["{g^{*}\rho}", from=1-3, to=2-2]
        \end{tikzcd}$$
        commutes. 
    \end{enumerate}
\end{definition}
Recall that we have descent for quasicoherent sheaves and thus $\relSpec_{\Xcal}\left(\Acal\right)$ is a stack in the \'{e}tale topology. There there is a natural forgetful map of stacks
$$\relSpec_{\Xcal}\left(\Acal\right)\longrightarrow\Xcal$$
by $(T,x,\rho)\mapsto(T,x)$. 
\\\\
Recall that by specializing \Cref{def: property of stack morphism via spaces}, a representable morphism of $S$-algebraic stacks $\Xcal\to\Ycal$ is affine if for all morphisms $y:Y\to\Xcal$ and $Y$ an algebraic space the map $Y\times_{\Ycal}\Xcal\to Y$ arising from the Cartesian square 
$$% https://q.uiver.app/#q=WzAsNCxbMCwwLCJZXFx0aW1lc197XFxZY2FsfVxcWGNhbCJdLFswLDEsIlxcWGNhbCJdLFsyLDAsIlkiXSxbMiwxLCJcXFljYWwiXSxbMSwzLCJmIiwyXSxbMiwzXSxbMCwxXSxbMCwyXV0=
    \begin{tikzcd}
        {Y\times_{\Ycal}\Xcal} && Y \\
        \Xcal && \Ycal
        \arrow["f"', from=2-1, to=2-3]
        \arrow[from=1-3, to=2-3]
        \arrow[from=1-1, to=2-1]
        \arrow[from=1-1, to=1-3]
    \end{tikzcd}$$
is an affine morphism. In particular, for $\Xcal\to\Zcal$ and $\Ycal\to\Zcal$ two affine morphisms, the $\Zcal$-morphisms $\Mor_{\Zcal}\left(\Xcal,\Ycal\right)$ are a set, and hence the full subacategory of the slice category of stacks over $\Zcal$ with structure morphisms affine morphisms form a 1-category. Relative spec then is a functor from quasicoherent sheaves on a stack to the slice category. 
\begin{proposition}
    Let $\mathsf{Aff}_{\Xcal}$ be the 1-category of the full subcategory of the slice category of $S$-algebraic stacks over $\Xcal$ with structure morphisms affine morphisms. The functor 
    $$\relSpec_{\Xcal}\left(-\right):\QCoh(\Xcal)_{\LisEt}\longrightarrow\mathsf{Aff}_{\Xcal}$$
    by $\Acal\mapsto\relSpec_{\Xcal}\left(\Acal\right)$ is an equivalence of categories. 
\end{proposition}
The construction of relative proj constructs a stack in a similar way. Recall the setup for schemes. Let $T$ be an $S$-scheme and $\Acal=\bigoplus_{d\geq0}\Acal_{d}$ a quasicoherent sheaf of graded $\Ocal_{T}$-algebras we have a scheme $\relProj_{T}\left(\Acal\right)$ constructed by gluing $\proj\left(\Gamma(U,\Acal)\right)$ over all $U\subseteq T$ affine open. There is a natural map $\pi:\relProj_{T}\left(\Acal\right)\to T$ that factors as 
$$% https://q.uiver.app/#q=WzAsNCxbNiwwLCJUIl0sWzQsMCwiVSJdLFsyLDAsIlxccHJvalxcbGVmdChcXEFjYWwoVSlcXHJpZ2h0KT1cXHByb2pcXGxlZnQoXFxHYW1tYShVLFxcQWNhbClcXHJpZ2h0KSJdLFswLDAsIlxccmVsUHJval97VH1cXGxlZnQoXFxBY2FsXFxyaWdodCl8X3tcXHBpXnstMX0oVSl9Il0sWzEsMCwiXFxpb3RhX3tVfSJdLFsyLDFdLFszLDIsIlxcc2ltIl1d
\begin{tikzcd}
	{\relProj_{T}\left(\Acal\right)|_{\pi^{-1}(U)}} && {\proj\left(\Acal(U)\right)=\proj\left(\Gamma(U,\Acal)\right)} && U && T
	\arrow["{\iota_{U}}", from=1-5, to=1-7]
	\arrow[from=1-3, to=1-5]
	\arrow["\sim", from=1-1, to=1-3]
\end{tikzcd}$$
on all restrictions 
$$\pi|_{\pi^{-1}(U)}:\relProj_{T}\left(\Acal\right)|_{\pi^{-1}(U)}\to T$$
for $U\subseteq T$ open and inclusion $\iota_{U}:U\to T$. One can think of $\pi:\relProj_{T}\left(\Acal\right)\to T$ as a fibration on $T$ by projective schemes. Naturally a section $\rho$ is a map $\rho:T\to\relProj_{T}\left(\Acal\right)$
$$% https://q.uiver.app/#q=WzAsMixbMCwwLCJcXHJlbFByb2pfe1R9XFxsZWZ0KFxcQWNhbFxccmlnaHQpIl0sWzAsMiwiVCJdLFswLDEsIlxccGkiXSxbMSwwLCJcXHJobyIsMCx7ImN1cnZlIjotM31dXQ==
\begin{tikzcd}
	{\relProj_{T}\left(\Acal\right)} \\
	\\
	T
	\arrow["\pi", from=1-1, to=3-1]
	\arrow["\rho", curve={height=-18pt}, from=3-1, to=1-1]
\end{tikzcd}$$
such that $\pi\circ\rho=\id_{T}$. 
\begin{definition}[Relative Proj for Stacks]\label{def: relative proj for stacks}
    Let $\Xcal$ be an $S$-algebraic stack and $\Acal=\bigoplus_{d\geq0}\Acal_{d}$ a quasicoherent sheaf of graded $\Ocal_{\Xcal}$-algebras on $\Xcal$. The stack $\relProj_{\Xcal}\left(\Acal\right)$ has 
    \begin{enumerate}[label=(\alph*)]
        \item objects triples $(T,x,\rho)$ with $T$ an $S$-scheme, $x\in\Xcal(T)$, and $\rho:T\to\relProj_{T}(x^{*}\Acal)$ a section of the $T$-scheme;
        \item morphisms $(g,\widetilde{g}):(T',x',\rho')\to(T,x,\rho)$ such that $g:T'\to T$ is a morphism of $S$-schemes and $\widetilde{g}:x'\to x$ a morphism in $\Xcal$ over $g$ such that the diagram 
        $$% https://q.uiver.app/#q=WzAsNCxbMCwwLCJUJyJdLFsyLDAsIlQiXSxbMiwxLCJcXHJlbFByb2pfe1R9XFxsZWZ0KHheeyp9XFxBY2FsXFxyaWdodCkiXSxbMCwxLCJcXHJlbFByb2pfe1QnfVxcbGVmdCh4J157Kn1cXEFjYWxcXHJpZ2h0KSJdLFszLDIsIlxcd2lkZXRpbGRle2d9IiwyXSxbMCwxLCJnIl0sWzEsMiwiXFxyaG8iXSxbMCwzLCJcXHJobyciLDJdXQ==
        \begin{tikzcd}
            {T'} && T \\
            {\relProj_{T'}\left(x'^{*}\Acal\right)} && {\relProj_{T}\left(x^{*}\Acal\right)}
            \arrow["{\widetilde{g}}"', from=2-1, to=2-3]
            \arrow["g", from=1-1, to=1-3]
            \arrow["\rho", from=1-3, to=2-3]
            \arrow["{\rho'}"', from=1-1, to=2-1]
        \end{tikzcd}$$
        commutes. 
    \end{enumerate}
\end{definition}
\begin{remark}
    For 
    $$\Acal=\bigoplus_{d\geq0}\Acal_{d}$$ 
    a quasicoherent sheaf of $\Ocal_{\Xcal}$-algebras for all $(T,t)\in\Obj\left(\LisEt(\Xcal)\right)$, $\Acal_{(T,t)}$ is a quasicoherent sheaf of graded $\Ocal_{T}$-algebras with equivalence $(\Ocal_{\Xcal})_{(T,t)}=\Gamma(T,\Ocal_{T})=\Ocal_{T}(T)$ by definition. 
\end{remark}
\subsection{Root Stacks}\label{subsec: root stacks}
One important construction one encounters is that of root stacks. Let $X$ be a scheme and $D$ an effective Cartier divisor on $X$. One might want to find an effective Cartier divisor $E$ and integer $n$ such that $nE\sim D$. This is not possible in general, but one could find a morphism of schemes $f:Y\to X$ such that there exists an effective Cartier divisor $E$ on $Y$ and an integer $n$ giving the following equivalence of divisors $nE\sim f^{*}D$. The root stack construction attempts to give a solution to this problem by finding a ``universal'' such $(Y,E)$. Recall that divisors do not pull back along arbitrary morphisms, necessitating the introduction of the following more general object. 
\begin{definition}[Generalized Effective Cartier Divisor]\label{def: generalized effective Cartier divisor}
    Let $X$ be a scheme. A generalized Cartier divisor on $X$ is a pair $(L,\rho)$ such that $L$ is an invertible sheaf on $X$ and $\rho:L\to\Ocal_{X}$ a morphism of $\Ocal_{X}$-modules. 
\end{definition}
An isomorphism of generalized Cartier divisors $(L',\rho')$ and $(L,\rho)$ is an isomorphism of line bundles $\sigma:L'\to L$ such that the diagram 
$$% https://q.uiver.app/#q=WzAsMyxbMCwwLCJMJyJdLFsyLDAsIkwiXSxbMSwxLCJcXE9jYWxfe1h9Il0sWzAsMSwiXFxzaWdtYSJdLFswLDIsIlxccmhvJyIsMl0sWzEsMiwiXFxyaG8iXV0=
\begin{tikzcd}
	{L'} && L \\
	& {\Ocal_{X}}
	\arrow["\sigma", from=1-1, to=1-3]
	\arrow["{\rho'}"', from=1-1, to=2-2]
	\arrow["\rho", from=1-3, to=2-2]
\end{tikzcd}$$
commutes. Indeed for $D\subset X$ an effective Cartier divisor and $\Ical_{D}$ its ideal sheaf, the inclusion $D\hookrightarrow X$ induces a morphism of $\Ocal_{X}$-modules $j_{D}:\Ical_{D}\to\Ocal_{X}$ where for $D,D'$ two effective Cartier divisors $(\Ical_{D},j_{D})$ and $(\Ical_{D'},j_{D'})$ are isomorphic if and only if $D\sim D'$ if and only if $\Ical_{D}\cong\Ical_{D'}$ as $\Ocal_{X}$-modules. One can define the product of two generalized Cartier divisors $(L,\rho)$ and $(L',\rho')$
$$(L,\rho)\cdot(L',\rho')=(L\otimes L', \rho\otimes\rho')$$
with 
$$\rho\otimes\rho':L\otimes L'\longrightarrow \Ocal_{X}\cong\Ocal_{X}\otimes_{\Ocal_{X}}\Ocal_{X}.$$
For $n\geq0$ one can define the effective Cartier divisor $(L^{\otimes n},\rho^{\otimes n})$ to be the $N$-fold product of $(L,\rho)$ with itself in the abovementioned way. Let $\stackyDiv^{+}\left(X\right)$ to be the set of isomorphism classes of generalized effective Cartier divisors. The tensor product endows $\stackyDiv^{+}(X)$ with the structure of a commutative monoid. 
\\\\
The advantage of generalized Cartier divisors is that they can be pulled back along morphisms of schemes as for $f:Y\to X$ and $(L,\rho)$ a generalized Cartier divisor on $X$, $g^{*}L$ is a generalized Cartier divisor on $Y$ with morphism to $\Ocal_{Y}$ given by $g^{*}\rho:g^{*}L\to g^{*}\Ocal_{X}=\Ocal_{Y}$. The construction of $\stackyDiv^{+}(X)$ is thus functorial on schemes, allowing us to make the following construction. 
\begin{definition}[$\mathsf{Div}$]\label{def: the fibered category of generalized cartier Divisors}
    Let $\mathsf{Div}$ be the category with objects pairs $\left(T,(L,\rho)\right)$ with $T$ a scheme and $(L,\rho)\in\stackyDiv^{+}(T)$ and morphisms 
    $$(g,g^{b}):\left(T',(L',\rho')\right)\longrightarrow\left(T,(L,\rho)\right)$$
    with $g:T'\to T$ a morphism of schemes and $g^{b}:(L',\rho')\to(g^{*}L,g^{*}\rho)$ an isomorphism of effective Cartier divisors on $T'$. 
\end{definition}
By descent for sheaves, $\mathsf{Div}$ is a stack on the category of schemes. In fact, this stack has an especially nice description. 
\begin{proposition}\label{prop: Div fibered category is isomorphic to the line mod Gm}
    There is an isomorphism of stacks $\mathsf{Div}\to[\A^{1}/\GG_{m}]$. 
\end{proposition}

\section{Coarse Moduli Spaces}\label{sec: coarse moduli spaces}
An important result in stack theory is the Keel-Mori theorem, showing the existence of coarse moduli spaces of algebraic stacks with finite diagonal, which in turn allows us to understand a number of important properties: the local structure of Deligne-Mumford stacks, a variant of Chow's lemma, and finiteness of cohomology of sheaves on Deligne-Mumford stacks. 
\begin{definition}[Coarse Moduli Space]\label{def: coarse moduli space}
    Let $\Xcal$ be an $S$-algebraic stack for a base scheme $S$. A coarse moduli space for the stack $\Xcal$ is a morphism $\pi:\Xcal\to X$ to a scheme $X$ such that the following conditions hold:
    \begin{enumerate}[label=(\alph*)]
        \item $\pi$ is initial for maps to $S$-algebraic spaces. 
        \item For an algebraically closed field $k$, there is a bijection between isomorphism classes of $\Xcal(k)$ to the $k$-rational points of $X$
        $$\left|\Xcal(k)\right|\to X(k).$$
    \end{enumerate}
\end{definition}
\begin{remark}
    $\pi$ being initial means that the scheme $X$ is the scheme $X$ is the initial object of the slice-over category $\Spaces_{(\Xcal/-)}$ with objects morphisms $(\Xcal\to Y)$ and morphisms between two objects $(\Xcal\to Y), (\Xcal\to Z)$ commuting triangles of the following type. 
    $$% https://q.uiver.app/#q=WzAsMyxbMSwwLCJcXFhjYWwiXSxbMCwxLCJYIl0sWzIsMSwiWSJdLFswLDFdLFsxLDJdLFswLDJdXQ==
    \begin{tikzcd}
        & \Xcal \\
        Y && Z
        \arrow[from=1-2, to=2-1]
        \arrow[from=2-1, to=2-3]
        \arrow[from=1-2, to=2-3]
    \end{tikzcd}$$
    More explicitly, for $g:\Xcal\to Z$ with $Z$ an algebraic space, 
    $$% https://q.uiver.app/#q=WzAsMyxbMCwwLCJcXFhjYWwiXSxbMiwxLCJaIl0sWzIsMCwiWCJdLFswLDEsImciLDJdLFswLDIsIlxccGkiXSxbMiwxLCJcXGV4aXN0cyFmIiwwLHsic3R5bGUiOnsiYm9keSI6eyJuYW1lIjoiZGFzaGVkIn19fV1d
    \begin{tikzcd}
        \Xcal && X \\
        && Z
        \arrow["g"', from=1-1, to=2-3]
        \arrow["\pi", from=1-1, to=1-3]
        \arrow["{\exists!f}", dashed, from=1-3, to=2-3]
    \end{tikzcd}$$
    there is a unique morphism of stacks such that $g=f\circ\pi$. 
\end{remark}
\subsection{The Theorem of Keel and Mori}\label{subsec: Keel Mori theorem}
We now state the Keel-Mori theorem. 
\begin{theorem}[Keel-Mori]\label{thm: Keel Mori theorem}
    Let $\Xcal$ be an $S$-algebraic stack locally of finite presentation over $S$ with a finite diagonal over a locally Noetherian base scheme $S$. The algebraic stack $\Xcal$ admits a coarse moduli space such that:
    \begin{enumerate}[label=(\alph*)]
        \item $X$ is an $S$-scheme locally of finite type. Furthermore if $\Xcal$ is a separated algebraic stack, then $X$ is a separated $S$-scheme. 
        \item $\pi$ is a proper morphism and $\Ocal_{X}\to\pi_{*}\Ocal_{\Xcal}$ is an isomorphism. 
        \item If $X'\to X$ is a flat morphism with $X'$ an algebraic space, 
        $$% https://q.uiver.app/#q=WzAsNCxbMCwxLCJcXFhjYWwiXSxbMiwxLCJYIl0sWzIsMCwiWCciXSxbMCwwLCJYJ1xcdGltZXNfe1h9XFxYY2FsIl0sWzAsMSwiXFxwaSIsMl0sWzIsMV0sWzMsMF0sWzMsMl1d
        \begin{tikzcd}
            {X'\times_{X}\Xcal} && {X'} \\
            \Xcal && X
            \arrow["\pi"', from=2-1, to=2-3]
            \arrow[from=1-3, to=2-3]
            \arrow[from=1-1, to=2-1]
            \arrow[from=1-1, to=1-3]
        \end{tikzcd}$$
        then $X'$ is a coarse moduli space for $X'\times_{X}\Xcal$. 
    \end{enumerate}
\end{theorem}
The Keel-Mori theorem allows us to connect Deligne-Mumford stacks (\ref{def: Deligne Mumford stack}) with (topological) orbifolds. 
\newpage
\subsection{Local Structure of Deligne-Mumford Stacks}\label{subsec: local structure of DM stacks}
\begin{theorem}[Local Structure of Deligne-Mumford Stacks]\label{def: local structure of DM stacks}
    Let $\Xcal$ be a Deligne-Mumford stack locally of finite type with finite diagonal over a locally Noetherian base scheme $S$, and $\pi:\Xcal\to X$ its coarse moduli space. Let $\widetilde{x}$ be a geometric point of $\Xcal$ with image $\pi(\widetilde{x})=\overline{x}$ and $G_{\widetilde{x}}$ the automorphism group of $\widetilde{x}$. There exists $\overline{x}\in U\subseteq X$ an \'{e}tale neighborhood of $\overline{x}$ and a finite $U$-scheme $V$ with action of $G_{\widetilde{x}}$ such that in the Cartesian square 
    $$% https://q.uiver.app/#q=WzAsNCxbMCwxLCJcXFhjYWwiXSxbMiwxLCJYIl0sWzIsMCwiVSJdLFswLDAsIlVcXHRpbWVzX3tYfVxcWGNhbCJdLFswLDEsIlxccGkiLDJdLFsyLDFdLFszLDBdLFszLDJdXQ==
    \begin{tikzcd}
        {U\times_{X}\Xcal} && U \\
        \Xcal && X
        \arrow["\pi"', from=2-1, to=2-3]
        \arrow[from=1-3, to=2-3]
        \arrow[from=1-1, to=2-1]
        \arrow[from=1-1, to=1-3]
    \end{tikzcd}$$
    we have an isomorphism of stacks 
    $$U\times_{X}\Xcal\cong\left[V/G_{\widetilde{x}}\right].$$
\end{theorem}
\begin{remark}
    Since $\Xcal$ is a Deligne-Mumford stack with $\widetilde{x}$ a geometric point, $G_{\widetilde{x}}$ is a finite group. 
\end{remark}
The finiteness of the automorphism group $G_{\widetilde{x}}$ allows us to define another property of Deligne-Mumford stacks. 
\begin{definition}[Tame Deligne-Mumford Stack]\label{def: tame DM stack}
    Let $\Xcal$ be an algebraic stack locally of finite type over a locally Noetherian base scheme $S$. $\Xcal$ is a tame stack if for every geometric point $\widetilde{x}:\spec(k)\to\Xcal$, the order of the automorphism group $|G_{\widetilde{x}}|$ is an invertible element in $k$. 
\end{definition}
In the case of sufficiently nice Deligne-Mumford stacks, its sheaves are characterized by the sheaves on the coarse space. 
\begin{proposition}\label{prop: exact map of QCoh sheaves on stacks to coarse space}
    Let $\Xcal$ be Deligne-Mumford stack locally of finite type with finite diagonal over a locally Noetherian base scheme $S$ and coarse space $\pi:\Xcal\to X$. If $\Xcal$ is a tame stack, then the functor 
    $$\pi_{*}:\QCoh\left(\Xcal\right)_{\LisEt}\longrightarrow\QCoh\left(X\right)$$
    is exact. 
\end{proposition}
The following theorem characterizes the behavior of coarse spaces under base change. 
\begin{theorem}\label{thm: coarse spaces under base change}
    Let $\Xcal$ be a separated Deligne-Mumford stack of finite type over a locally Noetherian base scheme $S$ and coarse space $\pi:\Xcal\to X$. For $S'\to S$, $\tau:\Xcal\times_{S}S'\to Y$ the coarse space of the base change of the stack $\Xcal$ and $p:Y\to X\times_{S}S'$ the morphism induced by the universal property of the coarse space, $p$ is a universal homeomorphism. If further $S'\to S$ is flat or $\Xcal$ is a tame stack, then $p$ is an isomorphism. 
\end{theorem}
\begin{remark}
    The properties of local finite typeness and separatedness are preserved by base change so the stack $\Xcal\times_{S}S'$ and the scheme $X\times_{S}S'$ obtained by base changes
    $$% https://q.uiver.app/#q=WzAsOCxbMCwwLCJcXFhjYWxcXHRpbWVzX3tTfVMnIl0sWzIsMSwiUyJdLFsyLDAsIlxcWGNhbCJdLFswLDEsIlMnIl0sWzQsMSwiUyciXSxbNiwxLCJTIl0sWzYsMCwiWCJdLFs0LDAsIlhcXHRpbWVzX3tTfVMnIl0sWzMsMV0sWzAsMl0sWzAsM10sWzIsMV0sWzQsNV0sWzYsNV0sWzcsNF0sWzcsNl1d
    \begin{tikzcd}
        {\Xcal\times_{S}S'} && \Xcal && {X\times_{S}S'} && X \\
        {S'} && S && {S'} && S
        \arrow[from=2-1, to=2-3]
        \arrow[from=1-1, to=1-3]
        \arrow[from=1-1, to=2-1]
        \arrow[from=1-3, to=2-3]
        \arrow[from=2-5, to=2-7]
        \arrow[from=1-7, to=2-7]
        \arrow[from=1-5, to=2-5]
        \arrow[from=1-5, to=1-7]
    \end{tikzcd}$$
    is also locally of finite type and separated, hence admitting a smooth moduli space by the Keel-Mori theorem \Cref{thm: Keel Mori theorem}. 
\end{remark}
\subsection{Chow's Lemma for Deligne-Mumford Stacks}\label{subsec: Chows lemma for DM stacks}
\begin{theorem}[Chow's Lemma for Stacks]\label{thm: Chows lemma for stacks}
    Let $\Xcal$ be a Deligne-Mumford stack of finite type with finite diagonal over a Noetherian base-scheme $S$. There exists a proper surjective morphism $X'\to\Xcal$ with $X'$ an $S$-scheme finite over a dense open substack of $\Xcal$ such that the composition 
    $$X'\longrightarrow\Xcal\longrightarrow S$$
    is a projective morphism of schemes. 
\end{theorem}
\section{Gerbes}\label{sec: gerbes}
Recall that for a scheme $X$ and $\Bmu$ a sheaf of Abelian groups on the small \'{e}tale site of $X$, the first sheaf cohomology $H^{1}(X,\Bmu)$ is isomorphic to the group of $\Bmu$-torsors on $X$. The interpretation can be extended to $H^{2}(X,\Bmu)$ as the set of $\Bmu$-gerbes which can be roughly thought of as a stack over $X$ which is a twisted form of the classifying stack $B\Bmu$. 
\subsection{Torsors and First Cohomology}\label{subsec: torsors and first cohomology}
Let $\Tcal$ be a topos on a site $\Ssf$ and $\Bmu\in\Obj(\Tcal)$ a sheaf of Abelian groups on $\Ssf$. We describe a bijection between the set of $\Bmu$-torsors and $H^{1}(\Tcal,\Bmu)$. 
\begin{definition}[Wedge of Torsors]\label{def: wedge of torsors}
    Let $(\Pcal,\rho),(\Pcal',\rho')$ be two $\Bmu$-torsors on a site $\Ssf$ with a final object. The wedge $(\Pcal\wedge\Pcal',\rho\wedge\rho')$ is the quotient of $\Pcal\times\Pcal'$ by the action 
    $$g\cdot(p,p')\mapsto(gp,g^{-1}p').$$
\end{definition}
This easily implies the following. 
\begin{proposition}[Magma Structure on $\Tors$]\label{def: magma on wedge of torsors}
    Let $(\Pcal,\rho),(\Pcal',\rho')$ be two $\Bmu$-torsors on a site $\Ssf$ with a final object. The operation 
    $$\wedge:\Tors(\Bmu)\times\Tors(\Bmu)\longrightarrow\Tors(\Bmu)$$
    $$\left((\Pcal,\rho),(\Pcal',\rho')\right)\mapsto\left(\Pcal\wedge\Pcal',\rho\wedge\rho'\right)$$
    endows $\Tors(\Bmu)$ with the structure of a magma.
\end{proposition}
\begin{remark}
    Recall here that a magma is a set closed under a binary operation. 
\end{remark}
\begin{proof}
    The action
    $$\rho\wedge\rho':\Bmu\times\Pcal\times\Pcal'\longrightarrow\Pcal$$
    by 
    $$g\cdot (p,p')\mapsto(gp,p')$$
    descends to $\Pcal\wedge\Pcal'$ endowing it with the structure of a $\Bmu$-torsor. 
\end{proof}
Denote $\underline{\ZZ}$ the constant sheaf on the integers $\ZZ$. We can define a category $\Extsf(\underline{\ZZ},\Bmu)$ in the following way. 
\begin{definition}[$\Extsf(\underline{\ZZ},\Bmu)$]\label{def: ext category on integer sheaf}
    Let $\Ssf$ be a site with a final object and $\Bmu$ a sheaf of Abelian groups on $\Ssf$. Define $\Extsf(\underline{\ZZ},\Bmu)$ be the category with: 
    \begin{enumerate}[label=(\alph*)]
        \item Objects short exact sequences of sheaves of Abelian groups on $\Ssf$, $(0\to\Bmu\to\Ecal\to\underline{\ZZ}\to0)$. 
        \item Morphisms between 
        $$(0\to\Bmu\to\Ecal'\to\underline{\ZZ}\to0),(0\to\Bmu\to\Ecal\to\underline{\ZZ}\to0)\in\Obj\left(\Extsf(\underline{\ZZ},\Bmu)\right)$$
        commutative diagrams of the following type:
        $$% https://q.uiver.app/#q=WzAsMTAsWzAsMCwiMCJdLFsxLDAsIlxcQm11Il0sWzIsMCwiXFxFY2FsJyJdLFszLDAsIlxcdW5kZXJsaW5le1xcWlp9Il0sWzQsMCwiMCJdLFswLDEsIjAiXSxbMSwxLCJcXEJtdSJdLFsyLDEsIlxcRWNhbCJdLFszLDEsIlxcdW5kZXJsaW5le1xcWlp9Il0sWzQsMSwiMCJdLFswLDFdLFsxLDJdLFsyLDNdLFszLDRdLFs1LDZdLFs2LDddLFs3LDhdLFs4LDldLFszLDgsIlxcd3IiXSxbMiw3XSxbMSw2LCJcXHdyIiwyXV0=
        \begin{tikzcd}
            0 & \Bmu & {\Ecal'} & {\underline{\ZZ}} & 0 \\
            0 & \Bmu & \Ecal & {\underline{\ZZ}} & 0
            \arrow[from=1-1, to=1-2]
            \arrow[from=1-2, to=1-3]
            \arrow[from=1-3, to=1-4]
            \arrow[from=1-4, to=1-5]
            \arrow[from=2-1, to=2-2]
            \arrow[from=2-2, to=2-3]
            \arrow[from=2-3, to=2-4]
            \arrow[from=2-4, to=2-5]
            \arrow["\wr", from=1-4, to=2-4]
            \arrow[from=1-3, to=2-3]
            \arrow["\wr"', from=1-2, to=2-2]
        \end{tikzcd}$$
    \end{enumerate}
\end{definition}
We can check locally to show that for any such morphism in \Cref{def: ext category on integer sheaf} (b), the morphism $\Ecal'\to\Ecal$ is in fact an isomorphism. One can then define a functor from $\Extsf(\underline{\ZZ},\Bmu)$ to the category of $\Bmu$-torsors $\Tors(\Bmu)$ and show that it is an equivalence of categories. 
\begin{proposition}\label{prop: integer ext category is equivalent to torsor category}
    The functor 
    $$\pi:\Extsf(\underline{\ZZ},\Bmu)\longrightarrow\Tors(\Bmu)$$
    by
    $$(0\to\Bmu\to\Ecal\to\underline{\ZZ}\to0)\mapsto \pi^{-1}(1)\in\Ecal; 1\in\Gamma(\Ssf,\Bmu)$$
    is an equivalence of categories. 
\end{proposition}
This allows us to deduce the desired result. 
\begin{corollary}\label{prop: bijection of first cohomology and iso classes of torsors}
    There is a bijection between $H^{1}(\Ssf,\Bmu)$ and the set of isomorphism classes of $\Bmu$-torsors. 
\end{corollary}
\subsection{Gerbes and Second Cohomology}\label{subsec: gerbes and second cohomology}
Recall for $p:\Fsf\to\Ssf$ a (categorical) stack over a site, for any $x\in\Obj(\Fsf_{X})$ the fiber over $X\in\Obj(\Ssf)$ there is a sheaf $\ulAut_{x}$ over the slice category $\Ssf_{(-/X)}$. For $\Bmu$ a sheaf of Abelian groups on $\Ssf$, a gerbe is a stack constructed such that the restriction of the sheaf $\Bmu$ to the slice category $\Ssf_{(-/X)}$ is isomorphic to the automorpihsm group $\ulAut_{x}$. More explicitly, we have the following. 
\begin{definition}[Gerbe]\label{def: gerbe}
    Let $\Ssf$ be a site and $\Bmu$ a sheaf of Abelian groups on $\Ssf$. A $\Bmu$-gerbe over $\Ssf$ is a stack $p:\Fsf\to\Ssf$ where $\Fsf$ is the fibered category with objects $(x,\iota_{x})$ where for all $x\in\Obj(\Fsf)$ an isomorphism of sheaves of groups 
    $$\iota_{x}:\Bmu|_{\Ssf_{(-/p(x))}}\longrightarrow\ulAut_{x}$$
    such that:
    \begin{enumerate}[label=(\alph*)]
        \item For any $X\in\Obj(\Ssf)$ there is a covering $\{\phi_{i}:X_{i}\to X\}_{i\in I}$ with $\Fsf_{X_{i}}\neq\emptyset$ for all $i\in I$.
        \item For any $x,x'\in\Fsf_{X}$ there is a covering $\{\phi_{i}:X_{i}\to X\}_{i\in I}$ with $\phi_{i}^{*}x=\phi_{i}^{*}x'$ in $\Fsf_{X_{i}}$ for all $i\in I$. 
        \item For all $X\in\Obj(\Ssf)$ and isomorphism $\sigma:x\to x'$ in $\Fsf_{X}$ a commutative diagram of the following type:
        $$% https://q.uiver.app/#q=WzAsMyxbMCwxLCJcXHVsQXV0X3t4fSJdLFsyLDEsIlxcdWxBdXRfe3gnfSJdLFsxLDAsIlxcQm11Il0sWzIsMCwiXFxpb3RhX3t4fSIsMl0sWzIsMSwiXFxpb3RhX3t4J30iXSxbMCwxLCJcXHNpZ21hIiwyXV0=
        \begin{tikzcd}
            & \Bmu \\
            {\ulAut_{x}} && {\ulAut_{x'}}
            \arrow["{\iota_{x}}"', from=1-2, to=2-1]
            \arrow["{\iota_{x'}}", from=1-2, to=2-3]
            \arrow["\sigma"', from=2-1, to=2-3]
        \end{tikzcd}$$
    \end{enumerate}
\end{definition}
\begin{remark}
    Morphisms between objects in fibers of $\Fsf$ are induced along morphisms in $\Ssf$. For $(g:Y\to X)\in\Mor_{\Ssf}$ there is a functor of categories $g^{*}:\Fsf_{Y}\to g^{*}\Fsf_{X}$ that for $p(y)=Y$ with $y=g^{*}x$ an isomorphism of groups $\iota_{y}:\Bmu|_{(-/p(y))}\to\ulAut_{y}$ satisfying the additional conditions (a), (b), and (c) in \Cref{def: gerbe} such that the following diagram commutes. 
    $$% https://q.uiver.app/#q=WzAsMyxbMSwwLCJcXEJtdSJdLFswLDEsIlxcdWxBdXRfe3l9Il0sWzIsMSwiXFx1bEF1dF97eH0iXSxbMCwyLCJcXGlvdGFfe3h9Il0sWzAsMSwiXFxpb3RhX3t5fSIsMl0sWzEsMiwiKGZeeyp9KV97Kn0iLDJdXQ==
    \begin{tikzcd}
        & \Bmu \\
        {\ulAut_{y}} && {\ulAut_{x}}
        \arrow["{\iota_{x}}", from=1-2, to=2-3]
        \arrow["{\iota_{y}}"', from=1-2, to=2-1]
        \arrow["{(f^{*})_{*}}"', from=2-1, to=2-3]
    \end{tikzcd}$$
    Note further that the morphism of sheaves of groups $(f^{*})_{*}$ need not be an isomorphism. This is not crucial for the definition of the gerbe, and we hence omit it from the definition. 
\end{remark}
A morphism of $\Bmu$-gerbes can be defined in the natural way. 
\newpage
\begin{definition}[Morphism of Gerbes]\label{def: morphism of gerbes}
    Let $\Ssf$ be a site and $\Bmu$ a sheaf of Abelian groups on $\Ssf$ with $p:\Fsf\to\Ssf$ and $p':\Fsf'\to\Ssf$ to $\Bmu$-gerbes on $\Ssf$. A morphism of $\Bmu$-gerbes is a morphism of stacks $f:\Fsf'\to\Fsf$ such that for $f(x')=x$ the diagram 
    $$% https://q.uiver.app/#q=WzAsMyxbMSwwLCJcXEJtdSJdLFswLDEsIlxcdWxBdXRfe3gnfSJdLFsyLDEsIlxcdWxBdXRfe3h9Il0sWzEsMiwiZl97Kn0iLDJdLFswLDIsIlxcaW90YV97eH0iXSxbMCwxLCJcXGlvdGFfe3gnfSIsMl1d
    \begin{tikzcd}
        & \Bmu \\
        {\ulAut_{x'}} && {\ulAut_{x}}
        \arrow["{f_{*}}"', from=2-1, to=2-3]
        \arrow["{\iota_{x}}", from=1-2, to=2-3]
        \arrow["{\iota_{x'}}"', from=1-2, to=2-1]
    \end{tikzcd}$$
    commutes. 
\end{definition}
We can in fact refine this characterization of morphisms of $\Bmu$-gerbes as we now show. 
\begin{proposition}\label{prop: a morphism of gerbes is an isomorphism}
    Let $\Ssf$ be a site and $\Bmu$ a sheaf of Abelian groups on $\Ssf$ with $p:\Fsf\to\Ssf$ and $p':\Fsf'\to\Ssf$ to $\Bmu$-gerbes on $\Ssf$. If $f:\Fsf'\to\Fsf$ is a morphism of gerbes then $f:\Fsf'\to\Fsf$ is an isomorphism of gerbes. 
\end{proposition}
We now show the correspondence with second cohomology. 
\\\\
Consider the following construction. Let $\Bmu$ be a sheaf of Abelian groups on a site $\Ssf$ and 
$$% https://q.uiver.app/#q=WzAsNSxbMCwwLCIxIl0sWzEsMCwiXFxCbXUiXSxbMiwwLCJHIl0sWzMsMCwiUSJdLFs0LDAsIjEiXSxbMCwxXSxbMSwyXSxbMiwzXSxbMyw0XV0=
\begin{tikzcd}
	1 & \Bmu & G & Q & 1
	\arrow[from=1-1, to=1-2]
	\arrow[from=1-2, to=1-3]
	\arrow[from=1-3, to=1-4]
	\arrow[from=1-4, to=1-5]
\end{tikzcd}$$
be an exact sequence of sheaves of groups where $G,Q$ need not be sheaves of Abelian groups. Given a $Q$-torsor $\Pcal$ as in \Cref{def: torsor} we can construct a fibered category over the site $\Ssf$. 
\begin{remark}
    We use multiplicative notation for the sequence above, since the construction in practice often uses multiplicative groups such as $\GG_{m}$. 
\end{remark}
\begin{definition}[The Category $\Gcal_{\Pcal}$]\label{def: category g sub p}
    Consider the following construction. Let $\Bmu$ be a sheaf of Abelian groups on a site $\Ssf$ and 
    $$% https://q.uiver.app/#q=WzAsNSxbMCwwLCIxIl0sWzEsMCwiXFxCbXUiXSxbMiwwLCJHIl0sWzMsMCwiUSJdLFs0LDAsIjEiXSxbMCwxXSxbMSwyXSxbMiwzXSxbMyw0XV0=
    \begin{tikzcd}
	1 & \Bmu & G & Q & 1
	\arrow[from=1-1, to=1-2]
	\arrow[from=1-2, to=1-3]
	\arrow[from=1-3, to=1-4]
	\arrow[from=1-4, to=1-5]
    \end{tikzcd}$$
    be an exact sequence of sheaves of groups and $\Pcal$ a $Q$-torsor. Let $\Gcal_{\Pcal}$ be the fibered category over $\Ssf$ with 
    \begin{enumerate}[label=(\alph*)]
        \item Objects triples $\left(X,\widetilde{\Pcal},\epsilon\right)$ with $X\in\Obj(\Ssf)$, $\widetilde{\Pcal}$ a $G|_{\Ssf_{(-/X)}}$-torsor on the slice category $\Ssf_{(-/X)}$, and $\epsilon:b_{*}\widetilde{\Pcal}\to\Pcal|_{\Ssf_{(-/X)}}$ an isomorphism of $Q|_{\Ssf_{(-/X)}}$-torsors. 
        \item Morphsims 
        $$(f,f^{b}):\left(X',\widetilde{\Pcal'},\epsilon'\right)\longrightarrow\left(X,\widetilde{\Pcal},\epsilon\right)$$
        where $f:X'\to X$ is a morphism in $\Ssf$ and $f^{b}:f^{*}\widetilde{\Pcal}\to\widetilde{\Pcal'}$ an isomorphism of $G|_{\Ssf_{(-/X')}}$-torsors such that the pentagon 
        $$% https://q.uiver.app/#q=WzAsNSxbMSwzLCJiX3sqfWZeeyp9XFx3aWRldGlsZGV7XFxQY2FsfSJdLFszLDMsImZeeyp9Yl97Kn1cXHdpZGV0aWxkZXtcXFBjYWx9Il0sWzAsMSwiYl97Kn1cXHdpZGV0aWxkZXtcXFBjYWwnfSJdLFs0LDEsImZeeyp9XFxQY2FsfF97XFxTc2ZfeygtL1gpfX0iXSxbMiwwLCJcXFBjYWx8X3tcXFNzZl97KC0vWCcpfX0iXSxbMCwxLCJcXHNpbSIsMl0sWzEsMywiZl57Kn1cXGVwc2lsb24iLDJdLFszLDQsIlxcc2ltIiwyXSxbMiw0LCJcXGVwc2lsb24nIl0sWzAsMiwiYl97Kn1mXntifSJdXQ==
        \begin{tikzcd}
            && {\Pcal|_{\Ssf_{(-/X')}}} \\
            {b_{*}\widetilde{\Pcal'}} &&&& {f^{*}\Pcal|_{\Ssf_{(-/X)}}} \\
            \\
            & {b_{*}f^{*}\widetilde{\Pcal}} && {f^{*}b_{*}\widetilde{\Pcal}}
            \arrow["\sim"', from=4-2, to=4-4]
            \arrow["{f^{*}\epsilon}"', from=4-4, to=2-5]
            \arrow["\sim"', from=2-5, to=1-3]
            \arrow["{\epsilon'}", from=2-1, to=1-3]
            \arrow["{b_{*}f^{b}}", from=4-2, to=2-1]
        \end{tikzcd}$$
        commutes. 
    \end{enumerate}
\end{definition}
% is this commutative diagram correct? 
\begin{remark}
    $p:\Gcal_{\Pcal}\to\Ssf$ is defined by the functorial assignment 
    $$\left(X,\widetilde{\Pcal},\epsilon\right)\mapsto X.$$
    In particular, the fibered category $\Gcal_{\Pcal}$ is a category fibered in groupoids over the site $\Ssf$. 
\end{remark}
Note that for any $\left(X,\widetilde{\Pcal},\epsilon\right)\in\Obj\left(\Gcal_{\Pcal}\right)$ the automorphism presheaf $\ulAut_{\left(X,\widetilde{\Pcal},\epsilon\right)}$ is the presheaf of groups such that for any $g:X'\to X$ associates the automorphism group of the $G|_{\Ssf_{(-/X')}}$-torsor $\widetilde{\Pcal}|_{\Ssf_{(-/X')}}$ such that the pushout 
$$% https://q.uiver.app/#q=WzAsNCxbMCwwLCJcXFBjYWx8X3tcXFNzZl97KC0vWCcpfX0iXSxbMCwxLCJYJyJdLFsyLDEsIlgiXSxbMiwwLCJcXFBjYWx8X3tcXFNzZl97KC0vWCl9fSJdLFszLDIsIiIsMix7InN0eWxlIjp7ImhlYWQiOnsibmFtZSI6Im5vbmUifX19XSxbMCwxLCIiLDIseyJzdHlsZSI6eyJoZWFkIjp7Im5hbWUiOiJub25lIn19fV0sWzEsMiwiZyIsMl0sWzAsMywiYl97Kn0iXV0=
\begin{tikzcd}
	{\Pcal|_{\Ssf_{(-/X')}}} && {\Pcal|_{\Ssf_{(-/X)}}} \\
	{X'} && X
	\arrow[no head, from=1-3, to=2-3]
	\arrow[no head, from=1-1, to=2-1]
	\arrow["g"', from=2-1, to=2-3]
	\arrow["{b_{*}}", from=1-1, to=1-3]
\end{tikzcd}$$
along $b_{*}$ is the identity automorphism. In particular this group of automorphisms is $\Bmu(X')$ giving a cannonical isomorphism 
$$\iota_{x}:\Bmu|_{\Ssf_{(-/X)}}\longrightarrow\ulAut_{\left(X,\widetilde{\Pcal},\epsilon\right)}.$$
\begin{proposition}\label{prop: g sub p is a mu gerbe}
    Consider the following construction. Let $\Bmu$ be a sheaf of Abelian groups on a site $\Ssf$ and 
    $$% https://q.uiver.app/#q=WzAsNSxbMCwwLCIxIl0sWzEsMCwiXFxCbXUiXSxbMiwwLCJHIl0sWzMsMCwiUSJdLFs0LDAsIjEiXSxbMCwxXSxbMSwyXSxbMiwzXSxbMyw0XV0=
    \begin{tikzcd}
	1 & \Bmu & G & Q & 1
	\arrow[from=1-1, to=1-2]
	\arrow[from=1-2, to=1-3]
	\arrow[from=1-3, to=1-4]
	\arrow[from=1-4, to=1-5]
    \end{tikzcd}$$
    be an exact sequence of sheaves of groups and $\Pcal$ a $Q$-torsor. The fibered category $p:\Gcal_{\Pcal}$ is a $\Bmu$-gerbe. Furthermore, the construction is functorial with respect to $Q$-torsors. 
\end{proposition}
We now describe the correspondence between second cohomology and $\Bmu$-torsors. 
\\\\
For $\alpha\in H^{2}(\Ssf,\Bmu)$ we can construct a $\Bmu$-gerbe $\Gcal_{\alpha}$ as follows: choose some inclusion of sheaves of Abelian groups $i:\Bmu\to I$ and since $i$ is injective, this yields an exact sequence of sheaves of Abelian groups 
$$% https://q.uiver.app/#q=WzAsNSxbMCwwLCIwIl0sWzEsMCwiXFxCbXUiXSxbMiwwLCJJIl0sWzMsMCwiSS9cXEJtdSJdLFs0LDAsIjAiXSxbMCwxXSxbMSwyXSxbMiwzXSxbMyw0XV0=
\begin{tikzcd}
	0 & \Bmu & I & {I/\Bmu} & 0
	\arrow[from=1-1, to=1-2]
	\arrow[from=1-2, to=1-3]
	\arrow[from=1-3, to=1-4]
	\arrow[from=1-4, to=1-5]
\end{tikzcd}$$
and once again with $i$ injective, the boundary map 
$$\partial: H^{1}(\Ssf,I/\Bmu)\longrightarrow H^{2}(\Ssf,\Bmu)$$
is an isomorphism. Thus our previous construction from $Q$-torsors as described in \Cref{def: category g sub p} suffices and is a $\Bmu$-gerbe by \Cref{prop: g sub p is a mu gerbe} with the correspondence described by 
$$\alpha\mapsto\Gcal_{\partial^{-1}(\alpha)}.$$
This is functorial on exact sequences: for $i':\Bmu\to I'$ another inclusion and for a choice $\rho:I\to I'$ descending to the quotient $\overline{\rho}:I/\Bmu\to I'/\Bmu$ we have the commutative diagram
$$% https://q.uiver.app/#q=WzAsMTAsWzAsMCwiMCJdLFsxLDAsIlxcQm11Il0sWzIsMCwiSSJdLFszLDAsIkkvXFxCbXUiXSxbNCwwLCIwIl0sWzAsMSwiMCJdLFsxLDEsIlxcQm11Il0sWzIsMSwiSSciXSxbMywxLCJJJy9cXEJtdSJdLFs0LDEsIjAiXSxbMCwxXSxbMSwyXSxbMiwzXSxbMyw0XSxbNSw2XSxbNiw3XSxbNyw4XSxbOCw5XSxbMSw2LCJcXHdyIiwyXSxbMiw3LCJcXHJobyIsMl0sWzMsOCwiXFxvdmVybGluZXtcXHJob30iLDJdXQ==
\begin{tikzcd}
	0 & \Bmu & I & {I/\Bmu} & 0 \\
	0 & \Bmu & {I'} & {I'/\Bmu} & 0
	\arrow[from=1-1, to=1-2]
	\arrow[from=1-2, to=1-3]
	\arrow[from=1-3, to=1-4]
	\arrow[from=1-4, to=1-5]
	\arrow[from=2-1, to=2-2]
	\arrow[from=2-2, to=2-3]
	\arrow[from=2-3, to=2-4]
	\arrow[from=2-4, to=2-5]
	\arrow["\wr"', from=1-2, to=2-2]
	\arrow["\rho"', from=1-3, to=2-3]
	\arrow["{\overline{\rho}}"', from=1-4, to=2-4]
\end{tikzcd}$$
where $i'=\rho\circ i$ that induces the commutative triangle 
$$% https://q.uiver.app/#q=WzAsMyxbMCwwLCJIXnsxfShcXFNzZixJL1xcQm11KSJdLFsyLDAsIkheezF9KFxcU3NmLEknL1xcQm11KSJdLFsxLDEsIkheezJ9KFxcU3NmLFxcQm11KSJdLFswLDIsIlxccGFydGlhbCIsMl0sWzAsMSwiXFxvdmVybGluZXtcXHJob30iXSxbMSwyLCJcXHBhcnRpYWwnIl1d
\begin{tikzcd}
	{H^{1}(\Ssf,I/\Bmu)} && {H^{1}(\Ssf,I'/\Bmu)} \\
	& {H^{2}(\Ssf,\Bmu)}
	\arrow["\partial"', from=1-1, to=2-2]
	\arrow["{\overline{\rho}}", from=1-1, to=1-3]
	\arrow["{\partial'}", from=1-3, to=2-2]
\end{tikzcd}$$
but taking $\gamma\in H^{1}(\Ssf,I/\Bmu)$ such that $\overline{\rho}(\gamma)=\gamma'\in H^{1}(\Ssf,I'/\Bmu)$ with $\Bmu$-gerbes $\Gcal_{\gamma},\Gcal_{\gamma'}$, respectively, and $\Pcal$ the $I/\Bmu$-torsor corresponding to $\Gcal_{\gamma}$ we can pushout along $\overline{\rho}$ where for $\left(X,\widetilde{\Pcal},\epsilon\right)\in\Gcal_{\Pcal}(X)$ we have $\left(X,\rho_{*}\widetilde{\Pcal},\overline{\rho}_{*}\epsilon\right)\in\in\Gcal_{\Pcal'}(X)$ inducing 
$$\Gcal_{\Pcal}\longrightarrow\Gcal_{\Pcal'}$$
but such morphisms are isomorphisms by \Cref{prop: a morphism of gerbes is an isomorphism} giving the claim. 
\begin{theorem}\label{thm: second cohomology is bijective with iso classes of gerbes}
    Let $\Ssf$ be a site and $\Bmu$ a sheaf of Abelian groups on $\Ssf$. There is a bijection between the cohomology classes in $H^{2}(\Ssf,\Bmu)$ and isomorphism classes of $\Bmu$-gerbes. 
\end{theorem}
\subsection{Twisted Sheaves}\label{subsec: twisted sheaves}
One application the theory of gerbes and torsors have is to twisted sheaves. Following \Cref{subsec: torsors and first cohomology}, let $\underline{\GG_{m}}$ be the sheaf 
$$\underline{\GG_{m}}:\left(\Sch_{S}\right)_{\et}^{\Opp}\longrightarrow\Grp$$
given on objects by 
$$(T\to S)\mapsto \Gamma\left(T,\Ocal_{T}^{\times}\right)$$
and on $S$-morphisms
$$(f:T'\to T)\mapsto \left(f^{\sharp}:\Gamma(T,f_{*}\Ocal_{T'}^{\times})\longrightarrow \Gamma(T',\Ocal_{T'}^{\times})\right).$$
One can show that an $\underline{\GG_{m}}$-gerbe is an algebraic stack. Let $\Gcal$ be an $\underline{\GG_{m}}$-gerbe and $E$ a locally free sheaf on $\Gcal$ of finite rank. Given a field $k$ and a geometric point $x:\spec(k)\to\Gcal$ we have an action of $\Aut_{x}\cong\GG_{m}(k)\cong k^{\times}$ on the $k$-vector space $E(x)=x^{*}E$. Any representation of $\GG_{m}(k)$ $V$ admits a decomposition as graded vector spaces 
$$V\cong\bigoplus_{i\in\ZZ}V_{i}$$
where the action of $\GG_{m}(k)$ acts by $u\cdot v\mapsto u^{i}v$ for $v\in V_{i}$. The set of integers $i\in\ZZ$ such that $V_{i}\neq0$ are known as the weights of the representation. 
\begin{definition}[Twisted Sheaf]\label{def: twisted sheaf}
    Let $\Gcal$ be a $\underline{\GG_{m}}$-gerbe on $(\Sch_{S})_{\et}$, $E$ a locally free sheaf of finite rank on $\Gcal$, and $n\in\ZZ$. $E$ is an $n$-twisted sheaf if for all fields $k$ and all geometric points $x:\spec(k)\to\Gcal$ the representation is of weight $n$, that is $E(x)=E(x)_{n}$.
\end{definition}
Twisted sheaves satisfy several nice properties. 
\begin{lemma}\label{lem: properties of twisted sheaves}
    Let $\Gcal$ be a $\underline{\GG_{m}}$-gerbe on $(\Sch_{S})_{\et}$ with structure morphism $\pi:\Gcal\to S$. 
    \begin{enumerate}[label=(\alph*)]
        \item If $E_{1}$ is an $n_{1}$-twisted sheaf and $E_{2}$ is an $n_{2}$-twisted sheaf on $\Gcal$ for $n_{1},n_{2}\in\ZZ$ then $E_{1}\otimes E_{2}$ is an $(n_{1}+n_{2})$-twisted sheaf. 
        \item If $M$ is a locally free sheaf of finite rank on $S$, then $\pi^{*}M$ is a 0-twisted sheaf on $\Gcal$ and defines an equivalence between the category of locally free sheaves of finite rank on $S$ and $0$-twisted sheaves on $\Gcal$. 
        \item If $E$ is an $n$-twisted sheaf on $\Gcal$, then the dual $E^{\vee}$ is a $(-n)$-twisted sheaf on $\Gcal$. 
    \end{enumerate}
\end{lemma}
Note that the correspondence in \Cref{lem: properties of twisted sheaves} (b) is compatible with tensor products of sheaves both on $\Gcal$ and $S$. In particular, for $E$ an $n$-twisted sheaf on the gerbe $\Gcal$, we can construct a sheaf of non-commutative algebras by taking endomorphisms 
$$\sheafEnd(E)=\Hom_{\Sh(\Gcal)}\left(E,E\right)=E^{\vee}\otimes E$$
is 0-twisted by \Cref{lem: properties of twisted sheaves} (a) and (c). In particular, it is isomorphic to a pullback $\pi^{*}\Acal_{E}$ of a sheaf of lcoally free algebras $\Acal_{E}$ on $S$. This connects the study of gerbes and Brauer groups. 
\begin{definition}[Azumaya Algebra]\label{def: Azumaya algebra}
    Let $S$ be a scheme. An Azumaya algebra is a sheaf of $\Ocal_{S}$-algebras $\Acal$ such that \'{e}tale locally is isomorphic to $\sheafEnd(E)$ for a locally free sheaf $E$ of finite rank. 
\end{definition}
\begin{remark}
    For $\Acal$ an Azumaya algebra on $S$ the ring of global sections $\Gamma(S,\Ocal_{S})$ is isomorphic to the center of $\Acal$. 
\end{remark}
If $E$ is a locally free sheaf of finite rank on $S$ and $L$ a line bundle, we can twist $E$ by the line bundle $L$ yielding a canonical isomorphism of algebras 
$$\sheafEnd(E)\cong\sheafEnd(E\otimes L)$$
and thus the locally free sheaf of finite rank $E$ is not unique. We can attempt to refine this construction in the following way. 
\begin{definition}[The Stack $\Gcal_{\Acal}$]\label{def: stack of Azumaya algebras}
    Let $\Acal$ be an Azumaya algebra on a scheme $S$ and define $\Gcal_{\Acal}$ to be the stack over the big \'{e}tale site of $S$-schemes $(\Sch_{S})_{\et}$ with:
    \begin{enumerate}[label=(\alph*)]
        \item Objects $(f:T\to S, E, \sigma)$ where $f$ is a morphism of $S$-schemes, $E$ a locally free sheaf of finite rank on $T$, and  $\sigma:\sheafEnd(E)\to f^{*}\Acal$ an isomorphism of $\Ocal_{T}$-algebras. 
        \item Morphisms 
        $$(g,g^{b}):(f':T'\to S, E',\sigma')\longrightarrow (f:T\to S, E, \sigma)$$
        where $g:T'\to T$ is an $S$-morphism and $g^{b}:g^{*}E\to E'$ an isomorphism of locally free sheaves of finite rank on $T'$ such that the diagram 
        $$% https://q.uiver.app/#q=WzAsMyxbMCwwLCJcXHNoZWFmRW5kKGdeeyp9RSkiXSxbMiwwLCJcXHNoZWFmRW5kKEUnKSJdLFsxLDEsImYnXnsqfVxcQWNhbCJdLFswLDIsIlxcc2lnbWEiLDJdLFsxLDIsIlxcc2lnbWEnIl0sWzAsMSwiZ157Yn0iXV0=
        \begin{tikzcd}
            {\sheafEnd(g^{*}E)} && {\sheafEnd(E')} \\
            & {f'^{*}\Acal}
            \arrow["\sigma"', from=1-1, to=2-2]
            \arrow["{\sigma'}", from=1-3, to=2-2]
            \arrow["{g^{b}}", from=1-1, to=1-3]
        \end{tikzcd}$$
        commutes. 
    \end{enumerate}
\end{definition}
\begin{remark}
    For any $(f:T\to S, E, \sigma)\in\Obj\left(\Gcal_{\Acal}\right)$ there is a natural map 
    $$\underline{\GG_{m}}(T)\to\Aut\left((f:T\to S, E,\sigma)\right); u\mapsto(\id,u).$$
\end{remark}
We can in fact show that this stack is a $\underline{\GG_{m}}$-gerbe. 
\begin{proposition}\label{def: stack of Azumaya algebras is a Gm gerbe}
    Let $\Acal$ be an Azumaya algebra on the scheme $S$. The algebraic stack $\Gcal_{\Acal}$ on the big \'{e}tale site of $S$-schemes $(\Sch_{S})_{\et}$ is a $\underline{\GG_{m}}$-gerbe. 
\end{proposition}
The construction of the stack $\Gcal_{\Acal}$ admits a refinement by the stack $\widetilde{\Gcal_{\Acal}}$. 
\newpage
\begin{definition}[The Stack $\widetilde{\Gcal_{\Acal}}$]
    Let $\Acal$ be an Azumaya algebra on a scheme $S$ and define $\widetilde{\Gcal_{\Acal}}$ to be the stack over the big \'{e}tale site of $S$-schemes $(\Sch_{S})_{\et}$ with:
    \begin{enumerate}[label=(\alph*)]
        \item Objects quadruples $(f:T\to S, E, \sigma, \lambda)$ where $f:T\to S$ is a morphism of $S$-schemes, $E$ a locally free sheaf of finite rank on $T$, $\sigma:\sheafEnd(E)\to f^{*}\Acal$ an isomorphism of $\Ocal_{T}$-algebras, and $\lambda:\Ocal_{T}\to\bigwedge^{\sqrt{\mathrm{rank}(\Acal)}}E$ a trivialization of the line bundle $\det E$. 
        \item Morphisms 
        $$(g,g^{b}):(f':T'\to S, E', \sigma', \lambda')\longrightarrow(f:T\to S, E, \sigma, \lambda)$$
        where $g:T'\to T$ is an $S$-morphism, $g^{b}:g^{*}E\to E'$ an isomorphism of locally free sheaves of finite rank on $T'$ such that the diagrams
        $$% https://q.uiver.app/#q=WzAsNixbMCwwLCJcXHNoZWFmRW5kKGdeeyp9RSkiXSxbMiwwLCJcXHNoZWFmRW5kKEUnKSJdLFsxLDEsImYnXnsqfVxcQWNhbCJdLFs0LDAsIlxcT2NhbF97VH0iXSxbNiwwLCJcXGJpZ3dlZGdlXntcXHNxcnR7XFxtYXRocm17cmFua30oXFxBY2FsKX19RSciXSxbNiwxLCJcXGJpZ3dlZGdlXntcXHNxcnR7XFxtYXRocm17cmFua30oXFxBY2FsKX19Z157Kn1FIl0sWzAsMiwiXFxzaWdtYSIsMl0sWzEsMiwiXFxzaWdtYSciXSxbMCwxLCJnXntifSJdLFszLDQsIlxcbGFtYmRhJyJdLFszLDUsImdeeyp9XFxsYW1iZGEiLDJdLFs0LDUsImdee2J9Il1d
        \begin{tikzcd}
            {\sheafEnd(g^{*}E)} && {\sheafEnd(E')} && {\Ocal_{T}} && {\bigwedge^{\sqrt{\mathrm{rank}(\Acal)}}E'} \\
            & {f'^{*}\Acal} &&&&& {\bigwedge^{\sqrt{\mathrm{rank}(\Acal)}}g^{*}E}
            \arrow["\sigma"', from=1-1, to=2-2]
            \arrow["{\sigma'}", from=1-3, to=2-2]
            \arrow["{g^{b}}", from=1-1, to=1-3]
            \arrow["{\lambda'}", from=1-5, to=1-7]
            \arrow["{g^{*}\lambda}"', from=1-5, to=2-7]
            \arrow["{g^{b}}", from=1-7, to=2-7]
        \end{tikzcd}$$
        commute.
    \end{enumerate}
\end{definition}
For $(f:T\to S, E, \sigma, \lambda)\in\Obj\left(\widetilde{\Gcal_{\Acal}}\right)$, the subgroup $\Bmu_{\sqrt{\mathrm{rank}(\Acal)}}\hookrightarrow\underline{\GG_{m}}$ of $(\sqrt{\mathrm{rank}(\Acal)})$-th roots of unity preserves $\lambda$. In particular, one can show the following. 
\begin{proposition}\label{prop: tidle GA is a mu r gerbe}
    Let $\Acal$ be an Azumaya algebra on the scheme $S$. The algebraic stack $\widetilde{\Gcal_{\Acal}}$ on the big \'{e}tale site of $S$-schemes is a $\Bmu_{r}$-gerbe where $r=\sqrt{\mathrm{rank}(\Acal)}$. Furthermore, the pushout along the inclusion $\Bmu_{r}\hookrightarrow\underline{\GG_{m}}$ yields an equivalence of categories from the pushout to the $\underline{\GG_{m}}$-gerbe $\Gcal_{\Acal}$.
\end{proposition}
\Cref{thm: second cohomology is bijective with iso classes of gerbes} suggests that there is a bijection between classes of the second cohomology on the big \'{e}tale site $H^{2}\left((\Sch_{S})_{\et},\underline{\GG_{m}}\right)$ and $\underline{\GG_{m}}$-torsors up to isomorphism. We describe the cohomology class 
$$[\Gcal_{\Acal}]\in H^{2}\left((\Sch_{S})_{\et},\underline{\GG_{m}}\right)$$
as follows. Let $r=\sqrt{\mathrm{rank}(\Acal)}$ and consider 
$$\underline{\mathrm{GL}_{r}}:(\Sch_{S})_{\et}^{\Opp}\to\Grp$$
by 
$$(T\to S)\mapsto \mathrm{GL}_{r}\left(\Ocal_{T}\right)$$
and 
$$\underline{\mathrm{PGL}_{r}}:(\Sch_{S})_{\et}^{\Opp}\to\Grp$$
by 
$$(T\to S)\mapsto \mathrm{PGL}_{r}\left(\Ocal_{T}\right)$$
sheaves of groups on the big \'{e}tale site $(\Sch_{S})_{\et}$ sending a scheme $(T\to S)$ to invertible matrices with entries in the ring of global sections and equivalence classes of such matrices modulo the diagonal action, respectively. 
\begin{remark}
    One can easily show compatibility with morphisms in the standard way. We omit the verification for ease of exposition. 
\end{remark}
There is a short exact sequence of sheaves on the big \'{e}tale site 
$$% https://q.uiver.app/#q=WzAsNSxbMCwwLCIxIl0sWzEsMCwiXFx1bmRlcmxpbmV7XFxHR197bX19Il0sWzIsMCwiXFx1bmRlcmxpbmV7XFxtYXRocm17R0x9X3tyfX0iXSxbMywwLCJcXHVuZGVybGluZXtcXG1hdGhybXtQR0x9X3tyfX0iXSxbNCwwLCIxIl0sWzAsMV0sWzEsMl0sWzIsM10sWzMsNF1d
\begin{tikzcd}
	1 & {\underline{\GG_{m}}} & {\underline{\mathrm{GL}_{r}}} & {\underline{\mathrm{PGL}_{r}}} & 1
	\arrow[from=1-1, to=1-2]
	\arrow[from=1-2, to=1-3]
	\arrow[from=1-3, to=1-4]
	\arrow[from=1-4, to=1-5]
\end{tikzcd}$$
and denote $P_{\Acal}$ the functor 
$$P_{\Acal}:(\Sch_{S})_{\et}\longrightarrow\Sets$$
by 
$$(T\to S)\mapsto\mathrm{Isom}\left(\mathrm{Mat}_{r}(\Ocal_{T}),f^{*}\Acal\right).$$
Any automorphism of $\mathrm{Mat}_{r}(\Ocal_{T})$ is locally inner and conjugation by an invertible matrix in $\mathrm{GL}_{r}(\Ocal_{T})$ is trivial if and only if the element is a scalar in $\mathrm{GL}_{r}(\Ocal_{T})$. In particular, $P_{\Acal}$ is a $\underline{\mathrm{PGL}_{r}}$-torsor and representable by an $S$-scheme. 
\begin{lemma}\label{lem: GA is the image of PA under the boundary map}
    Let $\Acal$ be an Azumaya algebra on $S$ and $\Gcal_{\Acal}$ the corresponding $\underline{\GG_{m}}$-gerbe on the big \'{e}tale site $(\Sch_{S})_{\et}^{\Opp}$ and $P_{\Acal}:(\Sch_{S})_{\et}\to\Sets$ be the $\underline{\mathrm{PGL}_{r}}$-torsor taking an $S$-scheme $T$ to isomorphisms between $\mathrm{Mat}_{r}(\Ocal_{T})$ and $f^{*}\Acal$ with $f:T\to S$. The cohomology class 
    $$[\Gcal_{\Acal}]\in H^{2}\left((\Sch_{S})_{\et},\underline{\GG_{m}}\right)$$
    is the image of $[P_{\Acal}]$ under the boundary map 
    $$\partial:H^{1}\left((\Sch_{S})_{\et},\underline{\mathrm{PGL}_{r}}\right)\longrightarrow H^{2}\left((\Sch_{S})_{\et},\underline{\GG_{m}}\right).$$
\end{lemma}
Let $\pi:\Gcal_{\Acal}\to S$ be the structure morphism. On the $\underline{\GG_{m}}$-gerbe there is a tautological locally free sheaf $\Ecal$ and an isomorphism 
$$\sigma:\sheafEnd(\Ecal)\longrightarrow \pi^{*}\Acal$$
of algebras over $\Gcal_{\Acal}$ and we can recover the Azumaya algebra $\Acal$ on $S$ as the pushforward $\Acal=\pi_{*}\sheafEnd(\Ecal)$ where $\Ecal$ is a 1-twisted sheaf on $\Gcal_{\Acal}$. A converse is given as follows. 
\begin{proposition}\label{prop: pushfoward of gerbes and azumaya algebras}
    Let $\pi:\Gcal\to S$ be a $\underline{\GG_{m}}$-gerbe and suppose there is a 1-twisted locally free sheaf of finite rank $\Ecal$ on $\Gcal$. $\pi_{*}\sheafEnd(\Ecal)$ is an Azumaya algebra on $S$, $\pi^{*}\pi_{*}\sheafEnd(\Ecal)\to\sheafEnd(\Ecal)$ is an isomorphism of sheaves on $\Gcal$, and $\Gcal\cong\Gcal_{\pi_{*}\sheafEnd(\Ecal)}$. 
\end{proposition}
\section{Cohomology of Stacks}\label{sec: cohomology of stacks}
\section{Derived Categories of Stacks}\label{sec: derived categories of stacks}
\part*{End Material}
\appendix
\section{Representable Functors}\label{sec: representable functors}
Let $\Csf$ be a category and consider functors from $\Csf^{\Opp}\to\Sets$. These are objects of the functor category $\Fun(\Csf^{\Opp},\Sets)$ whose objects are functors $\Csf^{\Opp}\to\Sets$ and whose morphisms are natural transformations between such functors. For any object $A\in\Obj(\Csf)$, we can define a set-valued functor $h_{A}$ as follows:
$$h_{A}:\Csf^{\Opp}\to\Sets \text{ by }A\mapsto\Mor_{\Csf}(-,A).$$
For an object $B\in\Obj(\Csf)$, our functor $h_{A}$ sends $B$ to the set $\Mor_{\Csf}(B,A)$ the set of morphisms from $B$ to $A$ in the category $\Csf$. Given $f\in\Mor_{\Csf}(B,C)$ we get a map of sets $h_{A}(f):h_{A}(C)\to h_{A}(B)$ defined by composition with $f$. For another $g\in \Mor_{\Csf}(A,D)$, we get a natural transformation of functors $h_{f}:h_{A}\to h_{D}$ where for any $f\in\Mor_{\Csf}(B,C)$ we have the following commutative diagram. 
$$% https://q.uiver.app/#q=WzAsNCxbMCwwLCJoX3tBfShDKSJdLFswLDEsImhfe0F9KEIpIl0sWzIsMCwiaF97RH0oQykiXSxbMiwxLCJoX3tEfShCKSJdLFsxLDMsImhfe2d9KEIpIiwyXSxbMiwzLCJoX3tEfShmKSJdLFswLDEsImhfe0F9KGYpIiwyXSxbMCwyLCJoX3tnfShDKSJdXQ==
\begin{tikzcd}
	{h_{A}(C)} && {h_{D}(C)} \\
	{h_{A}(B)} && {h_{D}(B)}
	\arrow["{h_{g}(B)}"', from=2-1, to=2-3]
	\arrow["{h_{D}(f)}", from=1-3, to=2-3]
	\arrow["{h_{A}(f)}"', from=1-1, to=2-1]
	\arrow["{h_{g}(C)}", from=1-1, to=1-3]
\end{tikzcd}$$
The map $h_{(-)}:\Csf\to\Fun(\Csf^{\Opp},\Sets)$ sends each object of $\Csf$ to a set-valued functor, the functor it represents, which we now define. 
\begin{definition}[Representable Functor]\label{def: representable functor}
    A functor $F:\Csf^{\Opp}\to\Sets$ is representable if there exists $A\in\Obj(\Csf)$ and a natural isomorphism $F\Longrightarrow h_{A}$. 
\end{definition}
Yoneda's lemma shows that the functor $h_{(-)}:\Csf\to\Fun(\Csf^{\Opp},\Sets)$ is fully faithful.
\begin{lemma}[Yoneda]\label{lem: Yoneda}
    Let $F:\Csf^{\Opp}\to\Sets$ be a functor. For any $A\in\Obj(\Csf)$, there is a bijection 
    $$\NatTrans(h_{A},F)\to F(A).$$
\end{lemma}
Yoneda's lemma tells us that $\Csf$ is embedded in the functor category $\Fun(\Csf^{\Opp},\Sets)$. Further, we know that for a functor $F:\Csf^{\Opp}\to\Sets$ we get a functor $h_{F}:\Fun(\Csf^{\Opp},\Sets)^{\Opp}\to\Sets$ taking $G\in\Obj(\Fun(\Csf^{\Opp},\Sets))$ to the set of natural transformations between $G$ and $F$ $\NatTrans(G,F)$. This suggests that every functor $\Csf^{\Opp}\to\Sets$ can be extended to a representable functor. This can be done via a process known as Kan extension, referring to 
\\\\
Note, however, that the functor $h_{(-)}:\Csf\to\Fun(\Csf^{\Opp},\Sets)$ is not essentially surjective, and hence would not define an equivalence of categories. This means that not every functor $\Csf^{\Opp}\to\Sets$ is representable by some object $A\in\Obj(\Csf)$. However, if we restrict to the full subcategory of representable functors in the functor category $\Fun(\Csf^{\Opp},\Sets)$, $h_{(-)}$ would indeed define a categorical equivalence between $\Csf$ and the subcategory of $\Fun(\Csf^{\Opp},\Sets)$ of representable functors. 
\\\\
From the proof of \Cref{lem: Yoneda}, we know that given a natural transformation of functors $T:h_{A}\to F$ on evaluation we get $T(A):h_{A}(A)\to F(A)$ yielding an element $T(A)(\id_{A})=\alpha\in F(A)$ for $\id_{A}\in h_{A}(A)=\Mor_{\Csf}(A,A)$ the identity morphism on $A$. This defined a set function $\NatTrans(h_{A},F)\to F(A)$. Conversely, given some $\alpha\in F(A)$, we can define a natural transformation $T:h_{A}\longrightarrow F$ as follows: for any $D\in\Obj(\Csf)$ an element of the set $h_{A}(D)$ is a morphism $g:D\to A$ which defines a map of sets $F(g):F(A)\to F(D)$. We define $T(D):h_{A}(D)\to F(D)$ by $g\mapsto F(g)(\alpha)$ where the latter lies in $F(D)$ by the definition of $T(D)$ above. This coheres into the data of a natural transformation by the appropriate pointwise verifications. This suggests that elements of $F(A)$ for an object $A\in\Obj(\Csf)$, can exhibit control over the functor $F$. This motivates the following discussion. 
\begin{definition}[Universal Object]\label{def: universal object}
    Let $F:\Csf^{\Opp}\to\Sets$ be a functor. A universal object for $F$ is a pair $(A,\alpha)\in\Obj(\Csf)\times F(A)$ such that for each $B\in\Obj(\Csf)$ and each $\beta\in F(B)$ there is a unique map $f\in\Mor_{\Csf}(B,A)$ such that $F(f)(\alpha)=\beta$. 
\end{definition}
From the exposition above, we can see that $(A,\alpha)$ is a universal object if the natural transformation $h_{A}\Longrightarrow F$ defined by $\alpha$ is a natural isomorphism. Since every natural transformation $h_{A}\Longrightarrow F$ is defined by some object $\alpha\in F(A)$, we can conclude the following. 
\begin{proposition}\label{prop: representable iff universal object}
    A functor $F:\Csf^{\Opp}\to\Sets$ is representable if and only if it has a universal object. 
\end{proposition}
Let us now consider some examples in the context of algebraic geometry. 
\begin{example}
    Let $S=\spec A$ and consider the affine line $\A^{1}_{S}=\spec A[t]$. We have a functor $\Ocal:\Sch_{S}^{\Opp}\to\Rings$ taking an $S$-scheme $X$ to $\Ocal(X)$ its ring of global sections. For $f:X\to Y$ a map of schemes, we get a ring homomorphism $\Ocal(Y)\to\Ocal(X)$ induced by the taking global sections of the pullback of sheaves $f^{\sharp}:\Ocal_{Y}\to f^{*}\Ocal_{X}$. Since $\A^{1}_{S}$ is an affine scheme, we have $t\in\Ocal(\A^{1}_{S})=A[t]$ and for any $S$-scheme $X$ and $g\in\Ocal(X)$ there is a unique morphism $X\to\A^{1}_{S}$ such that the pullback of $t$ to $x$ is exactly $g$. So $(\A^{1}_{S},t)$ is the universal object for and $\A^{1}_{S}$ represents the functor $\Ocal:\Sch_{S}^{\Opp}\to\Rings$. 
\end{example}
\begin{example}
    Let $S=\spec A$ and consider $\GG_{m,S}=\A^{1}_{S}\setminus\{0_{S}\}=\spec A[t,t^{-1}]$. A morphism of $S$-schemes $X\to\GG_{m,S}$ is determined by the image of $t\in\Ocal(\GG_{m,S})=A[t,t^{-1}]$ in $\Ocal(S)=A$. So $\GG_{m,S}$ represents $\Ocal(X)^{\times}$ the group of invertible sections of the structure sheaf. 
\end{example}
\section{Group Objects in Categories}
Let us recall the definition of group objects in a category. These are important examples in arithmetic and in geometry. For example, elliptic curves are group objects in the category of schemes $\Sch$ and topological groups are group objects in the category of topological spaces $\Top$. 
\begin{definition}
    Let $\Csf$ be a category with finite products and a final object $Z$. A group object is an object $X\in\Obj(\Csf)$ together with maps $m:X\times X\to X, i:X\to X, e:Z\to X$ satisfying the following axioms:
    \begin{enumerate}[label=(\alph*)]
        \item (Associativity Axiom) The diagram
        $$% https://q.uiver.app/#q=WzAsNCxbMCwwLCJYXFx0aW1lcyBYXFx0aW1lcyBYIl0sWzIsMCwiWFxcdGltZXMgWCJdLFswLDEsIlhcXHRpbWVzIFgiXSxbMiwxLCJYIl0sWzAsMSwiKG0sXFxpZF97WH0pIl0sWzEsMywibSJdLFsyLDMsIm0iLDJdLFswLDIsIihcXGlkX3tYfSxtfSIsMl1d
        \begin{tikzcd}
            {X\times X\times X} && {X\times X} \\
            {X\times X} && X
            \arrow["{(m,\id_{X})}", from=1-1, to=1-3]
            \arrow["m", from=1-3, to=2-3]
            \arrow["m"', from=2-1, to=2-3]
            \arrow["{(\id_{X},m)}"', from=1-1, to=2-1]
        \end{tikzcd}$$
        commutes. 
        \item (Identity Axiom) The two maps
        $$% https://q.uiver.app/#q=WzAsOCxbMCwwLCJYIl0sWzIsMCwiWlxcdGltZXMgWCJdLFs0LDAsIlhcXHRpbWVzIFgiXSxbNiwwLCJYIl0sWzAsMSwiWCJdLFsyLDEsIlhcXHRpbWVzIFoiXSxbNCwxLCJYXFx0aW1lcyBYIl0sWzYsMSwiWCJdLFsxLDIsIihlLFxcaWRfe1h9KSJdLFsyLDMsIm0iXSxbMCwxLCJcXHNpbSJdLFs2LDcsIm0iXSxbNSw2LCIoXFxpZF97WH0sZSkiXSxbNCw1LCJcXHNpbSJdXQ==
        \begin{tikzcd}
            X && {Z\times X} && {X\times X} && X \\
            X && {X\times Z} && {X\times X} && X
            \arrow["{(e,\id_{X})}", from=1-3, to=1-5]
            \arrow["m", from=1-5, to=1-7]
            \arrow["\sim", from=1-1, to=1-3]
            \arrow["m", from=2-5, to=2-7]
            \arrow["{(\id_{X},e)}", from=2-3, to=2-5]
            \arrow["\sim", from=2-1, to=2-3]
        \end{tikzcd}$$
        are both the identity map on $X$. 
        \item (Inverse Axiom) The maps
        $$% https://q.uiver.app/#q=WzAsNixbMCwwLCJYIl0sWzIsMCwiWFxcdGltZXMgWCJdLFs0LDAsIlgiXSxbMCwxLCJYIl0sWzIsMSwiWFxcdGltZXMgWCJdLFs0LDEsIlgiXSxbMCwxLCIoaSxcXGlkX3tYfSkiXSxbMSwyLCJtIl0sWzMsNCwiKFxcaWRfe1h9LGkpIl0sWzQsNSwibSJdXQ==
        \begin{tikzcd}
            X && {X\times X} && X \\
            X && {X\times X} && X
            \arrow["{(i,\id_{X})}", from=1-1, to=1-3]
            \arrow["m", from=1-3, to=1-5]
            \arrow["{(\id_{X},i)}", from=2-1, to=2-3]
            \arrow["m", from=2-3, to=2-5]
        \end{tikzcd}$$
        are both the composition 
        $$% https://q.uiver.app/#q=WzAsMyxbMCwwLCJYIl0sWzIsMCwiWiJdLFs0LDAsIlgiXSxbMCwxXSxbMSwyLCJlIl1d
        \begin{tikzcd}
            X && Z && X.
            \arrow[from=1-1, to=1-3]
            \arrow["e", from=1-3, to=1-5]
        \end{tikzcd}$$
    \end{enumerate}
\end{definition}
One can equivalently define a group object as an object $X\in\Obj(\Csf)$ with a functor $\Csf^{\Opp}\to\Grp$ such that the composition with the forgetful functor to $\Sets$ is $h_{X}$, or as an object $X\in\Obj(\Csf)$ and for all $Y\in\Obj(\Csf)$ a group structure on $\Mor_{\Csf}(Y,X)$ such that for all $Y\to W$ there is an induced homomorphism of groups $\Mor_{\Csf}(W,X)\to\Mor_{\Csf}(Y,X)$. 
\begin{example}
    Let $S=\spec A$ and consider $\A^{1}_{S}=\spec A[t]=\GG_{a,S}$. This is a group scheme since for any $X\in\Obj(\Sch_{S})$ there is a group structure on $\Mor_{\Csf}(X,\A^{1}_{S})=\Ocal(X)$ the global sections of the structure sheaf and for any $X\to Y$ there is a natural homomorphism of groups $\Ocal(Y)\to\Ocal(X)$ by the pullback of sheaves $f^{\sharp}:\Ocal_{Y}\to f^{*}\Ocal_{X}$. 
\end{example}
\begin{example}
    Similarly for $\GG_{m,S}=\spec A[t,t^{-1}]$ we know it represents the functor sending a scheme $X$ to its invertible sections $\Ocal(X)^{\times}$. For $X\to Y$, the homomorphism $\Ocal(Y)\to\Ocal_{X}$ by $\Ocal_{Y}\to f^{*}\Ocal_{X}$ restricts to a homomorphism of multiplicative groups $\Ocal(Y)^{\times}\to\Ocal(X)^{\times}$ as homomorphisms of rings send 0 to 0 and 1 to 1. 
\end{example}
\begin{example}
    Consider a functor $\Sch_{S}^{\Opp}\to\Sets$ by $X\mapsto M_{n}(\Ocal(X))$, $n\times n$ matrices with entries in $\Ocal(X)$ represented by the scheme $\A^{n\times n}_{S}$. One can apply the determinant map on matrices $\det:\A^{n\times n}_{S}\to \A^{1}_{S}$ and consider the preimage of $\GG_{m,S}\subseteq\A^{1}_{S}$, those matrices whose determinant is nonvanishing. This is an open subscheme of invertible $n\times n$ matrices $\mathrm{GL}_{n,S}\subseteq\A^{n\times n}_{S}$ which has a group structure by matrix multiplication. 
\end{example}
\begin{definition}[Homomorphism of Group Objects]\label{def: group object homomorphism}
    Let $\Csf$ be a category with finite products and final object $Z$. Let $X,Y\in\Obj(\Csf)$ be group objects. A homomorphism of group objects is a morphism $f:X\to Y$ in $\Csf$ such that the diagram 
    $$% https://q.uiver.app/#q=WzAsNCxbMCwwLCJYXFx0aW1lcyBYIl0sWzAsMSwiWVxcdGltZXMgWSJdLFsyLDAsIlgiXSxbMiwxLCJZIl0sWzAsMSwiZlxcdGltZXMgZiIsMl0sWzEsMywibV97WX0iLDJdLFswLDIsIm1fe1h9Il0sWzIsMywiZiJdXQ==
    \begin{tikzcd}
        {X\times X} && X \\
        {Y\times Y} && Y
        \arrow["{f\times f}"', from=1-1, to=2-1]
        \arrow["{m_{Y}}"', from=2-1, to=2-3]
        \arrow["{m_{X}}", from=1-1, to=1-3]
        \arrow["f", from=1-3, to=2-3]
    \end{tikzcd}$$
    commutes. 
\end{definition}
Group objects are inherently interesting objects of study in algebraic geometry, such as in the study of Abelian varieties. However, group objects also act on other objects in the category, creating a new direction of study. 
\begin{definition}[Action of Group Object]\label{def: action of group object}
    Let $\Csf$ be a category. A left action $\alpha$ of a functor $G:\Csf^{\Opp}\to\Grp$ on a functor $F:\Csf^{\Opp}\to\Sets$ is a natural transformation $\alpha:G\times F\to F$ such that for all $Y\in\Obj(\Csf)$ the induced map $\alpha(Y):G(Y)\times F(Y)\to F(Y)$ is an action of the group $G(Y)$ on the set $F(Y)$. 
\end{definition}
For a group object in a category, its action on another object is given by the action of the functor the group object represents to the functor that the other object represents. 
\begin{definition}[Equivariant]
    Let $\Csf$ be a category and $X,Y\in\Obj(\Csf)$ objects with an action by a group object $G$. A morphism $f:X\to Y$ is $G$-equivariant if the diagram
    $$% https://q.uiver.app/#q=WzAsNCxbMCwwLCJHXFx0aW1lcyBYIl0sWzAsMSwiR1xcdGltZXMgWSJdLFsyLDAsIlgiXSxbMiwxLCJZIl0sWzIsMywiZiJdLFswLDEsIlxcaWRfe0d9XFx0aW1lcyBmIiwyXSxbMCwyXSxbMSwzXV0=
    \begin{tikzcd}
        {G\times X} && X \\
        {G\times Y} && Y
        \arrow["f", from=1-3, to=2-3]
        \arrow["{\id_{G}\times f}"', from=1-1, to=2-1]
        \arrow[from=1-1, to=1-3]
        \arrow[from=2-1, to=2-3]
    \end{tikzcd}$$
    with horizontal maps given by the action commutes. 
\end{definition}
This is equivalent to the data of the data of $X(W)\to Y(W)$ being $G(W)$-equivariant for all $W\in\Obj(\Csf)$. 
\printbibliography
\end{document}