\documentclass{amsart}
\usepackage[margin=1.1in]{geometry} 
\usepackage{amsmath}
\usepackage{tcolorbox}
\usepackage{amssymb}
\usepackage{amsthm}
\usepackage{lastpage}
\usepackage{fancyhdr}
\usepackage{accents}
\usepackage{hyperref}
\usepackage{xcolor}
\usepackage{color}
% Fields
\newcommand{\CC}{\mathbb{C}}
\newcommand{\RR}{\mathbb{R}}
\newcommand{\QQ}{\mathbb{Q}}
\newcommand{\ZZ}{\mathbb{Z}}
\newcommand{\NN}{\mathbb{N}}
\newcommand{\FF}{\mathbb{F}}
\newcommand{\PP}{\mathbb{P}}
\newcommand{\GG}{\mathbb{G}}

% mathcal letters
\newcommand{\Acal}{\mathcal{A}}
\newcommand{\Bcal}{\mathcal{B}}
\newcommand{\Ccal}{\mathcal{C}}
\newcommand{\Dcal}{\mathcal{D}}
\newcommand{\Ecal}{\mathcal{E}}
\newcommand{\Fcal}{\mathcal{F}}
\newcommand{\Gcal}{\mathcal{G}}
\newcommand{\Hcal}{\mathcal{H}}
\newcommand{\Ical}{\mathcal{I}}
\newcommand{\Jcal}{\mathcal{J}}
\newcommand{\Kcal}{\mathcal{K}}
\newcommand{\Lcal}{\mathcal{L}}
\newcommand{\Mcal}{\mathcal{M}}
\newcommand{\Ncal}{\mathcal{N}}
\newcommand{\Ocal}{\mathcal{O}}
\newcommand{\Pcal}{\mathcal{P}}
\newcommand{\Qcal}{\mathcal{Q}}
\newcommand{\Rcal}{\mathcal{R}}
\newcommand{\Scal}{\mathcal{S}}
\newcommand{\Tcal}{\mathcal{T}}
\newcommand{\Ucal}{\mathcal{U}}
\newcommand{\Vcal}{\mathcal{V}}
\newcommand{\Wcal}{\mathcal{W}}
\newcommand{\Xcal}{\mathcal{X}}
\newcommand{\Ycal}{\mathcal{Y}}
\newcommand{\Zcal}{\mathcal{Z}}

% abstract categories
\newcommand{\Asf}{\mathsf{A}}
\newcommand{\Bsf}{\mathsf{B}}
\newcommand{\Csf}{\mathsf{C}}
\newcommand{\Dsf}{\mathsf{D}}
\newcommand{\Esf}{\mathsf{E}}
\newcommand{\Ssf}{\mathsf{S}}

% algebraic geometry
\newcommand{\spec}{\operatorname{Spec}}
\newcommand{\proj}{\operatorname{Proj}}

% categories 
\newcommand{\id}{\mathrm{id}}
\newcommand{\Obj}{\mathrm{Obj}}
\newcommand{\Mor}{\mathrm{Mor}}
\newcommand{\Hom}{\mathrm{Hom}}
\newcommand{\Aut}{\mathrm{Aut}}
\newcommand{\Sets}{\mathsf{Sets}}
\newcommand{\SSets}{\mathsf{SSets}}
\newcommand{\kVec}{\mathsf{Vec}_{k}}
\newcommand{\Alg}{\mathsf{Alg}}
\newcommand{\Ring}{\mathsf{Ring}}
\newcommand{\Mod}{\mathsf{Mod}}
\newcommand{\Grp}{\mathsf{Grp}}
\newcommand{\AbGrp}{\mathsf{AbGrp}}
\newcommand{\PSh}{\mathsf{PSh}}
\newcommand{\Sh}{\mathsf{Sh}}
\newcommand{\PSch}{\mathsf{PSch}}
\newcommand{\Sch}{\mathsf{Sch}}
\newcommand{\Top}{\mathsf{Top}}
\newcommand{\Com}{\mathsf{Com}}
\newcommand{\Coh}{\mathsf{Coh}}
\newcommand{\QCoh}{\mathsf{QCoh}}
\newcommand{\Opens}{\mathsf{Opens}}
\newcommand{\Opp}{\mathsf{Opp}}
\newcommand{\Cat}{\mathsf{Cat}}
\newcommand{\NatTrans}{\mathrm{NatTrans}}
\newcommand{\pr}{\mathrm{pr}}
\newcommand{\Fun}{\mathrm{Fun}}
\newcommand{\colim}{\mathrm{colim}}
\newcommand{\lifts}{\boxslash}
\DeclareMathOperator\squarediv{\lifts}

% simplicial sets
\newcommand{\DDelta}{\Updelta}
\newcommand{\Sing}{\operatorname{Sing}}

% ideal theory
\newcommand{\mfrak}{\mathfrak{m}}
\newcommand{\afrak}{\mathfrak{a}}
\newcommand{\bfrak}{\mathfrak{b}}
\newcommand{\pfrak}{\mathfrak{p}}
\newcommand{\qfrak}{\mathfrak{q}}

% number theory
\newcommand{\Tr}{\mathrm{Tr}}
\newcommand{\Nm}{\mathrm{Nm}}
\newcommand{\Gal}{\mathrm{Gal}}
\newcommand{\Frob}{\mathrm{Frob}}

% stacks
\newcommand{\et}{\mathsf{\acute{e}t}}
\newcommand{\fppf}{\mathsf{fppf}}
\newcommand{\fpqc}{\mathsf{fpqc}}
\newcommand{\Nis}{\mathsf{Nis}}
\newcommand{\Zar}{\mathsf{Zar}}
\setlength{\headheight}{40pt}


\newenvironment{solution}
  {\renewcommand\qedsymbol{$\blacksquare$}
  \begin{proof}[Solution]}
  {\end{proof}}
\renewcommand\qedsymbol{$\blacksquare$}

\usepackage{amsmath, amssymb, tikz, amsthm, csquotes, multicol, footnote, tablefootnote, biblatex, wrapfig, float, quiver, mathrsfs, cleveref, enumitem, upgreek,stmaryrd}
\addbibresource{refs.bib}
\theoremstyle{definition}
\newtheorem{theorem}{Theorem}[section]
\newtheorem{lemma}[theorem]{Lemma}
\newtheorem{corollary}[theorem]{Corollary}
\newtheorem{exercise}[theorem]{Exercise}
\newtheorem{question}[theorem]{Question}
\newtheorem{example}[theorem]{Example}
\newtheorem{proposition}[theorem]{Proposition}
\newtheorem{conjecture}[theorem]{Conjecture}
\newtheorem*{remark}{Remark}
\newtheorem{definition}[theorem]{Definition}
\numberwithin{equation}{section}
\begin{document}
\large
\title[Algebraic Stacks]{Algebraic Stacks}
\author{Wern Juin Gabriel Ong}
\address{Bowdoin College, Brunswick, Maine 04011}
\email{gong@bowdoin.edu}
\urladdr{https://wgabrielong.github.io/}
\maketitle
\section*{Overview}
These notes roughly correspond to an attempt to learn stack theory that began in the winter of 2023. We will begin with the categorical preliminaries as laid out in \cite{Vistoli} before developing the basic theory per the text of Olsson \cite{Olsson} and conclude with a sampling of the more advanced topics in \cite[Part 7]{stacks-project}. The standard texts are \cite{LaumonMoret-Bailly} and \cite{Olsson}. The compendium \cite{stacks-project} is encyclopedic. 
\tableofcontents
\newpage
\part*{Grothendieck Topologies, Sites, and Fibered Categories}
\section{Grothendieck Topologies}
Let us recall the following definitions. 
\begin{definition}[Presheaf]\label{def: presheaf on topological space}
    Let $X$ be a topological space. A presheaf of sets $\Fcal$ on $X$ is a functor $(X^{\Opens})^{\Opp}\to\Sets$. 
\end{definition}
A presheaf is a sheaf if it satisfies additional gluing axioms. 
\begin{definition}[Sheaf]\label{def: topological sheaf}
    Let $X$ be a topological space. A sheaf of sets $\Fcal$ on $X$ is a functor $(X^{\Opens})^{\Opp}\to\Sets$ such that the sequence 
    $$% https://q.uiver.app/#q=WzAsMyxbMCwwLCJcXEZjYWwoWCkiXSxbMiwwLCJcXHByb2Rfe2l9XFxGY2FsKFhfe2l9KSJdLFs0LDAsIlxccHJvZF97aSxqfVxcRmNhbChYX3tpfVxcY2FwIFhfe2p9KSJdLFswLDFdLFsxLDIsIiIsMCx7Im9mZnNldCI6LTF9XSxbMSwyLCIiLDAseyJvZmZzZXQiOjF9XV0=
    \begin{tikzcd}
        {\Fcal(X)} && {\prod_{i}\Fcal(X_{i})} && {\prod_{i,j}\Fcal(X_{i}\cap X_{j})}
        \arrow[from=1-1, to=1-3]
        \arrow[shift left, from=1-3, to=1-5]
        \arrow[shift right, from=1-3, to=1-5]
    \end{tikzcd}$$
    is an equalizer for $\{X_{i}\}$ an open cover of $X$.  
\end{definition}
In this way, we say that a sheaf on $X$ is a presheaf on $X$ satisfying descent. Note here that we implicitly used the fact that $X^{\Opens}$ can be naturally endowed with the strucutre of a category with objects open sets of the topological space $X$ and morphisms inclusions of such open sets. This begs the question if we can replace $X^{\Opens}$ with some other category $\Csf$, allowing us to define a sheaf on an arbitrary category $\Csf$. We do this via the construction of a Grothendieck topology by replacing open sets of a topological space with maps into this space. 
\begin{definition}[Grothendieck Topology]\label{def: Grothendieck topology}
    Let $\Csf$ be a category. A Grothendieck topology on $\Csf$ is the data of a set $\{X_{i}\to X\}$ for each object $X\in\Obj(\Csf)$ known as a covering of $X$ such that the following conditions hold:
    \begin{enumerate}[label=(\alph*)]
        \item If $Y\to X$ is an isomorphism then $\{Y\to X\}$ is a covering.  
        \item If $\{X_{i}\to X\}$ is a covering and $Y\to X$ any morphism then $\{X_{i}\times_{X}Y\}$ exist and $\{X_{i}\times_{X}Y\to X\}$ is a covering. 
        \item If $\{X_{i}\to X\}$ is a covering and for each $i$ $\{X_{ij}\to X_{i}\}$ is a covering then the composites $\{X_{ij}\to X_{i}\to X\}$ is a covering. 
    \end{enumerate}
\end{definition}
This allows us to define a site. 
\begin{definition}[Site]\label{def: site}
    A site on a category $\Csf$ is the category $\Csf$ endowed with a Grothendieck topology. 
\end{definition}
Let us see some examples.
\begin{example}[Site of a Topological Space]
    Let $X$ be a topological space and $X^{\Opens}$ the category of open sets of $X$ with morphisms inclusions. One can endow $X^{\Opens}$ with a Grothendieck topology by associating to each $U\subseteq X$ open, the set of open covers of $U$. The fiber product is given by intersection of open sets, which agrees with our previous discussion of sheaves and presheaves. 
\end{example}
\begin{example}
    Consider the category of topological spaces $\Top$. The category $\Top$ can be endowed with a Grothendieck topology by taking covers of a topological space $X\in\Obj(\Top)$ to open, continuous, injective maps $X_{i}\to X$. 
\end{example}
\begin{example}[\'{E}tale Site of a Scheme]
    Let $X$ be a scheme. The \'{e}tale site of $X$, denoted $X_{\et}$, is the full subcategory of $\Sch_{X}$ where covers are \'{e}tale morphisms. 
\end{example}
\section{Fibered Categories}\label{sec: fibered categories}
Fix a base category $\Ssf$. We consider categories over $\Ssf$. That is, those categories $\Fsf$ with a functor $p:\Fsf\to\Ssf$. More explicitly, we have the following definition. 
\begin{definition}[Category Over]\label{def: category over}
    We say that $\Fsf$ is a category over $\Ssf$ if there exists a functor $p:\Fsf\to\Ssf$. 
\end{definition}
The relationship between the categories $\Fsf$ and $\Ssf$ are given by ``lying over'' in the following sense. 
\begin{definition}[Lying Over]
    Consider $\Fsf$ over $\Ssf$ with $p:\Fsf\to\Ssf$. 
    \begin{enumerate}[label=(\alph*)]
        \item (Objects) An object $\alpha\in\Obj(\Fsf)$ lies over $A\in\Obj(\Ssf)$ if $p(\alpha)=A$. 
        \item (Morphisms) An morphism $\phi:\alpha\to\beta$ in $\Fsf$ lies over $f:A\to B$ in $\Ssf$ if the diagram 
        $$% https://q.uiver.app/#q=WzAsNCxbMCwwLCJcXGFscGhhIl0sWzIsMCwiXFxiZXRhIl0sWzAsMSwiQSJdLFsyLDEsIkIiXSxbMCwxLCIoXFxhbHBoYVxcdG9cXGJldGEpIl0sWzEsMywicCgtKSJdLFswLDIsInAoLSkiLDJdLFsyLDMsInAoXFxhbHBoYVxcdG9cXGJldGEpIiwyXV0=
        \begin{tikzcd}
            \alpha && \beta \\
            A && B
            \arrow["{\phi}", from=1-1, to=1-3]
            \arrow["{p(-)}", from=1-3, to=2-3]
            \arrow["{p(-)}"', from=1-1, to=2-1]
            \arrow["{p(\phi)=f}"', from=2-1, to=2-3]
        \end{tikzcd}$$
        commutes. 
    \end{enumerate}
\end{definition}
This allows us to define the notion of a Cartesian morphism in $\Fsf$. 
\begin{definition}[Cartesian Morphism]\label{def: cartesian morphism}
    Let $\Fsf$ be a category over $\Ssf$. A morphism $\phi:\alpha\to\beta$ in $\Fsf$ is Cartesian if for any $\psi:\eta\to\beta$ in $\Fsf$ and $g:p(\eta)\to p(\alpha)$ in $\Ssf$ with $p(\phi)\circ g= p(\psi)$ in $\Ssf$ there exists a unique $\rho:\eta\to\alpha$ lying over $g$ making the diagram 
    $$% https://q.uiver.app/#q=WzAsNixbMCwwLCJcXGV0YSJdLFsyLDEsIlxcYWxwaGEiXSxbMiwyLCJwKFxcYWxwaGEpIl0sWzAsMSwicChcXGV0YSkiXSxbNCwxLCJcXGJldGEiXSxbNCwyLCJwKFxcYmV0YSkiXSxbMCw0LCJcXHBzaSIsMV0sWzAsMSwiXFxleGlzdHMhXFxyaG8iLDEseyJzdHlsZSI6eyJib2R5Ijp7Im5hbWUiOiJkYXNoZWQifX19XSxbMCwzXSxbMywyLCJnIiwxXSxbMSwyXSxbMyw1XSxbMSw0LCJcXHBoaSIsMV0sWzIsNV0sWzQsNV1d
    \begin{tikzcd}
        \eta \\
        {p(\eta)} && \alpha && \beta \\
        && {p(\alpha)} && {p(\beta)}
        \arrow["\psi"{description}, from=1-1, to=2-5]
        \arrow["{\exists!\rho}"{description}, dashed, from=1-1, to=2-3]
        \arrow[from=1-1, to=2-1]
        \arrow["g"{description}, from=2-1, to=3-3]
        \arrow[from=2-3, to=3-3]
        \arrow[from=2-1, to=3-5]
        \arrow["\phi"{description}, from=2-3, to=2-5]
        \arrow[from=3-3, to=3-5]
        \arrow[from=2-5, to=3-5]
    \end{tikzcd}$$
    commute, and universal with respect to that property. 
\end{definition}
One can show that Cartesian morphisms satisfy the following properties. 
\begin{proposition}
    Let $\Fsf$ be a category over $\Ssf$. 
    \begin{enumerate}[label=(\alph*)]
        \item The composite of Cartesian arrows in $\Fsf$ is Cartesian. 
        \item If $\alpha\to\beta$ and $\beta\to\eta$ are morphisms in $\Fsf$ and $\beta\to\eta$ is Cartesian then $\alpha\to\beta$ is Cartesian if and only if the composite $\alpha\to\beta\to\eta$ is Cartesian. 
        \item A morphism $\phi$ in $\Fsf$ such that $p(\phi)$ is an isomorphism in $\Ssf$ is Cartesian if and only if $\phi$ is an isomorphism in $\Fsf$. 
        \item Let $p:\Csf\to\Ssf$ and $F:\Fsf\to\Csf$ be functors and $\phi:\alpha\to\beta$ a morphism in $\Fsf$. If $\phi$ is Cartesian over $F(\phi):F(\alpha)\to F(\beta)$ in $\Csf$ and $F(\phi)$ is Cartesian over $p(F(\phi)):p(F(\alpha))\to p(F(\beta))$ in $\Ssf$ then $\phi$ is Cartesian over $p(F(\phi)):p(F(\alpha))\to p(F(\beta))$ in $\Ssf$. 
    \end{enumerate}
\end{proposition}
This in turn allows us to define fibered categories. 
\begin{definition}[Fibered Category]\label{def: fibered category}
    Let $\Fsf$ be a category over $\Ssf$. $\Fsf$ is a fibered category over $\Ssf$ if for all $f:A\to B$ in $\Ssf$ and $\beta$ lying over $B$ there is a Cartesian morphism $\phi:\alpha\to\beta$ in $\Fsf$ lying over $f$ in $\Ssf$. 
\end{definition}
This means that for $p:\Fsf\to\Ssf$ a fibered category, objects of $\Fsf$ can be pulled back along any morphism in $\Ssf$ and that these pullbacks are unique up to unique isomorphism. Naturally we want these fibered functors to amalgamate into the data of a (possibly higher) category. To that end, we define morphisms of fibered categories as follows. 
\begin{definition}[Morphism of Fibered Categories]\label{def: morphism of fibered categories}
    Let $p:\Fsf\to\Ssf, q:\Gsf\to\Ssf$ be categories fibered over $\Ssf$. A morphism of fibered categories is a functor $F:\Fsf\to\Gsf$ such that $q\circ F=p$ and $F$ takes Cartesian morphisms in $\Fsf$ to Cartesian morphisms in $\Gsf$. 
\end{definition}
The language of fibered categories should be quite suggestive of analogous concepts in algebraic topology. Given a topological space $B$ and $E\to B$ a fibration, one can consider $E_{b}$ the fiber over $b\in B$. In the context of fibered categories, we define the following. 
\begin{definition}[Fiber]\label{def: fiber}
    Let $p:\Fsf\to\Ssf$ be a fibered category. For $X\in\Obj(\Ssf)$ the fiber $\Fsf_{X}$ is the subcategory of $\Fsf$ whose objects are those lying over $X$ and whose morphisms are those lying over $\id_{X}$. 
\end{definition}
For a morphism of fibered categories $F$ as in \Cref{def: morphism of fibered categories}, the functor $F$ sends $\Fsf_{X}$ to $\Gsf_{X}$ and hence restricts to a subfunctor $F_{X}:\Fsf_{X}\to\Gsf_{X}$. Let us consider how this definition of a fiber is reasonably compatible with the pullback structure we have previously discussed. 
\\\\
Let $\Fsf$ be fibered over $\Ssf$ and $f:A\to B$ a morphism in $\Ssf$. For each $\beta$ lying over $B$ in $\Fsf$ we choose a pullback $\phi_{\beta}:f^{*}\beta\to\beta$ with $\phi_{\beta}$ lying over $f$ and $f^{*}\beta$ lying over $A$ we can define a functor on the fiber categories $f^{*}:\Fsf_{B}\to\Fsf_{A}$ by defining a map on objects $\beta\mapsto f^{*}\beta$ and for $\tau:\beta\to\beta'$ over $\id_{B}$ in $\Fsf_{B}$ the unique map $f^{*}\tau:f^{*}\beta\to f^{*}\beta'$ that makes the diagram 
$$% https://q.uiver.app/#q=WzAsNCxbMCwwLCJmXnsqfVxcYmV0YSJdLFswLDEsImZeeyp9XFxiZXRhJyJdLFsyLDAsIlxcYmV0YSJdLFsyLDEsIlxcYmV0YSciXSxbMCwyLCJcXHBoaV97XFxiZXRhfSJdLFsxLDMsIlxccGhpX3tcXGJldGEnfSIsMl0sWzAsMSwiZl57Kn1cXHRhdSIsMix7InN0eWxlIjp7ImJvZHkiOnsibmFtZSI6ImRhc2hlZCJ9fX1dLFsyLDMsIlxcdGF1Il1d
\begin{tikzcd}
	{f^{*}\beta} && \beta \\
	{f^{*}\beta'} && {\beta'}
	\arrow["{\phi_{\beta}}", from=1-1, to=1-3]
	\arrow["{\phi_{\beta'}}"', from=2-1, to=2-3]
	\arrow["{f^{*}\tau}"', dashed, from=1-1, to=2-1]
	\arrow["\tau", from=1-3, to=2-3]
\end{tikzcd}$$
commute. 
\begin{definition}[Cleavage]\label{def: cleavage}
    Let $p:\Fsf\to\Ssf$ be a fibered category. A cleavage of $\Fsf$ is a collection of Cartesian morphisms $K$ of $\Fsf$ such that for each $f:A\to B$ in $\Ssf$ and $\beta\in\Obj(\Fsf_{B})$ there is $\alpha\in\Fsf_{A}$ such that $p(\alpha)=A$ and a unique Cartesian morphism $\phi:\alpha\to\beta$ in $K$ lying over $f$. 
\end{definition}
Visually we have the following diagram 
$$% https://q.uiver.app/#q=WzAsNCxbMCwwLCJcXGFscGhhIl0sWzIsMCwiXFxiZXRhIl0sWzAsMSwiQSJdLFsyLDEsIkIiXSxbMiwzLCJcXHBoaSIsMl0sWzAsMSwiZiJdLFswLDIsInAiLDJdLFsxLDMsInAiXV0=
\begin{tikzcd}
	\alpha && \beta \\
	A && B
	\arrow["f"', from=2-1, to=2-3]
	\arrow["\phi", from=1-1, to=1-3]
	\arrow["p"', from=1-1, to=2-1]
	\arrow["p", from=1-3, to=2-3]
\end{tikzcd}$$
where $\phi$ is a Cartesian morphism in $K$. 
\subsection{Higher Functors and the 2-Category $\Cat$}\label{subsec: higher functors}
By the axiom of choice, every fibered category has a cleavage. For $p:\Fsf\to\Ssf$ a fibered category and each $f:A\to B$ there is a functor $f^{*}:\Fsf_{B}\to\Fsf_{A}$ between the fibers. However, there are several fundamental issues. Firstly, the pullback along identities $\id_{A}^{*}:\Fsf_{A}\to\Fsf_{A}$ need not be identities; secondly, the category of categories does not form a category itself, but instead a 2-category $\Cat$ -- the 2-category $\Cat$ then would have objects categories, morphisms functors between these categories, and 2-morphisms natural transformations between functors. Instead of a functor, we get a lax 2-functor, which we now define. 
\newpage
\begin{definition}[Lax 2-Functor]\label{def: lax 2 functor}
    Let $\Ssf$ be a category. A lax 2-functor $\Phi$ on $\Ssf$ consists of the following data. 
    \begin{enumerate}[label=(\alph*)]
        \item For each $X\in\Obj(\Ssf)$ a category $\Phi\left(X\right)$.
        \item For each $(f:X\to Y)\in\Mor_{\Ssf}$ a functor $f^{*}:\Phi\left(Y\right)\to\Phi\left(X\right)$. 
        \item For each $X\in\Obj(\Csf)$ a natural isomorphism $\varepsilon_{X}:\id_{X}^{*}\Longrightarrow \id_{\Phi\left(X\right)}$ between functors $\Phi\left(X\right)\to\Phi\left(X\right)$. 
        \item For $(f:X\to Y), (g:Y\to Z)\in\Mor_{\Ssf}$ a natural isomorphism $\alpha_{f,g}:f^{*}g^{*}\Longrightarrow (fg)^{*}$ between functors $\Phi\left(X\right)\to\Phi\left(Z\right)$ such that the following conditions hold:
        \begin{enumerate}[label=(\roman*)]
            \item For $\beta\in\Obj\left(\Phi\left(Y\right)\right)$ we have 
            $$\alpha_{\id_{X},f}(\beta)=\varepsilon_{X}\left(f^{*}\beta\right):\id_{X}^{*}f^{*}\beta\to f^{*}\beta$$
            and
            $$\alpha_{f,\id_{Y}}(\beta)=f^{*}\varepsilon_{Y}:f^{*}\id_{Y}^{*}\beta\to f^{*}\beta.$$
            \item If also we have $(h:Z\to W)\in\Mor_{\Ssf}$ and $\gamma\in\Obj\left(\Phi\left(W\right)\right)$ the diagram 
            $$% https://q.uiver.app/#q=WzAsNCxbMCwwLCJmXnsqfWdeeyp9aF57Kn1cXGdhbW1hIl0sWzIsMCwiKGdmKV57Kn1oXnsqfVxcZ2FtbWEiXSxbMCwxLCJmXnsqfShnaCleeyp9XFxnYW1tYSJdLFsyLDEsIihoZ2YpXnsqfVxcZ2FtbWEiXSxbMCwyLCJmXnsqfVxcYWxwaGFfe2csaH0oXFxnYW1tYSkiLDJdLFsxLDMsIlxcYWxwaGFfe2dmLGh9KFxcZ2FtbWEpIl0sWzAsMSwiXFxhbHBoYV97ZixnfShoXnsqfVxcZ2FtbWEpIl0sWzIsMywiXFxhbHBoYV97ZixoZ30oXFxnYW1tYSkiLDJdXQ==
            \begin{tikzcd}
                {f^{*}g^{*}h^{*}\gamma} && {(gf)^{*}h^{*}\gamma} \\
                {f^{*}(gh)^{*}\gamma} && {(hgf)^{*}\gamma}
                \arrow["{f^{*}\alpha_{g,h}(\gamma)}"', from=1-1, to=2-1]
                \arrow["{\alpha_{gf,h}(\gamma)}", from=1-3, to=2-3]
                \arrow["{\alpha_{f,g}(h^{*}\gamma)}", from=1-1, to=1-3]
                \arrow["{\alpha_{f,hg}(\gamma)}"', from=2-1, to=2-3]
            \end{tikzcd}$$
            commutes. 
        \end{enumerate}
    \end{enumerate}
\end{definition}
This construction of a lax 2-functor rigidifies the construction of a (higher) functor into $\Cat$, the category of categories. Indeed, \emph{a priori}, even if it is natural to do so, there is neither a reason why the functor induced by pullback along $(\id_{X}:X\to X)\in\Mor_{\Ssf}$ induces an equivalence of categories $\id_{\Phi\left(X\right)}:\Phi\left(X\right)\to\Phi\left(X\right)$ nor is there a reason why for $(f:X\to Y),(g:Y\to Z)\in\Mor_{\Ssf}$ the functors 
$$% https://q.uiver.app/#q=WzAsMyxbMCwwLCJcXFBoaVxcbGVmdChaXFxyaWdodCkiXSxbMiwwLCJcXFBoaVxcbGVmdChZXFxyaWdodCkiXSxbNCwwLCJcXFBoaVxcbGVmdChaXFxyaWdodCkiXSxbMCwxLCJnXnsqfSJdLFsxLDIsImZeeyp9Il1d
\begin{tikzcd}
	{\Phi\left(Z\right)} && {\Phi\left(Y\right)} && {\Phi\left(Z\right)}
	\arrow["{g^{*}}", from=1-1, to=1-3]
	\arrow["{f^{*}}", from=1-3, to=1-5]
\end{tikzcd}$$
necessarily agrees with the composition
$$% https://q.uiver.app/#q=WzAsMixbMCwwLCJcXFBoaVxcbGVmdChaXFxyaWdodCkiXSxbNCwwLCJcXFBoaVxcbGVmdChaXFxyaWdodCkiXSxbMCwxLCIoZmcpXnsqfSJdXQ==
\begin{tikzcd}
	{\Phi\left(Z\right)} &&&& {\Phi\left(Z\right)}.
	\arrow["{(fg)^{*}}", from=1-1, to=1-5]
\end{tikzcd}$$
This makes the diagram
$$% https://q.uiver.app/#q=WzAsMyxbMCwwLCJcXFBoaVxcbGVmdChYXFxyaWdodCkiXSxbMiwwLCJcXFBoaVxcbGVmdChaXFxyaWdodCkiXSxbMSwxLCJcXFBoaVxcbGVmdChZXFxyaWdodCkiXSxbMSwyLCJnXnsqfSJdLFsyLDAsImZeeyp9Il0sWzEsMCwiKGZnKV57Kn0iLDJdLFs1LDIsIlxcYWxwaGFfe2YsZ30iLDEseyJsYWJlbF9wb3NpdGlvbiI6NjAsInNob3J0ZW4iOnsic291cmNlIjoyMH0sInN0eWxlIjp7InRhaWwiOnsibmFtZSI6ImFycm93aGVhZCJ9LCJoZWFkIjp7Im5hbWUiOiJub25lIn19fV1d
\begin{tikzcd}
	{\Phi\left(X\right)} && {\Phi\left(Z\right)} \\
	& {\Phi\left(Y\right)}
	\arrow["{g^{*}}", from=1-3, to=2-2]
	\arrow["{f^{*}}", from=2-2, to=1-1]
	\arrow[""{name=0, anchor=center, inner sep=0}, "{(fg)^{*}}"', from=1-3, to=1-1]
	\arrow["{\alpha_{f,g}}"{description, pos=0.6}, shorten <=3pt, Rightarrow, 2tail reversed, no head, from=0, to=2-2]
\end{tikzcd}$$
rigid in the sense that the composition is unique. 
\begin{remark}
    The astute would note this is the definition of a contravariant lax 2-functor. We omit the more general definitions for ease of exposition. 
\end{remark}
\begin{remark}
    Those familiar with higher categorical language will note that the exposition above is equivalent to the 2-category of categories $\Cat$ being a strict 2-category, in the sense that all coherence data above the level of 2-morphisms are rigid. 
\end{remark}
One would expect a fibered category $p:\Fsf\to\Ssf$ equipped with a cleavage gives rise to a lax 2-functor -- the cleavage here gives uniqueness of the functor up to a choice of preimage. This is indeed the case as we now show. 
\newpage
\begin{lemma}\label{lem: fibered cat with cleavage is lax2}
    Let $p:\Fsf\to\Ssf$ be a fibered category. $p$ defines a lax 2-functor to the category of categories associating to each object $A\in\Obj(\Ssf)$ a category
    $$A\mapsto \Fsf_{A}$$
    and to each morphism a functor
    $$(f:A\to B)\mapsto (f^{*}:\Fsf_{B}\to\Fsf_{A}).$$
\end{lemma}
\begin{proof}
    Evidently (a) and (b) are fulfilled. (c) follows from the fact that pullbacks are unique up to unique isomorphism so $\id_{A}^{*}:\Fsf_{A}\to\Fsf_{A}$ is naturally isomorphic to the identity functor. For (d) any functor completing the solid diagram 
    $$% https://q.uiver.app/#q=WzAsMyxbMCwwLCJcXEZzZl97WH0iXSxbMiwwLCJcXEZzZl97Wn0iXSxbMSwxLCJcXEZzZl97WX0iXSxbMSwyLCJnXnsqfSJdLFsyLDAsImZeeyp9Il0sWzEsMCwiIiwyLHsic3R5bGUiOnsiYm9keSI6eyJuYW1lIjoiZGFzaGVkIn19fV1d
    \begin{tikzcd}
        {\Fsf_{A}} && {\Fsf_{C}} \\
        & {\Fsf_{B}}
        \arrow["{g^{*}}", from=1-3, to=2-2]
        \arrow["{f^{*}}", from=2-2, to=1-1]
        \arrow[dashed, from=1-3, to=1-1]
    \end{tikzcd}$$
    it is unique since $\id_{\Fsf_{A}}:\Fsf_{A}\to\Fsf_{A}$ is Cartesian. 
    \\\\
    (d)(i) is given by the commutativity of the following diagram 
    $$% https://q.uiver.app/#q=WzAsMTIsWzAsMCwiXFxpZF97QX1eeyp9Zl57Kn1cXGJldGEiXSxbMCwxLCJBIl0sWzIsMiwiZl57Kn1cXGJldGEiXSxbMiwzLCJBIl0sWzQsMSwiXFxiZXRhIl0sWzQsMiwiQiJdLFs2LDAsImZeeyp9XFxpZF97Qn1eeyp9XFxiZXRhIl0sWzYsMSwiQSJdLFs4LDIsIlxcaWRfe0J9XnsqfVxcYmV0YSJdLFs4LDMsIkIiXSxbMTAsMSwiXFxiZXRhIl0sWzEwLDIsIkIiXSxbNCw1XSxbMiwzXSxbMyw1LCJmIiwxXSxbMSw1LCJmIiwxLHsibGFiZWxfcG9zaXRpb24iOjYwLCJjdXJ2ZSI6LTJ9XSxbMSwzLCJcXGlkX3tBfSIsMV0sWzIsMCwiXFxpZF97QX1eeyp9IiwxXSxbMCwxXSxbNCwwLCJcXGlkX3tcXEZzZl97QX19Zl57Kn0iLDEseyJjdXJ2ZSI6Mn1dLFs0LDIsImZeeyp9IiwxXSxbNyw5LCJmIiwxXSxbOCw2LCJmXnsqfSIsMSx7ImxhYmVsX3Bvc2l0aW9uIjozMH1dLFs2LDddLFs4LDldLFs5LDExLCJcXGlkX3tCfSIsMV0sWzEwLDgsIlxcaWRfe0J9XnsqfSIsMSx7ImxhYmVsX3Bvc2l0aW9uIjo0MH1dLFsxMCw2LCJmXnsqfVxcaWRfe1xcRnNmX3tCfX0iLDEseyJjdXJ2ZSI6Mn1dLFsxMCwxMV0sWzcsMTEsImYiLDEseyJjdXJ2ZSI6LTJ9XSxbMjcsMjIsIiIsMSx7InNob3J0ZW4iOnsic291cmNlIjoyMCwidGFyZ2V0IjoyMH19XSxbMTksMTcsIiIsMSx7InNob3J0ZW4iOnsic291cmNlIjoyMCwidGFyZ2V0IjoyMH19XV0=
    \begin{tikzcd}
        {\id_{A}^{*}f^{*}\beta} &&&&&& {f^{*}\id_{B}^{*}\beta} \\
        A &&&& \beta && A &&&& \beta \\
        && {f^{*}\beta} && B &&&& {\id_{B}^{*}\beta} && B \\
        && A &&&&&& B
        \arrow[from=2-5, to=3-5]
        \arrow[from=3-3, to=4-3]
        \arrow["f"{description}, from=4-3, to=3-5]
        \arrow["f"{description, pos=0.6}, curve={height=-12pt}, from=2-1, to=3-5]
        \arrow["{\id_{A}}"{description}, from=2-1, to=4-3]
        \arrow[""{name=0, anchor=center, inner sep=0}, "{\id_{A}^{*}\cong\id_{\Fsf_{A}}}"{description}, from=3-3, to=1-1]
        \arrow[from=1-1, to=2-1]
        \arrow[""{name=1, anchor=center, inner sep=0}, "{\id_{A}^{*}f^{*}}"{description}, curve={height=12pt}, from=2-5, to=1-1]
        \arrow["{f^{*}}"{description}, from=2-5, to=3-3]
        \arrow["f"{description}, from=2-7, to=4-9]
        \arrow[""{name=2, anchor=center, inner sep=0}, "{f^{*}}"{description, pos=0.3}, from=3-9, to=1-7]
        \arrow[from=1-7, to=2-7]
        \arrow[from=3-9, to=4-9]
        \arrow["{\id_{B}}"{description}, from=4-9, to=3-11]
        \arrow["{\id_{B}^{*}\cong\id_{\Fsf_{B}}}"{description, pos=0.4}, from=2-11, to=3-9]
        \arrow[""{name=3, anchor=center, inner sep=0}, "{f^{*}\id_{B}^{*}}"{description}, curve={height=12pt}, from=2-11, to=1-7]
        \arrow[from=2-11, to=3-11]
        \arrow["f"{description}, curve={height=-12pt}, from=2-7, to=3-11]
        \arrow[shorten <=8pt, shorten >=8pt, Rightarrow, from=3, to=2]
        \arrow[shorten <=8pt, shorten >=8pt, Rightarrow, from=1, to=0]
    \end{tikzcd}$$
    \\\\
    To verify (d)(ii) with the setup $f:A\to B, g:B\to C, h:C\to D$ we have $f^{*}g^{*}h^{*}\delta$ and $(hgf)^{*}\delta$ both pullbacks of some $\delta\in\Obj\left(\Fsf_{D}\right)$ but there is a unique morphism in $\Fsf_{A}$ lying over $\id_{A}$ such that the diagram 
    $$% https://q.uiver.app/#q=WzAsNCxbMCwwLCJmXnsqfWdeeyp9aF57Kn1cXGRlbHRhIl0sWzAsMSwiQSJdLFsyLDEsIkEiXSxbMiwwLCIoaGdmKV57Kn1cXGRlbHRhIl0sWzEsMiwiXFxpZF97QX0iLDJdLFswLDNdLFswLDEsInAiLDJdLFszLDIsInAiXV0=
    \begin{tikzcd}
        {f^{*}g^{*}h^{*}\delta} && {(hgf)^{*}\delta} \\
        A && A
        \arrow["{\id_{A}}"', from=2-1, to=2-3]
        \arrow[from=1-1, to=1-3]
        \arrow["p"', from=1-1, to=2-1]
        \arrow["p", from=1-3, to=2-3]
    \end{tikzcd}$$
    commutes by the definition of Cartesian arrows and the natural isomorphisms in the following diagram
    $$% https://q.uiver.app/#q=WzAsNSxbMCwwLCJmXnsqfWdeeyp9aF57Kn1cXGRlbHRhIl0sWzAsMiwiZl57Kn0oaGcpXnsqfVxcZGVsdGEiXSxbMywwLCIoZ2YpXnsqfWheeyp9XFxkZWx0YSJdLFszLDIsIihoZ2YpXnsqfVxcZGVsdGEiXSxbMSwyXSxbMCwyLCJcXGFscGhhX3tmLGd9KGheeyp9XFxkZWx0YSkiXSxbMiwzLCJcXGFscGhhX3tnZixofShcXGRlbHRhKSJdLFsxLDMsIlxcYWxwaGFfe2YsaGd9KFxcZGVsdGEpIiwyXSxbMCwxLCJmXnsqfVxcYWxwaGFfe2csaH0oXFxkZWx0YSkiLDJdLFswLDNdLFsyLDksIiIsMSx7InNob3J0ZW4iOnsidGFyZ2V0IjoyMH19XSxbMSw5LCIiLDEseyJzaG9ydGVuIjp7InRhcmdldCI6MjB9fV1d
    \begin{tikzcd}
        {f^{*}g^{*}h^{*}\delta} &&& {(gf)^{*}h^{*}\delta} \\
        \\
        {f^{*}(hg)^{*}\delta} & {} && {(hgf)^{*}\delta}
        \arrow["{\alpha_{f,g}(h^{*}\delta)}", from=1-1, to=1-4]
        \arrow["{\alpha_{gf,h}(\delta)}", from=1-4, to=3-4]
        \arrow["{\alpha_{f,hg}(\delta)}"', from=3-1, to=3-4]
        \arrow["{f^{*}\alpha_{g,h}(\delta)}"', from=1-1, to=3-1]
        \arrow[""{name=0, anchor=center, inner sep=0}, from=1-1, to=3-4]
        \arrow[shorten >=8pt, Rightarrow, from=1-4, to=0]
        \arrow[shorten >=8pt, Rightarrow, from=3-1, to=0]
    \end{tikzcd}$$
    as desired. 
\end{proof}
Under special circumstances, a lax 2-functor from a category $\Ssf$ to the category of categories can be made into a functor. 
\begin{definition}[Splitting Cleavage]\label{def: split cleavage}
    Let $p:\Fsf\to\Ssf$ be a fibered category with cleavage $K$. The cleavage $K$ is splitting if it contains all identities and is closed under composition. 
\end{definition}
One then shows the following. 
\begin{lemma}\label{lem: lax2 is functor iff split cleavage}
    The lax 2-functor associated to the cleavage is a functor if and only if the cleavage is splitting. 
\end{lemma}
\begin{proof}
    If the cleavage is splitting then the composition law is rigid with unitality and associativity by \Cref{def: lax 2 functor}.
\end{proof}
\begin{remark}
    Generally, fibered categories do not admit a splitting. 
\end{remark}
The converse to \Cref{lem: fibered cat with cleavage is lax2} also holds. That is, given a fibered category with a lax 2-functor, one can construct a cleavage.
\begin{lemma}\label{lem: lax2 fibered cat has cleavage}
   Suppose $p:\Fsf\to\Ssf$ is a fibered category and $\Phi$ some lax 2-functor on $\Ssf$ given on objects by 
   $$A\mapsto\Fsf_{A}$$
   and on morphisms by 
   $$(f:A\to B)\mapsto (f^{*}:\Fsf_{B}\to\Fsf_{A}).$$
   There exists a cleavage $K$ realizing $\Phi$.  
\end{lemma} 
Evidently this results in the following theorem. 
\begin{theorem}\label{thm: lax2 iff cleavage}
    Let $p:\Fsf\to\Csf$ be a fibered category. The following are equivalent. 
    \begin{enumerate}[label=(\alph*)]
        \item There is a lax 2-functor $\Phi:\Ssf\to\Cat$.
        \item The category $\Ssf$ has a cleavage $K$.
    \end{enumerate}
\end{theorem}
\begin{proof}
    Immediate from \Cref{lem: lax2 fibered cat has cleavage} and \Cref{lem: fibered cat with cleavage is lax2}.
\end{proof}
We list an additional property of fibered categories.
\begin{definition}[Stable Property of Morphisms]\label{def: stable property of morphisms}
    A collection of morphisms $\Psf\subseteq\Mor_{\Csf}$ of a category $\Csf$ is stable if the following two conditions hold:
    \begin{enumerate}[label=(\alph*)]
        \item If $f:A\to B$ is in $\Psf$ and $\phi:A'\to A,\psi:B'\to B$ are isomorphisms then the composition $(\psi\circ f\circ\phi):A'\to B'$ is in $\Psf$ as well. 
        \item Given $A\to B$ in $\Psf$ and any map $C\to B$ the fibered product 
        $$% https://q.uiver.app/#q=WzAsNCxbMCwwLCJBXFx0aW1lc197Qn1DIl0sWzAsMSwiQSJdLFsyLDEsIkIiXSxbMiwwLCJDIl0sWzMsMl0sWzEsMl0sWzAsM10sWzAsMV1d
        \begin{tikzcd}
            {A\times_{B}C} && C \\
            A && B
            \arrow[from=1-3, to=2-3]
            \arrow[from=2-1, to=2-3]
            \arrow[from=1-1, to=1-3]
            \arrow[from=1-1, to=2-1]
        \end{tikzcd}$$
        exists and $A\times_{B}C\to C$ is in $\Psf$. 
    \end{enumerate}
\end{definition}
\subsection{Categories Fibered in Groupoids}\label{subsec: categories fibered in groupoids}
We now turn to a very important example of fibered categories. Categories fibered in groupoids. Recall here the following definition from category theory. 
\begin{definition}[Groupoid]\label{def: groupoid}
    Let $\Csf$ be a category. $\Csf$ is a groupoid if and only if every morphism in $\Csf$ is an isomorphism. 
\end{definition}
Naturally, we define a category fibered in groupoids as follows. 
\begin{definition}[Category Fibered in Groupoids]\label{def: category fibered in groupoids}
    Let $p:\Fsf\to\Ssf$ be a fibered category. $p:\Fsf\to\Ssf$ is a category fibered in groupoids if for all $A\in\Obj(\Ssf)$ the category $\Fsf_{A}$ is a groupoid. 
\end{definition}
One can alternatively characterize categories fibered in groupoids as follows. 
\begin{lemma}\label{lem: cfg iff cartesian plus}
    Let $p:\Fsf\to\Ssf$ be a fibered category. $p:\Fsf\to\Ssf$ is fibered in groupoids if and only if the following two properties hold:
    \begin{enumerate}[label=(\alph*)]
        \item Every arrow in $\Fsf$ is Cartesian. 
        \item Given $\beta\in\Obj(\Fsf)$ and $f:A\to p(\beta)$ in $\Ssf$, there exists $\phi:\alpha\to\beta$ in $\Fsf$ such that $p(\phi)=f$. 
    \end{enumerate}
\end{lemma}
\begin{proof}
    $(\Longrightarrow)$ Suppose $p:\Fsf\to\Ssf$ is a category fibered in groupoids. Lifts exist and are unique by $\id_{\Fsf_{B}}$ Cartesian satisfying (ii). For (i), let $\phi:\alpha\to\beta$ lying over $f:A\to B$ and $\phi':\alpha'\to\beta$ a pullback of $\beta\in\Obj(\Fsf_{B})$ to $\Fsf_{A}$, there is a map $\eta:\alpha\to\alpha'$ in $\Fsf_{A}$ which is an isomorphism since $p:\Fsf\to\Ssf$ is a category fibered in groupoids. But isomorphisms are unique so $\phi$ is Cartesian. 
    \\\\
    $(\Longleftarrow)$ Suppose conditions (a) and (b) hold and let $\phi:\alpha'\to\alpha$ be a morphism $\Fsf_{A}$ over $A\in\Obj(\Ssf)$. Since this morphism $\phi$ is Cartesian there is $\psi:\alpha\to\alpha'$ filling in the following diagram. 
    $$% https://q.uiver.app/#q=WzAsNixbMiwxLCJcXGFscGhhJyJdLFsyLDIsIkEiXSxbNCwxLCJcXGFscGhhIl0sWzQsMiwiQSJdLFswLDAsIlxcYWxwaGEiXSxbMCwxLCJBIl0sWzUsMSwiXFxpZF97QX0iLDFdLFs0LDAsIlxccHNpIiwxLHsic3R5bGUiOnsiYm9keSI6eyJuYW1lIjoiZGFzaGVkIn19fV0sWzAsMV0sWzQsMiwiXFxpZF97XFxhbHBoYX0iLDFdLFs1LDMsIlxcaWRfe0F9IiwxLHsibGFiZWxfcG9zaXRpb24iOjYwfV0sWzQsNV0sWzIsM10sWzAsMiwiXFxwaGkiLDFdLFsxLDMsIlxcaWRfe0F9IiwxXV0=
    \begin{tikzcd}
        \alpha \\
        A && {\alpha'} && \alpha \\
        && A && A
        \arrow["{\id_{A}}"{description}, from=2-1, to=3-3]
        \arrow["\psi"{description}, dashed, from=1-1, to=2-3]
        \arrow[from=2-3, to=3-3]
        \arrow["{\id_{\alpha}}"{description}, from=1-1, to=2-5]
        \arrow["{\id_{A}}"{description, pos=0.6}, from=2-1, to=3-5]
        \arrow[from=1-1, to=2-1]
        \arrow[from=2-5, to=3-5]
        \arrow["\phi"{description}, from=2-3, to=2-5]
        \arrow["{\id_{A}}"{description}, from=3-3, to=3-5]
    \end{tikzcd}$$
    By the commutativity of the upper triangle, $\phi\circ\psi\Longrightarrow\id_{\alpha}$ so $\psi$ is a right inverse of $\phi$ and by the diagram 
    $$% https://q.uiver.app/#q=WzAsNixbMiwxLCJcXGFscGhhIl0sWzIsMiwiQSJdLFs0LDEsIlxcYWxwaGEnIl0sWzQsMiwiQSJdLFswLDAsIlxcYWxwaGEnIl0sWzAsMSwiQSJdLFs1LDEsIlxcaWRfe0F9IiwxXSxbNCwwLCJcXHBoaSIsMSx7InN0eWxlIjp7ImJvZHkiOnsibmFtZSI6ImRhc2hlZCJ9fX1dLFswLDFdLFs0LDIsIlxcaWRfe1xcYWxwaGEnfSIsMV0sWzUsMywiXFxpZF97QX0iLDEseyJsYWJlbF9wb3NpdGlvbiI6NjB9XSxbNCw1XSxbMiwzXSxbMCwyLCJcXHBzaSIsMV0sWzEsMywiXFxpZF97QX0iLDFdXQ==
    \begin{tikzcd}
        {\alpha'} \\
        A && \alpha && {\alpha'} \\
        && A && A
        \arrow["{\id_{A}}"{description}, from=2-1, to=3-3]
        \arrow["\phi"{description}, dashed, from=1-1, to=2-3]
        \arrow[from=2-3, to=3-3]
        \arrow["{\id_{\alpha'}}"{description}, from=1-1, to=2-5]
        \arrow["{\id_{A}}"{description, pos=0.6}, from=2-1, to=3-5]
        \arrow[from=1-1, to=2-1]
        \arrow[from=2-5, to=3-5]
        \arrow["\psi"{description}, from=2-3, to=2-5]
        \arrow["{\id_{A}}"{description}, from=3-3, to=3-5]
    \end{tikzcd}$$
    so too is $\phi$ the right inverse of $\psi$, showing $\phi$ is an isomorphism. 
\end{proof}
We now consider the higher categorical variant of Yoneda's lemma. Recall that we have seen how a category $\Csf$ can be embedded into the functor category $\Fun(\Csf^{\Opp},\Sets)$, equivalently described as the category of set-valued presheaves on $\Csf$. One can similarly define for a fibered category $p:\Fsf\to\Ssf$ an embedding $\Fun(\Ssf^{\Opp},\Sets)$ into the 2-category of fibered categories over $\Ssf$ denoted $\FibCat(\Ssf)$ by taking for $F\in\Obj\left(\Fun(\Ssf^{\Opp},\Sets)\right)$ defining the fiber to be the collection of objects
$$\Fsf_{A}=\left\{(A,\alpha)\in\Obj(\Ssf)\times F(A)\right\}$$
and morphisms $f^{*}:\Fsf_{B}\to\Fsf_{A}$ such that 
$$f^{*}\left((B,F(f(\alpha)))\right)=(A,\alpha).$$
By composing these embeddings, we have an embedding of a category $\Csf$ to the 2-category of categories fibered over $\Csf$, $\FibCat(\Csf)$. For a category $\Csf$, the map on objects takes $Z\in\Obj(\Csf)$ to the slice category $\Csf_{(-/Z)}$ whose objects are morphisms $X\to Z$ and whose morphisms are commutative triangles 
$$% https://q.uiver.app/#q=WzAsMyxbMCwwLCJYIl0sWzIsMCwiWSJdLFsxLDEsIloiXSxbMCwyXSxbMSwyXSxbMCwxLCJmIl1d
\begin{tikzcd}
	X && Y \\
	& Z
	\arrow[from=1-1, to=2-2]
	\arrow[from=1-3, to=2-2]
	\arrow["f", from=1-1, to=1-3]
\end{tikzcd}$$
which we denote $f$, taking the morphisms to $Z$, known as the structure morphisms, as implicit. One yields a map $\Csf_{(-/Z)}\to\Csf$ by $(X\to Z)\mapsto X$, compatible with morphisms in the obvious way. 
\begin{definition}[Representable Fibered Category]\label{def: representable fibered category}
   A fibered category over $\Csf$ is representable if it is equivalent as a category to a slice category $\Csf_{(-/Z)}$ for some $Z\in\Obj(\Csf)$. 
\end{definition}
We now state and prove the 2-categorical Yoneda lemma. 
\begin{lemma}[2-Categorical Yoneda]\label{lem: 2cat Yoneda}
    Let $p:\Fsf\to\Csf$ be a fibered category and $Z\in\Obj(\Csf)$. There is an equivalence of categories 
    $$\Fun\left(\Csf_{(-/Z)},\Fsf\right)\to\Fsf_{Z}.$$
\end{lemma}
\section{Descent}\label{sec: descent}
Descent generalizes the identity and gluing axioms for sheaves on topological spaces to the setting of sites. We have already seen that for $\Ssf$ a site, we should think of a fibered category over $\Ssf$ as a lax 2-functor in the sense of \Cref{def: lax 2 functor}, that is, as a presheaf of categories on $\Ssf$. A stack, which we will soon encounter, is a sheaf of categories over $\Ssf$. 
\subsection{Descent in Fibered Categories}\label{subsec: descent in fibered categories}
Let $p:\Fsf\to\Ssf$ be a category fibered over a site $\Ssf$ with a fixed cleavage $K$. For a covering $\{X_{i}\to X\}$, denote $X_{ij}=X_{i}\times_{X}X_{j}$ and $X_{ijk}=X_{i}\times_{X}X_{j}\times_{X}X_{k}$. We now define an object with descent data. 
\begin{definition}[Object With Descent Data]\label{def: object w descent data}
    Let $\{X_{i}\to X\}$ be a covering in $\Ssf$. An object with descent data on $\{X_{i}\to X\}$ is the tuple $(\{\xi_{i}\},\{\phi_{ij}\})$ with objects $\xi_{i}\in\Fsf_{X_{i}}$ and isomorphisms $\phi_{ij}:\pr_{2}^{*}\xi_{j}\to\pr_{1}^{*}\xi_{i}$ in $\Fsf_{X_{ij}}$ such that 
    $$\pr_{1,3}^{*}\phi_{ik}=\pr_{1,2}^{*}\phi_{ij}\circ\pr_{2,3}^{*}\phi_{jk}:\pr_{3}^{*}\xi_{k}\to\pr_{1}^{*}\xi_{i}.$$
\end{definition}
\begin{remark}
    The maps $\phi_{ij}$ are known as transition isomorphisms of the object $X$ with descent data. 
\end{remark}
One naturally defines morphisms of objects with descent data as follows. 
\begin{definition}[Morphisms of Objects with Descent Data]
    A morphism between objects with descent data on $\{X_{i}\to X\}$, $(\{\xi_{i}\},\{\phi_{ij}\})$ and $(\{\upsilon_{i}\},\{\psi_{i,j}\})$, is a tuple $\{\alpha_{i}\}$ with $\alpha_{i}:\Fsf_{X_{i}}\to\Fsf_{X_{i}}$ where $\xi_{i}\mapsto\upsilon_{i}$ such that for each pair $i,j$ the diagram 
    $$% https://q.uiver.app/#q=WzAsNCxbMCwwLCJcXHByX3syfV57Kn1cXHhpX3tqfSJdLFswLDEsIlxccHJfezF9XnsqfVxceGlfe2l9Il0sWzIsMCwiXFxwcl97Mn1eeyp9XFx1cHNpbG9uX3tqfSJdLFsyLDEsIlxccHJfezF9XnsqfVxcdXBzaWxvbl97aX0iXSxbMSwzLCJcXHByX3sxfV57Kn1cXGFscGhhX3tpfSIsMl0sWzIsMywiXFxwc2lfe2lqfSJdLFswLDEsIlxccGhpX3tpan0iLDJdLFswLDIsIlxccHJfezJ9XnsqfVxcYWxwaGFfe2p9Il1d
    \begin{tikzcd}
        {\pr_{2}^{*}\xi_{j}} && {\pr_{2}^{*}\upsilon_{j}} \\
        {\pr_{1}^{*}\xi_{i}} && {\pr_{1}^{*}\upsilon_{i}}
        \arrow["{\pr_{1}^{*}\alpha_{i}}"', from=2-1, to=2-3]
        \arrow["{\psi_{ij}}", from=1-3, to=2-3]
        \arrow["{\phi_{ij}}"', from=1-1, to=2-1]
        \arrow["{\pr_{2}^{*}\alpha_{j}}", from=1-1, to=1-3]
    \end{tikzcd}$$
    commutes. 
\end{definition}
The morphisms $\alpha_{i}$ compose in the obvious way making objects with descent data the objects of a category $\Fsf_{\{X_{i}\to X\}}$. 
\\\\
More explicitly, let $\xi\in\Fsf_{X}$ and $\{\sigma_{i}:X_{i}\to X\}$ a covering. We can construct an object with descent data as follows. Set $\xi_{i}=\sigma_{i}^{*}\xi$ and transition isomorphisms the identity after identifying $\pr_{2}^{*}\sigma_{j}^{*}\xi$ and $\pr_{1}^{*}\sigma_{i}^{*}\xi$, the pullbacks of $\xi$ to $X_{ij}$. For some $\alpha:\xi\to\upsilon$ in $\Fsf_{X}$ we get $\alpha_{i}=\sigma_{i}^{*}\alpha:\sigma_{i}^{*}\xi\to\sigma_{i}^{*}\upsilon$, yielding a morphism of objects with descent data $\{\alpha_{i}\}$ from $(\{\xi_{i}\},\{\phi_{ij}\})$ to $(\{\upsilon_{i}\},\{\psi_{ij}\})$. Evidently this is a functor $\Fsf_{X}\to\Fsf_{\{X_{i}\to X\}}$ where on objects is given by 
$$\xi\mapsto(\{\xi_{i}\},\{\phi_{ij}\})$$
and on morphisms by 
$$(\alpha:\xi\to\upsilon)\mapsto \{\alpha_{i}:\xi_{i}\to\upsilon_{i}\}$$
which are compatible with the transition isomorphisms $\phi_{ij},\psi_{ij}$. 
\begin{remark}
    This construction does not depend on the cleavage in the sense that the categories $\Fsf_{\{X_{i}\to X\}}$ are equivalent regardless of the choice of cleavage.
\end{remark}
We can now define some long-awaited notions. 
\begin{definition}[Prestack]\label{def: categorical prestack}
    Let $p:\Fsf\to\Ssf$ be a category fibered over a site. $\Fsf$ is a prestack over $\Ssf$ if for each covering $\{X_{i}\to X\}$ in $\Ssf$ the functor $\Fsf_{X}\to\Fsf_{\{X_{i}\to X\}}$ is fully faithful. 
\end{definition}
\begin{definition}[Stack]\label{def: categorical stack}
    Let $p:\Fsf\to\Ssf$ be a category fibered over a site. $\Fsf$ is a stack over $\Ssf$ if for each covering $\{X_{i}\to X\}$ in $\Ssf$ the functor $\Fsf_{X}\to\Fsf_{\{X_{i}\to X\}}$ is an equivalence of categories. 
\end{definition}
Unpacking the definition, to be a prestack means that for $X\in\Obj(\Ssf)$; $\xi,\upsilon\in\Obj(\Fsf_{X})$; $\{X_{i}\to X\}$ a covering; $\xi_{i},\upsilon_{i}$ pullbacks of $\xi,\upsilon$ to the $X_{i}$; $\xi_{ij},\upsilon_{ij}$ pullbacks to $X_{ij}$ and there was some $a_{i}:\xi_{i}\to\upsilon_{i}$ in $\Fsf_{X_{i}}$ such that $\pr_{1}^{*}\alpha_{i}=\pr_{2}^{*}\alpha_{j}:\xi_{ij}\to\upsilon_{ij}$ then there is a unique $\alpha:\xi\to\upsilon$ in $\Fsf_{X}$ whose pullback is $\alpha_{i}:\xi_{i}\to\upsilon_{i}$ for all $i$. Clarifying what happens if $p:\Fsf\to\Ssf$ is a stack will require introducing the following notion. 
\begin{definition}[Effective Descent Data]\label{def: effective descent data}
    An object with descent data $(\{\xi_{i}\},\{\phi_{ij}\})$ in $\Fsf_{\{X_{i}\to X\}}$ is effective if it is isomorphic to the image of an object in $\Fsf_{X}$. 
\end{definition} 
In other words an object with descent data $(\{\xi_{i}\},\{\phi_{ij}\})$ in $\Fsf_{\{X_{i}\to X\}}$ is effective if there exists $\xi\in\Obj(\Fsf_{X})$ and Cartesian arrows $\xi_{i}\to\xi$ over $\sigma_{i}:X_{i}\to X$ such that the pentagonal diagram 
$$% https://q.uiver.app/#q=WzAsNixbMiwwLCJcXHhpIl0sWzEsMV0sWzAsMSwiXFx4aV97aX0iXSxbNCwxLCJcXHhpX3tqfSJdLFsxLDMsIlxccHJfezF9XnsqfVxceGlfe2l9Il0sWzMsMywiXFxwcl97Mn1eeyp9XFx4aV97an0iXSxbMiwwXSxbMywwXSxbNCwyXSxbNSwzXSxbNSw0LCJcXHBoaV97aWp9Il1d
\begin{tikzcd}
	&& \xi \\
	{\xi_{i}} & {} &&& {\xi_{j}} \\
	\\
	& {\pr_{1}^{*}\xi_{i}} && {\pr_{2}^{*}\xi_{j}}
	\arrow[from=2-1, to=1-3]
	\arrow[from=2-5, to=1-3]
	\arrow[from=4-2, to=2-1]
	\arrow[from=4-4, to=2-5]
	\arrow["{\phi_{ij}}", from=4-4, to=4-2]
\end{tikzcd}$$
commutes for all $i,j$.  
\\\\
If $p:\Fsf\to\Ssf$ is a stack as in \Cref{def: categorical stack}, we know the functor $\Fsf_{X}\to\Fsf_{\{X_{i}\to X\}}$ is fully faithful and essentially surjective. So all $(\{\xi_{i}\},\{\phi_{ij}\})$ are isomorphic to the image of some object in $\Fsf_{X}$, that is, all descent data is effective. 
\subsection{Descent for Torsors}\label{subsec: descent for torsors}
Torsors are closely related to principal $G$-bundles in algebraic topology, and are central objects in the study of both algebraic and arithmetic geometry. 
\begin{definition}[Torsor]\label{def: torsor}
    Let $\Ssf$ be a site and $\Bmu$ a sheaf of groups on $\Ssf$. A $\Bmu$-torsor on $\Ssf$ is a pair $(\Pcal,\rho)$ where $\Pcal$ is a sheaf on $\Ssf$ with a left action 
    $$\rho:\Bmu\times\Pcal\longrightarrow\Pcal$$
    such that the following conditions hold:
    \begin{enumerate}
        \item For all $X\in\Obj(\Ssf)$ there is a covering $\{X_{i}\to X\}_{i\in I}$ such that $\Pcal(X_{i})\neq\emptyset$ for all $i\in I$.
        \item The morphism of sheaves
        $$\Bmu\times\Pcal\to\Pcal\times\Pcal; (g,p)\mapsto(p,g\cdot p)$$
        is an isomorphism. 
    \end{enumerate}
\end{definition}
\begin{remark}
    The sheaf of groups $\Bmu$ need not be a sheaf of Abelian groups. 
\end{remark}
\begin{remark}
    The condition (b) in \Cref{def: torsor} says that for $\Pcal(X)$ the action of the group $\Bmu(X)$ on $\Pcal(X)$ is simply transitive: that it is both free -- the identity is the only element fixing any point -- and for any two sections $p,p'\in\Pcal(X)$ there is a unique $g\in \mu(X)$ such that $g\cdot p=p'$. 
\end{remark}
\begin{definition}[Trivial Torsor]\label{def: trivial torsor}
    Let $\Ssf$ be a site, $\Bmu$ a sheaf of groups on $\Ssf$, and $(\Pcal,\rho)$ a $\Bmu$-torsor on $\Ssf$ if $\Pcal$ has a global section. 
\end{definition}
If $(\Pcal,\rho)$ is a trivial $\Bmu$-torsor on a site $\Ssf$, then for a global section $p\in\Pcal(X)$ we have an isomorphism 
$$\Bmu\to\Pcal; g\mapsto g\cdot p$$
identifying $\Pcal$ with $\Bmu$ and the action $\rho$ with the endomorphism by left multiplication. Morphisms of torsors are defined in the natural way as follows. 
\begin{definition}[Morphism of Torsors]\label{def: morphism of torosrs}
    Let $\Ssf$ be a site, $\Bmu$ a sheaf of groups on $\Ssf$, and $(\Pcal,\rho),(\Pcal,\rho')$ two $\Bmu$-torsors on $\Ssf$. A morphism of torsors $f:(\Pcal',\rho')\to(\Pcal,\rho)$ is the data of a morphism of sheaves $f:\Pcal'\to\Pcal$ such that the diagram 
    $$% https://q.uiver.app/#q=WzAsNCxbMCwwLCJcXEJtdVxcdGltZXNcXFBjYWwnIl0sWzIsMCwiXFxCbXVcXHRpbWVzXFxQY2FsIl0sWzAsMSwiXFxQY2FsJyJdLFsyLDEsIlxcUGNhbCJdLFsyLDMsImYiLDJdLFswLDEsIlxcaWRfe1xcQm11fVxcdGltZXMgZiJdLFswLDIsIlxccmhvJyIsMl0sWzEsMywiXFxyaG8iXV0=
    \begin{tikzcd}
        {\Bmu\times\Pcal'} && \Bmu\times\Pcal \\
        {\Pcal'} && \Pcal
        \arrow["f"', from=2-1, to=2-3]
        \arrow["{\id_{\Bmu}\times f}", from=1-1, to=1-3]
        \arrow["{\rho'}"', from=1-1, to=2-1]
        \arrow["\rho", from=1-3, to=2-3]
    \end{tikzcd}$$
    commutes. 
\end{definition}
Torsors are intimately connected to principal $G$-bundles in algebraic geometry and algebraic topology. 
\begin{definition}[Principal $G$-Bundle]\label{def: principal G bundle}
    Let $(\Sch_{X})_{\fppf}$ denote the fppf site of $X$-schemes for $X$ a base scheme and $\Bmu$ a sheaf of groups on $(\Sch_{X})_{\fppf}$ representable by a flat locally finitely presented $X$-group scheme $G$. A principal $G$-bundle on $X$ is a pair $(\pi:P\to X,\rho)$ where $\pi:P\to X$ is a flat, locally finitely presented, surjective morphism of schemes and 
    $$\rho:G\times_{X}P\to P$$
    a morphism such that the following conditions hold:
    \begin{enumerate}[label=(\alph*)]
        \item The diagram 
        $$% https://q.uiver.app/#q=WzAsNCxbMCwwLCJHXFx0aW1lc197WH1HXFx0aW1lc197WH1QIl0sWzIsMCwiR1xcdGltZXNfe1h9UCJdLFswLDEsIkdcXHRpbWVzX3tYfVAiXSxbMiwxLCJQIl0sWzIsMywiXFxyaG8iLDJdLFsxLDMsIlxccmhvIl0sWzAsMSwiXFxpZF97R31cXHRpbWVzXFxyaG8iXSxbMCwyLCJtXFx0aW1lc1xcaWRfe1B9IiwyXV0=
        \begin{tikzcd}
            {G\times_{X}G\times_{X}P} && {G\times_{X}P} \\
            {G\times_{X}P} && P
            \arrow["\rho"', from=2-1, to=2-3]
            \arrow["\rho", from=1-3, to=2-3]
            \arrow["{\id_{G}\times\rho}", from=1-1, to=1-3]
            \arrow["{m\times\id_{P}}"', from=1-1, to=2-1]
        \end{tikzcd}$$
        commutes, here denoting $m:G\times_{X}G\to G$ the group operation on the group scheme $G$. 
        \item If $e:X\to G$ is the identity action, the composition 
        $$% https://q.uiver.app/#q=WzAsMyxbMCwwLCJQIl0sWzIsMCwiR1xcdGltZXNfe1h9UCJdLFs0LDAsIlAiXSxbMCwxLCJlXFxjaXJjXFxwaVxcdGltZXNcXGlkX3tQfSJdLFsxLDIsIlxccmhvIl1d
        \begin{tikzcd}
            P && {G\times_{X}P} && P
            \arrow["{e\circ\pi\times\id_{P}}", from=1-1, to=1-3]
            \arrow["\rho", from=1-3, to=1-5]
        \end{tikzcd}$$
        is the identity on $P$. 
        \item The map $(\rho,\pr_{2}):G\times_{X}P\to P\times_{X}P$ is an isomorphism 
    \end{enumerate}
\end{definition}
The morphisms of principal $G$-bundles are given in the obvious way. 
\begin{definition}[Morphisms of Principal $G$-Bundles]\label{def: morphism of principal G bundles}
    Let $(\Sch_{X})_{\fppf}$ denote the fppf site of $X$-schemes for $X$ a base scheme and $\Bmu$ a sheaf of groups on $(\Sch_{X})_{\fppf}$ representable by a flat locally finitely presented $X$-group scheme $G$. A morphism of principal $G$-bundles 
    $$f:(\pi':P'\to X,\rho')\to(\pi:P\to X,\rho)$$
    is a morphism of $X$-schemes $f:P'\to P$ such that the diagram 
    $$% https://q.uiver.app/#q=WzAsNCxbMCwwLCJHXFx0aW1lc197WH1QJyJdLFswLDEsIlAnIl0sWzIsMCwiR1xcdGltZXNfe1h9UCJdLFsyLDEsIlAiXSxbMSwzLCJmIiwyXSxbMCwyLCJcXGlkX3tHfVxcdGltZXMgZiJdLFswLDEsIlxccmhvJyIsMl0sWzIsMywiXFxyaG8iXV0=
    \begin{tikzcd}
        {G\times_{X}P'} && {G\times_{X}P} \\
        {P'} && P
        \arrow["f"', from=2-1, to=2-3]
        \arrow["{\id_{G}\times f}", from=1-1, to=1-3]
        \arrow["{\rho'}"', from=1-1, to=2-1]
        \arrow["\rho", from=1-3, to=2-3]
    \end{tikzcd}$$
    commutes. 
\end{definition}
Given a principal $G$-bundle $(\pi:P\to X,\rho)$ on $X$, we get a $\mu$-torsor bytaking $\Pcal$ to be the sheaf on $(\Sch_{X})_{\fppf}$ represented by the scheme $P$ with action that induced by $\rho$. The conditions (a) and (b) of \Cref{def: principal G bundle} imply that the map $\Bmu\times\Pcal\to\Pcal$ is an action while condition (c) imposes that the action is simply transitive. Furthermore, since $\pi:P\to X$ is flat, locally finitely presented, and surjective as a map of schemes, there exists an fppf cover $\{X_{i}\to X\}$ such that $\Pcal(X_{i})\neq\emptyset$ for all $i\in I$. 
\\\\
Furthermore, one can observe that these constructions are functorial, yielding a categories of principal $G$-bundles on a scheme $X$ and $\Bmu$-torsors on the fppf site of $X$-schemes $(\Sch_{X})_{\fppf}$. Indeed, Yoneda's lemma tells us that this functor is fully faithful that is essentially surjective on the condition that the structure morphism of the algebraic group $G$ is affine. 
\begin{proposition}\label{prop: if affine then G bundles are mu torsors}
    Let $(\Sch_{X})_{\fppf}$ denote the fppf site of $X$-schemes for $X$ a base scheme and $\Bmu$ a sheaf of groups on $(\Sch_{X})_{\fppf}$ representable by a flat locally finitely presented $X$-group scheme $G$. If the structure morphism $G\to X$ is an affine morphism then there is a equivalence between the category of principal $G$-bundles on $X$ and the category $\Bmu$-torsors on the fppf site of $X$-schemes $(\Sch_{X})_{\fppf}$.
\end{proposition}
In some simple cases, we can describe the category of torsors in a relatively concrete manner. 
\\\\
Consider the scheme $X$ and $n$ an integer invertible on the scheme $X$. Denote $\Bmu_{n}$ the group scheme such that 
$$\Bmu_{n}(S)=\left\{f\in\Ocal_{X}^{\times}:f^{n}=1\right\}.$$
The category of $\Bmu_{n}$-torsors $\mathsf{Tors}(\Bmu_{n})$ on the small \'{e}tale site of $X$ $X_{\et}$ can be described as follows: consider $\Sigma_{n}$ the category with objects pairs $(L,\sigma)$ where $L$ is an invertible sheaf on the scheme $X$ and $\sigma:L^{\otimes n}\to\Ocal_{X}$ a trivialization of the $n$-th power of $L$, considering $L$ as a sheaf in the \'{e}tale topology. The morphisms in $\Sigma_{n}$ between two objects $(L',\sigma')$ and $(L,\sigma)$ are morphisms of line bundles $\rho:L'\to L$ such that the diagram 
$$% https://q.uiver.app/#q=WzAsMyxbMCwwLCJMJ157XFxvdGltZXMgbn0iXSxbMiwwLCJMXntcXG90aW1lcyBufSJdLFsxLDEsIlxcT2NhbF97WH0iXSxbMCwyLCJcXHNpZ21hJyIsMl0sWzEsMiwiXFxzaWdtYSJdLFswLDEsIlxccmhvXntcXG90aW1lcyBufSJdXQ==
\begin{tikzcd}
	{L'^{\otimes n}} && {L^{\otimes n}} \\
	& {\Ocal_{X}}
	\arrow["{\sigma'}"', from=1-1, to=2-2]
	\arrow["\sigma", from=1-3, to=2-2]
	\arrow["{\rho^{\otimes n}}", from=1-1, to=1-3]
\end{tikzcd}$$
commutes. One can construct a functor 
$$F:\Sigma_{n}\longrightarrow\mathsf{Tors}(\Bmu_{n})$$
by associating to each $(L,\sigma)\in\Sigma_{n}$ a sheaf $\Pcal_{(L,\sigma)}$ a sheaf on the \'{e}tale site $X_{\et}$ such that for any $U\to X$ \'{e}tale $\Pcal_{(L,\sigma)}(U)$ is the set of trivializations $\lambda:\Ocal_{U}\to L|_{U}$ such that the composite 
$$% https://q.uiver.app/#q=WzAsMyxbMCwwLCJcXE9jYWxfe1V9Il0sWzIsMCwiTF57XFxvdGltZXMgbn18X3tVfSJdLFs0LDAsIlxcT2NhbF97VX0iXSxbMSwyLCJcXHNpZ21hfF97VX0iXSxbMCwxLCJcXGxhbWJkYV57XFxvdGltZXMgbn0iXV0=
\begin{tikzcd}
	{\Ocal_{U}} && {L^{\otimes n}|_{U}} && {\Ocal_{U}}
	\arrow["{\sigma|_{U}}", from=1-3, to=1-5]
	\arrow["{\lambda^{\otimes n}}", from=1-1, to=1-3]
\end{tikzcd}$$
is the identity on the sheaf $\Ocal_{U}$. There is an action of $\Bmu_{n}(U)$ on $\Pcal_{(L,\sigma)}(U)$ for which $\zeta\in\Bmu_{n}(U)$ acts by $\lambda\mapsto \zeta\cdot\lambda$ which is simply transitive, endowing $\Pcal$ with the structure of a $\Bmu_{n}$-torsor restricting on fibers in the previously described way. 
\begin{remark}
    It is necessary to work in the \'{e}tale topology here, since it is not always possible to find a trivialization of the line bundle Zariski-locally. 
\end{remark}
\section{Descent for Quasicoherent Sheaves}\label{sec: descent for quasicoherent sheaves}
\part*{Algebraic Spaces}\label{part: algebraic spaces}
\section{Algebraic Spaces}\label{sec: algebraic spaces}
We begin with a discussion of various algebro-geometric and categorical preliminaries required to define algebraic spaces. 
\subsection{The Category of Schemes}
First recall that we have defined stable properties of morphisms in \Cref{def: stable property of morphisms}. We now define a stable property of objects in a category. 
\begin{definition}[Stable Class of Objects]\label{def: stable class of objects}
    Let $\Ssf$ be a site such that all representable presheaves are sheaves. A subclass of objects $\Obj(\Csf)\subseteq\Obj(\Ssf)$ is stable with respect to a property $\Psf$ if for all coverings $\{X_{i}\to X\}$, $X$ has property $\Psf$ if and only if $X_{i}$ have property $\Psf$ for all $i$. 
\end{definition}
\begin{definition}[Stable Property of Objects]\label{def: stable property of objects}
    A property $\Psf$ is stable if the class of objects containing $\Psf$ is stable.
\end{definition}
Turning to some categorical notions, we define the following. 
\begin{definition}[Closed Subcategory]\label{def: closed subcategory}
    Let $\Csf$ be a category. A closed subcategory $\Dsf$ of $\Csf$ is a subcategory such that the following conditions hold:
    \begin{enumerate}[label=(\alph*)]
        \item $\Dsf$ contains all isomorphisms in $\Csf$.
        \item For all Cartesian diagrams in $\Csf$
        $$% https://q.uiver.app/#q=WzAsNCxbMCwwLCJYJyJdLFswLDEsIlknIl0sWzIsMCwiWCJdLFsyLDEsIlkiXSxbMiwzLCJmIl0sWzAsMSwiZiciLDJdLFsxLDNdLFswLDJdXQ==
        \begin{tikzcd}
            {X'} && X \\
            {Y'} && Y
            \arrow["f", from=1-3, to=2-3]
            \arrow["{f'}"', from=1-1, to=2-1]
            \arrow[from=2-1, to=2-3]
            \arrow[from=1-1, to=1-3]
        \end{tikzcd}$$
        if $f\in\Mor_{\Dsf}$ then $f'\in\Mor_{\Dsf}$.
    \end{enumerate}
\end{definition}
\begin{definition}[Stable Subcategory]\label{def: stable subcategory}
    Let $\Dsf$ be a closed subcategory of a category $\Csf$. $\Dsf$ is stable if for all $(f:X\to Y)\in\Mor_{\Csf}$ and all coverings $\{Y_{i}\to Y\}$, $f$ is a morphism in $\Dsf$ if and only if all morphisms $f_{i}:X\times_{Y}XY_{i}\to Y_{i}$ are in $\Dsf$. 
\end{definition}
\begin{definition}[Local On Domain]\label{def: local on domain}
    A stable closed subcategory $\Dsf$ of a category $\Csf$ is local on domain if for all $f:X\to Y$ and coverings $\{X_{i}\to X\}$, $f$ is a morphism in $\Dsf$ if and only if $X_{i}\to X\to Y$ is in $\Dsf$. 
\end{definition}
We alternatively define stable properties of morphisms \Cref{def: stable property of morphisms} as follows. 
\begin{definition}[Stable Property of Morphisms]\label{def: stable property of morphisms 2}
    Let $\Psf$ be some property of morphisms in $\Csf$ satisfied by isomorphisms and closed under compositions and $\Dsf_{\Psf}$ the subcategory of $\Csf$ with the same objects but morphisms those satisfying property $\Psf$. $\Psf$ is a stable property of morphisms if $\Dsf_{\Psf}$ is a stable subcategory. 
\end{definition}
Similarly, we define locality on domain as follows. 
\begin{definition}[Morphisms Local on Domain]\label{def: morphisms local on domain}
    Let $\Psf$ be some property of morphisms in $\Csf$ satisfied by isomorphisms and closed under compositions and $\Dsf_{\Psf}$ the subcategory of $\Csf$ with the same objects but morphisms those satisfying property $\Psf$. $\Psf$ is a stable property of morphisms if $\Dsf_{\Psf}$ is local on domain as a subcategory. 
\end{definition}
One can show the following fact about morphisms of schemes. \newpage
\begin{proposition}\label{prop: stable properties of morphisms of schemes}
    Let $S$ be a scheme and consider the big \'{e}tale site $(\Sch_{S})_{\et}$. 
    \begin{enumerate}[label=(\alph*)]
        \item The following properties of morphisms in $(\Sch_{S})_{\et}$ are stable: proper, separated, surjective, and quasi-compact. 
        \item The following properties of morphisms in $(\Sch_{S})_{\et}$ are stable and local on domain: locally of finite type, locally of finite presentation, flat, \'{e}tale, universally open, locally quasi-finite, and smooth. 
    \end{enumerate}
\end{proposition}
\subsection{Representability and Sheaves on the \'{E}tale Site}
We now introduce some language to describe morphisms of sheaves on the \'{e}tale site. 
\begin{definition}[Representable by Schemes]\label{def: morphism of etale sheaves representable by schemes}
    Let $S$ be a scheme and let $\Fsf,\Gsf$ be sheaves on the site $(\Sch_{S})_{\et}$ and $f:\Fsf\to\Gsf$ a morphism of sheaves. The morphism $f$ is representable by schemes if for every $S$-scheme $T$ and morphism $T\to \Gsf$ the fibered product functor $\Fsf\times_{\Gsf}T$ is representable by a scheme. 
\end{definition}
\begin{definition}[Properties of Morphisms of \'{E}tale Sheaves]
    Let $S$ be a scheme and let $\Fsf,\Gsf$ be sheaves on the site $(\Sch_{S})_{\et}$, $f:\Fsf\to\Gsf$ a morphism of sheaves, and $\Psf$ a stable property of morphisms of schemes. If $f$ is representable by schemes, then $f$ has property $\Psf$ if for every $S$-scheme $T$ the morphism $\pr_{2}:\Fsf\times_{\Gsf}T\to T$ has $\Psf$. 
\end{definition}
\begin{example}
    Let $\Fsf,\Gsf$ be representable sheaves, say by $X,Y$, respectively. Any morphism $f:\Fsf\to\Gsf$ is representable by sheaves, induced by the corresponding morphism of schemes $X\to Y$. For any $S$-scheme $T$ with a map $T\to Y$ we have that $\Fsf\times_{\Gsf}T$ is the scheme $T\times_{Y}X$. 
\end{example}
\begin{lemma}\label{lem: morphisms of schemes and etale sheaves}
    Let $S$ be a scheme and $f:X\to Y$ be a $S$-morphism of schemes and $\Psf$ a stable property of morphisms of schemes. The morphism $f$ has property $\Psf$ if and only if the morphism of representable sheaves $h_{f}:h_{X}\to Y$ has property $\Psf$. 
\end{lemma}
\begin{proof}
    Let $T$ be an $S$-scheme and $g:T\to Y$ with $h_{g}:h_{T}\to h_{Y}$ where we have a natural isomorphism $h_{T}\times_{h_{Y}}h_{X}\Longrightarrow h_{X\times_{Y}T}$.
    $$% https://q.uiver.app/#q=WzAsOCxbMiwxLCJZIl0sWzIsMCwiVCJdLFswLDEsIlgiXSxbMCwwLCJYXFx0aW1lc197WX1UIl0sWzQsMCwiaF97WH1cXHRpbWVzX3toX3tZfX1oX3tUfVxcUmlnaHRhcnJvdyBoX3tYXFx0aW1lc197WX1UfSJdLFs0LDEsImhfe1h9Il0sWzYsMCwiaF97VH0iXSxbNiwxLCJoX3tZfSJdLFsyLDBdLFszLDFdLFsxLDBdLFszLDJdLFs1LDddLFs0LDZdLFs0LDVdLFs2LDddXQ==
    \begin{tikzcd}
        {X\times_{Y}T} && T && {h_{T}\times_{h_{Y}}h_{X}\Rightarrow h_{X\times_{Y}T}} && {h_{T}} \\
        X && Y && {h_{X}} && {h_{Y}}
        \arrow[from=2-1, to=2-3]
        \arrow[from=1-1, to=1-3]
        \arrow[from=1-3, to=2-3]
        \arrow[from=1-1, to=2-1]
        \arrow[from=2-5, to=2-7]
        \arrow[from=1-5, to=1-7]
        \arrow[from=1-5, to=2-5]
        \arrow[from=1-7, to=2-7]
    \end{tikzcd}$$
    We have that $f$ is $\Psf$ if and only if $X\times_{Y}T\to T$ is $\Psf$ if and only if $h_{X\times_{Y}T}\to h_{T}$ is $\Psf$ if and only if $h_{f}$ is $\Psf$ upon repeated application of the stable property of morphisms. 
\end{proof}
We prove an additional lemma before defining algebraic spaces.
\begin{lemma}[Representable Diagonal Implies Representability]\label{lem: rep diagonal then rep}
    Let $S$ be a scheme and $\Fsf$ a sheaf on $(\Sch_{S})_{\et}$. If $\delta_{\Fsf}:\Fsf\to\Fsf\times\Fsf$ is representable by schemes and $T$ any $S$-scheme then the morphism of sheaves $f:T\to\Fsf$ is representable by schemes. 
\end{lemma} 
\begin{proof}
    This follows from general abstract nonsense. See \cite[\href{https://stacks.math.columbia.edu/tag/0022}{Lemma 0022}]{stacks-project} and \cite[\href{https://stacks.math.columbia.edu/tag/0024}{Lemma 0024}]{stacks-project}. 
\end{proof}
\newpage
\subsection{Algebraic Spaces}
We are now able to define algebraic spaces, an important, slightly more restricted notion of an algebraic stack. 
\begin{definition}[Algebraic Space]\label{def: algebraic space}
    Let $S$ be a scheme. An algebraic space over $S$ is a functor $X:\Sch_{S}^{\Opp}\to\Sets$ such that the following hold:
    \begin{enumerate}[label=(\alph*)]
        \item $X$ is a sheaf on the big \'{e}tale site $(\Sch_{S})_{\et}$. 
        \item $\delta_{\Fsf}:\Fsf\to\Fsf\times_{S}\Fsf$ is representable by schemes. 
        \item There is an $S$-scheme $U$ and a surjective \'{e}tale morphism $U\to X$. 
    \end{enumerate}
\end{definition}
\begin{remark}
    Let $S$ be a scheme. Objects of the category of $S$-schemes $\Sch_{S}$ are tautologically algebraic spaces. 
\end{remark}
Given a base scheme $S$, we can consider the category of algebraic spaces $\Spaces_{S}$ with objects sheaves of sets on the big \'{e}tale site $(\Sch_{S})_{\et}$ and morphisms base-preserving 2-functors, or morphisms of sheaves on the site.  

\subsection{Sheaf Quotients}
We begin with the definition of an \'{e}tale equivalence relation. 
\begin{definition}[\'{E}tale Equivalence Relation]\label{def: etale equiv relation}
    Let $S$ be a scheme. An \'{e}tale equivalence relation on an $S$-scheme $X$ is a monomorphism of schemes $R\hookrightarrow X\times_{S}X$ such that 
    \begin{enumerate}[label=(\alph*)]
        \item For all $S$-schemes $T$, the $T$-points $R(T)\subseteq X(T)\times X(T)$ is an equivalence relation on $X(T)$. 
        \item The maps $s,t:R\to X$ induced by the projections from $X\times_{S}X$ are \'{e}tale morphisms.
        $$% https://q.uiver.app/#q=WzAsNCxbMCwwLCJSIl0sWzEsMSwiWFxcdGltZXNfe1N9WCJdLFsxLDIsIlgiXSxbMywxLCJYIl0sWzEsM10sWzEsMl0sWzAsMiwidCIsMix7ImN1cnZlIjoyfV0sWzAsMywicyIsMCx7ImN1cnZlIjotMn1dLFswLDEsIiIsMSx7InN0eWxlIjp7InRhaWwiOnsibmFtZSI6Imhvb2siLCJzaWRlIjoidG9wIn19fV1d
        \begin{tikzcd}
            R \\
            & {X\times_{S}X} && X \\
            & X
            \arrow[from=2-2, to=2-4]
            \arrow[from=2-2, to=3-2]
            \arrow["t"', curve={height=12pt}, from=1-1, to=3-2]
            \arrow["s", curve={height=-12pt}, from=1-1, to=2-4]
            \arrow[hook, from=1-1, to=2-2]
        \end{tikzcd}$$
    \end{enumerate}
\end{definition}
\begin{remark}
    If $S=\spec\ZZ$ then \Cref{def: etale equiv relation} defines an ``absolute'' \'{e}tale equivalence relation, as opposed to the $S$-relative notion we just discussed. We omit these technicalities in our discussion going forward. 
\end{remark}
Taking the quotient by $R$, we have a functor
$$(\Sch_{S})^{\Opp}\longrightarrow\Sets$$
by
$$T\mapsto X(T)/R(T).$$
We denote the \'{e}tale sheaf $X/R$ which is in fact an algebraic space. 
\begin{proposition}\label{prop: X mod R is an algebraic space}
    \begin{enumerate}[label=(\alph*)]
        \item Let $S$ be a scheme. If $R$ is an \'{e}tale equivalence relation on an $S$-scheme $X$, then  $X/R$ is an algebraic space. 
        \item If further $Y$ is an $S$-algebraic space and $X\to Y$ an \'{e}tale surjective morphism, then $R$ is an \'{e}tale equivalence relation and $X/R\to Y$ is an isomorphism. 
    \end{enumerate}
\end{proposition}
\subsection{Properties of Algebraic Spaces}
We consider various properties of algebraic spaces, and how they relate to to properties of schemes. 
\begin{definition}[Property of Algebraic Space]\label{def: property of algebraic space}
    Let $\Psf$ be a property of schemes stable on the \'{e}tale site. An algebraic space $X$ has the property $\Psf$ if there exists a scheme $U$ with property $\Psf$ and an \'{e}tale surjection $U\to X$.
\end{definition}
One can define a property of a morphism of algebraic spaces in the following way. 
\begin{definition}[Properties of Morphisms of Algebraic Spaces]\label{def: property of morphisms of spaces}
    Let $f:X\to Y$ be a morphism of $S$-algebraic spaces and $\Psf$ a property of morphisms stable on the \'{e}tale site. The morphism $f$ has the property $\Psf$ if there exists a scheme $V$ and an \'{e}tale surjection $V\to Y$ such that 
    $$% https://q.uiver.app/#q=WzAsNCxbMCwwLCJWXFx0aW1lc197WX1YIl0sWzAsMSwiWCJdLFsyLDEsIlkiXSxbMiwwLCJWIl0sWzMsMl0sWzEsMl0sWzAsM10sWzAsMV1d
    \begin{tikzcd}
        {V\times_{Y}X} && V \\
        X && Y
        \arrow[from=1-3, to=2-3]
        \arrow[from=2-1, to=2-3]
        \arrow[from=1-1, to=1-3]
        \arrow[from=1-1, to=2-1]
    \end{tikzcd}$$
    the morphism $V\times_{Y}X\to V$ has $\Psf$. 
\end{definition}
Using the definition of a property of a morphism of algebraic spaces representable by schemes, we have the following. 
\begin{definition}[Embedding of Algebraic Spaces]\label{def: embedding of algebraic spaces}
    Let $f:X\to Y$ be a morphism of $S$-algebraic spaces. $f$ is an open embedding if $f$ is representable by schemes and there exists a scheme $V$ and an \'{e}tale surjection $V\to Y$ such that the morphism $V\times_{Y}X\to V$ is an embedding. 
\end{definition}
\begin{definition}[Open Embedding of Algebraic Spaces]\label{def: open embedding of algebraic spaces}
    Let $f:X\to Y$ be a morphism of $S$-algebraic spaces. $f$ is an open embedding if $f$ is representable by schemes and there exists a scheme $V$ and an \'{e}tale surjection $V\to Y$ such that the morphism $V\times_{Y}X\to V$ is an open embedding. 
\end{definition}
\begin{definition}[Closed Embedding of Algebraic Spaces]\label{def: closed embedding of algebraic spaces}
    Let $f:X\to Y$ be a morphism of $S$-algebraic spaces. $f$ is an open embedding if $f$ is representable by schemes and there exists a scheme $V$ and an \'{e}tale surjection $V\to Y$ such that the morphism $V\times_{Y}X\to V$ is a closed embedding.
\end{definition}
\begin{remark}
    \Cref{def: embedding of algebraic spaces,def: open embedding of algebraic spaces,def: closed embedding of algebraic spaces} use the fibered product 
    $$% https://q.uiver.app/#q=WzAsNCxbMCwwLCJWXFx0aW1lc197WX1YIl0sWzAsMSwiWCJdLFsyLDEsIlkiXSxbMiwwLCJWIl0sWzMsMl0sWzEsMl0sWzAsM10sWzAsMV1d
    \begin{tikzcd}
        {V\times_{Y}X} && V \\
        X && Y
        \arrow[from=1-3, to=2-3]
        \arrow[from=2-1, to=2-3]
        \arrow[from=1-1, to=1-3]
        \arrow[from=1-1, to=2-1]
    \end{tikzcd}$$
    where $f$ is representable by schemes inducing an associated morphism of schemes $V\times_{Y}X\to V$ that is an embedding (resp. open embedding, resp. closed embedding). 
\end{remark}
One particularly nice property of the category of $S$-algebraic spaces is the following. 
\begin{proposition}\label{prop: spaces has all colimits}
    The category of $S$-algebraic spaces $\Spaces_{S}$ has all finite limits. 
\end{proposition}
Building on \Cref{def: property of algebraic space,def: property of morphisms of spaces}, we can define separatedness of morphisms of algebraic spaces and of algebraic spaces themselves as follows. 
\begin{definition}[Quasiseparated Morphism of Spaces]\label{def: quasiseparated morphism of spaces}
    Let $f:X\to Y$ be a morphism of $S$-algebraic spaces. $f$ is quasiseparated if the relative diagonal $\delta_{X/Y}:X\to X\times_{Y}X$ is quasicompact. 
\end{definition}
\begin{definition}[Locally Separated Morphism of Spaces]\label{def: locally separated morphism of spaces}
    Let $f:X\to Y$ be a morphism of $S$-algebraic spaces. $f$ is locally separated if the relative diagonal $\delta_{X/Y}:X\to X\times_{Y}X$ is an embedding.
\end{definition}
\begin{definition}[Separated Morphism of Spaces]\label{def: separated morphism of spaces}
    Let $f:X\to Y$ be a morphism of $S$-algebraic spaces. $f$ is separated if the relative diagonal $\delta_{X/Y}:X\to X\times_{Y}X$ is a closed embedding.
\end{definition}
The notions for schemes are defined by the structure morphism to $S$ having that particular property. More explicitly, we have the following. 
\begin{definition}[Quasiseparated Algebraic Space]\label{def: quasiseparated space}
    Let $X$ be an $S$ algebraic space. $X$ is quasiseparated if the morphism $X\to S$ is quasiseparated.
\end{definition}
\begin{definition}[Locally Separated Algebraic Space]\label{def: locally separated space}
    Let $X$ be an $S$ algebraic space. $X$ is locally separated if the morphism $X\to S$ is locally separated.
\end{definition}
\begin{definition}[Separated Algebraic Space]\label{def: separated space}
    Let $X$ be an $S$ algebraic space. $X$ is separated if the morphism $X\to S$ is separated. 
\end{definition}
\begin{remark}
    Note here that everything can be checked in the category of schemes since the diagonals are representable by schemes. 
\end{remark}
\begin{remark}
    This style of definition will recur in considering algebraic stacks, \Cref{def: quasiseparated stack morphism,def: separated stack morphism,def: quasiseparated stack,def: separated stack}.
\end{remark}
If we are considering a property morphisms of algebraic spaces not only stable on the \'{e}tale site, but also local on domain, we can refine \Cref{def: property of algebraic space} to a property simply about morphisms of schemes with \Cref{def: property of spaces via schemes}. 
\begin{definition}[Property of Morphisms of Algebraic Spaces via Schemes]\label{def: properties of morphisms of spaces via schemes}
    Let $\Psf$ be a property of morphisms stable on the \'{e}tale site and local on domain and $f:X\to Y$ a morphism of $S$-algebraic spaces. $f$ has $\Psf$ if there exist schemes $U,V$ and surjective \'{e}tale morphisms $U\to X, V\to Y$ such that the morphism $V\times_{Y}U\to V$ has $\Psf$. 
\end{definition}
\begin{remark}
    The morphism $V\times_{Y}U\to V$ in \Cref{def: properties of morphisms of spaces via schemes} above is obtained from the diagram 
    $$% https://q.uiver.app/#q=WzAsNixbMiwxLCJYIl0sWzQsMSwiWSJdLFs0LDAsIlYiXSxbMiwwLCJWXFx0aW1lc197WX1YIl0sWzAsMSwiVSJdLFswLDAsIlxcbGVmdChWXFx0aW1lc197WX1YXFxyaWdodClcXHRpbWVzX3tYfVUiXSxbNCwwXSxbMiwxXSxbMCwxXSxbMywwXSxbMywyXSxbNSw0XSxbNSwzXV0=
    \begin{tikzcd}
        {\left(V\times_{Y}X\right)\times_{X}U} && {V\times_{Y}X} && V \\
        U && X && Y
        \arrow[from=2-1, to=2-3]
        \arrow[from=1-5, to=2-5]
        \arrow[from=2-3, to=2-5]
        \arrow[from=1-3, to=2-3]
        \arrow[from=1-3, to=1-5]
        \arrow[from=1-1, to=2-1]
        \arrow[from=1-1, to=1-3]
    \end{tikzcd}$$
    where the two squares are Cartesian making the outer square Cartesian and giving a canonical isomorphism $U\times_{X}\left(V\times_{Y}X\right)\cong U\times_{Y}V$. 
    $$% https://q.uiver.app/#q=WzAsNyxbMiwyLCJYIl0sWzQsMiwiWSJdLFs0LDEsIlYiXSxbMiwxLCJWXFx0aW1lc197WX1YIl0sWzAsMiwiVSJdLFswLDEsIlxcbGVmdChWXFx0aW1lc197WX1YXFxyaWdodClcXHRpbWVzX3tYfVUiXSxbMCwwLCJWXFx0aW1lc197WX1VIl0sWzQsMF0sWzIsMV0sWzAsMV0sWzMsMF0sWzMsMl0sWzUsNF0sWzUsM10sWzYsNSwiXFxleGlzdHMhXFxjb25nIiwyLHsic3R5bGUiOnsiYm9keSI6eyJuYW1lIjoiZGFzaGVkIn19fV0sWzYsMl1d
    \begin{tikzcd}
        {V\times_{Y}U} \\
        {\left(V\times_{Y}X\right)\times_{X}U} && {V\times_{Y}X} && V \\
        U && X && Y
        \arrow[from=3-1, to=3-3]
        \arrow[from=2-5, to=3-5]
        \arrow[from=3-3, to=3-5]
        \arrow[from=2-3, to=3-3]
        \arrow[from=2-3, to=2-5]
        \arrow[from=2-1, to=3-1]
        \arrow[from=2-1, to=2-3]
        \arrow["{\exists!\cong}"', dashed, from=1-1, to=2-1]
        \arrow[from=1-1, to=2-5]
    \end{tikzcd}$$
\end{remark}
We now continue with several definitions of properties of algebraic spaces and their morphisms. 
\begin{definition}[Quasicompact Algebraic Space]\label{def: quasicompact algebraic space}
    An $S$-algebraic space $S$ is quasicompact if there exists a quasicompact scheme $U$ and a surjective \'{e}tale morphism $U\to X$. 
\end{definition}
\begin{definition}[Noetherian Algebraic Space]\label{def: Noetherian algebraic space}
    An $S$-algebraic space $X$ is Noetherian if it is there exists a Noetherian scheme $U$ and a surjective \'{e}tale morphism $U\to X$. 
\end{definition}
\begin{remark}
    These properties are in fact stable on the \'{e}tale site, so we apply \Cref{def: property of algebraic space}.
\end{remark}
\begin{definition}[Quasicompact Morphism of Algebraic Spaces]\label{def: quasicompact morphism of algebraic spaces}
    Let $f:X\to Y$ be a morphism of $S$-algebraic spaces. $f$ is quasicompact if for any quasicompact scheme $Z$ and morphism $Z\to Y$ the algebraic space $X\times_{Y}Z$ is quasicompact. 
\end{definition}
\subsection{Characterization as $\fppf$-Sheaves}
Another useful definition of morphisms of spaces is as follows. 
\begin{definition}[Properties of Morphisms of Algebraic Spaces]\label{def: property of spaces via schemes}
    Let $f:X\to Y$ be a morphism of $S$-algebraic spaces and $\Psf$ a property of morphisms stable on the \'{e}tale site and local on domain. The morphism $f$ has the property $\Psf$ if there are schemes $U,V$ and \'{e}tale maps $u:U\to X, v:V\to Y$ such that 
    $$% https://q.uiver.app/#q=WzAsNCxbMiwxLCJZIl0sWzIsMCwiViJdLFswLDEsIlUiXSxbMCwwLCJWXFx0aW1lc197WX1VIl0sWzIsMCwiZlxcY2lyYyB1IiwyXSxbMSwwLCJ2Il0sWzMsMl0sWzMsMV1d
    \begin{tikzcd}
        {V\times_{Y}U} && V \\
        U && Y
        \arrow["{f\circ u}"', from=2-1, to=2-3]
        \arrow["v", from=1-3, to=2-3]
        \arrow[from=1-1, to=2-1]
        \arrow[from=1-1, to=1-3]
    \end{tikzcd}$$
    the morphism of schemes $V\times_{Y}U\to V$ has $\Psf$.
\end{definition}
We can refine the topology under consideration and think of algebraic spaces as fppf sheaves as well. 
\begin{theorem}\label{thm: spaces are fppf sheaves}
    If $S$ is a scheme and $X$ an $S$-algebraic space with quasicompact diagonal then $X$ is an fppf sheaf on the site $(\Sch_{S})_{\fppf}$.
\end{theorem}

\section{Quotients in Algebraic Spaces}\label{sec: quotients in algebraic spaces}
In the category of affine schemes and $G$ a finite group acting on an affine scheme $\spec A$, the quotient of the affine scheme by the group action is the affine scheme $\spec A^{G}$, where $A^{G}$ is the ring of $G$-invariant elements of $A$. We consider more general quotients using the tools of algebraic spaces. 
\part*{Algebraic Stacks}
\section{Algebraic Stacks}
\section{Quasicoherent Sheaves on Algebraic Stacks}
\part*{The Geometry of Stacks}
\section{Geometric Properties of Stacks}
\section{Coarse Moduli Spaces}
\section{Coarse Moduli Spaces}\label{sec: coarse moduli spaces}
An important result in stack theory is the Keel-Mori theorem, showing the existence of coarse moduli spaces of algebraic stacks with finite diagonal, which in turn allows us to understand a number of important properties: the local structure of Deligne-Mumford stacks, a variant of Chow's lemma, and finiteness of cohomology of sheaves on Deligne-Mumford stacks. 
\begin{definition}[Coarse Moduli Space]\label{def: coarse moduli space}
    Let $\Xcal$ be an $S$-algebraic stack for a base scheme $S$. A coarse moduli space for the stack $\Xcal$ is a morphism $\pi:\Xcal\to X$ to a scheme $X$ such that the following conditions hold:
    \begin{enumerate}[label=(\alph*)]
        \item $\pi$ is initial for maps to $S$-algebraic spaces. 
        \item For an algebraically closed field $k$, there is a bijection between isomorphism classes of $\Xcal(k)$ to the $k$-rational points of $X$
        $$\left|\Xcal(k)\right|\to X(k).$$
    \end{enumerate}
\end{definition}
\begin{remark}
    $\pi$ being initial means that the scheme $X$ is the scheme $X$ is the initial object of the slice-over category $\Spaces_{(\Xcal/-)}$ with objects morphisms $(\Xcal\to Y)$ and morphisms between two objects $(\Xcal\to Y), (\Xcal\to Z)$ commuting triangles of the following type. 
    $$% https://q.uiver.app/#q=WzAsMyxbMSwwLCJcXFhjYWwiXSxbMCwxLCJYIl0sWzIsMSwiWSJdLFswLDFdLFsxLDJdLFswLDJdXQ==
    \begin{tikzcd}
        & \Xcal \\
        Y && Z
        \arrow[from=1-2, to=2-1]
        \arrow[from=2-1, to=2-3]
        \arrow[from=1-2, to=2-3]
    \end{tikzcd}$$
    More explicitly, for $g:\Xcal\to Z$ with $Z$ an algebraic space, 
    $$% https://q.uiver.app/#q=WzAsMyxbMCwwLCJcXFhjYWwiXSxbMiwxLCJaIl0sWzIsMCwiWCJdLFswLDEsImciLDJdLFswLDIsIlxccGkiXSxbMiwxLCJcXGV4aXN0cyFmIiwwLHsic3R5bGUiOnsiYm9keSI6eyJuYW1lIjoiZGFzaGVkIn19fV1d
    \begin{tikzcd}
        \Xcal && X \\
        && Z
        \arrow["g"', from=1-1, to=2-3]
        \arrow["\pi", from=1-1, to=1-3]
        \arrow["{\exists!f}", dashed, from=1-3, to=2-3]
    \end{tikzcd}$$
    there is a unique morphism of stacks such that $g=f\circ\pi$. 
\end{remark}
\subsection{The Theorem of Keel and Mori}\label{subsec: Keel Mori theorem}
We now state the Keel-Mori theorem. 
\begin{theorem}[Keel-Mori]\label{thm: Keel Mori theorem}
    Let $\Xcal$ be an $S$-algebraic stack locally of finite presentation over $S$ with a finite diagonal over a locally Noetherian base scheme $S$. The algebraic stack $\Xcal$ admits a coarse moduli space such that:
    \begin{enumerate}[label=(\alph*)]
        \item $X$ is an $S$-scheme locally of finite type. Furthermore if $\Xcal$ is a separated algebraic stack, then $X$ is a separated $S$-scheme. 
        \item $\pi$ is a proper morphism and $\Ocal_{X}\to\pi_{*}\Ocal_{\Xcal}$ is an isomorphism. 
        \item If $X'\to X$ is a flat morphism with $X'$ an algebraic space, 
        $$% https://q.uiver.app/#q=WzAsNCxbMCwxLCJcXFhjYWwiXSxbMiwxLCJYIl0sWzIsMCwiWCciXSxbMCwwLCJYJ1xcdGltZXNfe1h9XFxYY2FsIl0sWzAsMSwiXFxwaSIsMl0sWzIsMV0sWzMsMF0sWzMsMl1d
        \begin{tikzcd}
            {X'\times_{X}\Xcal} && {X'} \\
            \Xcal && X
            \arrow["\pi"', from=2-1, to=2-3]
            \arrow[from=1-3, to=2-3]
            \arrow[from=1-1, to=2-1]
            \arrow[from=1-1, to=1-3]
        \end{tikzcd}$$
        then $X'$ is a coarse moduli space for $X'\times_{X}\Xcal$. 
    \end{enumerate}
\end{theorem}
The Keel-Mori theorem allows us to connect Deligne-Mumford stacks (\ref{def: Deligne Mumford stack}) with (topological) orbifolds. 
\newpage
\subsection{Local Structure of Deligne-Mumford Stacks}\label{subsec: local structure of DM stacks}
\begin{theorem}[Local Structure of Deligne-Mumford Stacks]\label{def: local structure of DM stacks}
    Let $\Xcal$ be a Deligne-Mumford stack locally of finite type with finite diagonal over a locally Noetherian base scheme $S$, and $\pi:\Xcal\to X$ its coarse moduli space. Let $\widetilde{x}$ be a geometric point of $\Xcal$ with image $\pi(\widetilde{x})=\overline{x}$ and $G_{\widetilde{x}}$ the automorphism group of $\widetilde{x}$. There exists $\overline{x}\in U\subseteq X$ an \'{e}tale neighborhood of $\overline{x}$ and a finite $U$-scheme $V$ with action of $G_{\widetilde{x}}$ such that in the Cartesian square 
    $$% https://q.uiver.app/#q=WzAsNCxbMCwxLCJcXFhjYWwiXSxbMiwxLCJYIl0sWzIsMCwiVSJdLFswLDAsIlVcXHRpbWVzX3tYfVxcWGNhbCJdLFswLDEsIlxccGkiLDJdLFsyLDFdLFszLDBdLFszLDJdXQ==
    \begin{tikzcd}
        {U\times_{X}\Xcal} && U \\
        \Xcal && X
        \arrow["\pi"', from=2-1, to=2-3]
        \arrow[from=1-3, to=2-3]
        \arrow[from=1-1, to=2-1]
        \arrow[from=1-1, to=1-3]
    \end{tikzcd}$$
    we have an isomorphism of stacks 
    $$U\times_{X}\Xcal\cong\left[V/G_{\widetilde{x}}\right].$$
\end{theorem}
\begin{remark}
    Since $\Xcal$ is a Deligne-Mumford stack with $\widetilde{x}$ a geometric point, $G_{\widetilde{x}}$ is a finite group. 
\end{remark}
The finiteness of the automorphism group $G_{\widetilde{x}}$ allows us to define another property of Deligne-Mumford stacks. 
\begin{definition}[Tame Deligne-Mumford Stack]\label{def: tame DM stack}
    Let $\Xcal$ be an algebraic stack locally of finite type over a locally Noetherian base scheme $S$. $\Xcal$ is a tame stack if for every geometric point $\widetilde{x}:\spec(k)\to\Xcal$, the order of the automorphism group $|G_{\widetilde{x}}|$ is an invertible element in $k$. 
\end{definition}
In the case of sufficiently nice Deligne-Mumford stacks, its sheaves are characterized by the sheaves on the coarse space. 
\begin{proposition}\label{prop: exact map of QCoh sheaves on stacks to coarse space}
    Let $\Xcal$ be Deligne-Mumford stack locally of finite type with finite diagonal over a locally Noetherian base scheme $S$ and coarse space $\pi:\Xcal\to X$. If $\Xcal$ is a tame stack, then the functor 
    $$\pi_{*}:\QCoh\left(\Xcal\right)_{\LisEt}\longrightarrow\QCoh\left(X\right)$$
    is exact. 
\end{proposition}
The following theorem characterizes the behavior of coarse spaces under base change. 
\begin{theorem}\label{thm: coarse spaces under base change}
    Let $\Xcal$ be a separated Deligne-Mumford stack of finite type over a locally Noetherian base scheme $S$ and coarse space $\pi:\Xcal\to X$. For $S'\to S$, $\tau:\Xcal\times_{S}S'\to Y$ the coarse space of the base change of the stack $\Xcal$ and $p:Y\to X\times_{S}S'$ the morphism induced by the universal property of the coarse space, $p$ is a universal homeomorphism. If further $S'\to S$ is flat or $\Xcal$ is a tame stack, then $p$ is an isomorphism. 
\end{theorem}
\begin{remark}
    The properties of local finite typeness and separatedness are preserved by base change so the stack $\Xcal\times_{S}S'$ and the scheme $X\times_{S}S'$ obtained by base changes
    $$% https://q.uiver.app/#q=WzAsOCxbMCwwLCJcXFhjYWxcXHRpbWVzX3tTfVMnIl0sWzIsMSwiUyJdLFsyLDAsIlxcWGNhbCJdLFswLDEsIlMnIl0sWzQsMSwiUyciXSxbNiwxLCJTIl0sWzYsMCwiWCJdLFs0LDAsIlhcXHRpbWVzX3tTfVMnIl0sWzMsMV0sWzAsMl0sWzAsM10sWzIsMV0sWzQsNV0sWzYsNV0sWzcsNF0sWzcsNl1d
    \begin{tikzcd}
        {\Xcal\times_{S}S'} && \Xcal && {X\times_{S}S'} && X \\
        {S'} && S && {S'} && S
        \arrow[from=2-1, to=2-3]
        \arrow[from=1-1, to=1-3]
        \arrow[from=1-1, to=2-1]
        \arrow[from=1-3, to=2-3]
        \arrow[from=2-5, to=2-7]
        \arrow[from=1-7, to=2-7]
        \arrow[from=1-5, to=2-5]
        \arrow[from=1-5, to=1-7]
    \end{tikzcd}$$
    is also locally of finite type and separated, hence admitting a smooth moduli space by the Keel-Mori theorem \Cref{thm: Keel Mori theorem}. 
\end{remark}
\subsection{Chow's Lemma for Deligne-Mumford Stacks}\label{subsec: Chows lemma for DM stacks}
\begin{theorem}[Chow's Lemma for Stacks]\label{thm: Chows lemma for stacks}
    Let $\Xcal$ be a Deligne-Mumford stack of finite type with finite diagonal over a Noetherian base-scheme $S$. There exists a proper surjective morphism $X'\to\Xcal$ with $X'$ an $S$-scheme finite over a dense open substack of $\Xcal$ such that the composition 
    $$X'\longrightarrow\Xcal\longrightarrow S$$
    is a projective morphism of schemes. 
\end{theorem}
\section{Cohomology of Stacks}
\section{Derived Categories of Stacks}
\part*{End Material}
\printbibliography
\end{document}