\part*{Algebraic Spaces}\label{part: algebraic spaces}
\section{Algebraic Spaces}\label{sec: algebraic spaces}
We begin with a discussion of various algebro-geometric and categorical preliminaries required to define algebraic spaces. 
\subsection{The Category of Schemes}
First recall that we have defined stable properties of morphisms in \Cref{def: stable property of morphisms}. We now define a stable property of objects in a category. 
\begin{definition}[Stable Class of Objects]\label{def: stable class of objects}
    Let $\Ssf$ be a site such that all representable presheaves are sheaves. A subclass of objects $\Obj(\Csf)\subseteq\Obj(\Ssf)$ is stable with respect to a property $\Psf$ if for all coverings $\{X_{i}\to X\}$, $X$ has property $\Psf$ if and only if $X_{i}$ have property $\Psf$ for all $i$. 
\end{definition}
\begin{definition}[Stable Property of Objects]\label{def: stable property of objects}
    A property $\Psf$ is stable if the class of objects containing $\Psf$ is stable.
\end{definition}
Turning to some categorical notions, we define the following. 
\begin{definition}[Closed Subcategory]\label{def: closed subcategory}
    Let $\Csf$ be a category. A closed subcategory $\Dsf$ of $\Csf$ is a subcategory such that the following conditions hold:
    \begin{enumerate}[label=(\alph*)]
        \item $\Dsf$ contains all isomorphisms in $\Csf$.
        \item For all Cartesian diagrams in $\Csf$
        $$% https://q.uiver.app/#q=WzAsNCxbMCwwLCJYJyJdLFswLDEsIlknIl0sWzIsMCwiWCJdLFsyLDEsIlkiXSxbMiwzLCJmIl0sWzAsMSwiZiciLDJdLFsxLDNdLFswLDJdXQ==
        \begin{tikzcd}
            {X'} && X \\
            {Y'} && Y
            \arrow["f", from=1-3, to=2-3]
            \arrow["{f'}"', from=1-1, to=2-1]
            \arrow[from=2-1, to=2-3]
            \arrow[from=1-1, to=1-3]
        \end{tikzcd}$$
        if $f\in\Mor_{\Dsf}$ then $f'\in\Mor_{\Dsf}$.
    \end{enumerate}
\end{definition}
\begin{definition}[Stable Subcategory]\label{def: stable subcategory}
    Let $\Dsf$ be a closed subcategory of a category $\Csf$. $\Dsf$ is stable if for all $(f:X\to Y)\in\Mor_{\Csf}$ and all coverings $\{Y_{i}\to Y\}$, $f$ is a morphism in $\Dsf$ if and only if all morphisms $f_{i}:X\times_{Y}XY_{i}\to Y_{i}$ are in $\Dsf$. 
\end{definition}
\begin{definition}[Local On Domain]\label{def: local on domain}
    A stable closed subcategory $\Dsf$ of a category $\Csf$ is local on domain if for all $f:X\to Y$ and coverings $\{X_{i}\to X\}$, $f$ is a morphism in $\Dsf$ if and only if $X_{i}\to X\to Y$ is in $\Dsf$. 
\end{definition}
We alternatively define stable properties of morphisms \Cref{def: stable property of morphisms} as follows. 
\begin{definition}[Stable Property of Morphisms]\label{def: stable property of morphisms 2}
    Let $\Psf$ be some property of morphisms in $\Csf$ satisfied by isomorphisms and closed under compositions and $\Dsf_{\Psf}$ the subcategory of $\Csf$ with the same objects but morphisms those satisfying property $\Psf$. $\Psf$ is a stable property of morphisms if $\Dsf_{\Psf}$ is a stable subcategory. 
\end{definition}
Similarly, we define locality on domain as follows. 
\begin{definition}[Morphisms Local on Domain]\label{def: morphisms local on domain}
    Let $\Psf$ be some property of morphisms in $\Csf$ satisfied by isomorphisms and closed under compositions and $\Dsf_{\Psf}$ the subcategory of $\Csf$ with the same objects but morphisms those satisfying property $\Psf$. $\Psf$ is a stable property of morphisms if $\Dsf_{\Psf}$ is local on domain as a subcategory. 
\end{definition}
One can show the following fact about morphisms of schemes. \newpage
\begin{proposition}\label{prop: stable properties of morphisms of schemes}
    Let $S$ be a scheme and consider the big \'{e}tale site $(\Sch_{S})_{\et}$. 
    \begin{enumerate}[label=(\alph*)]
        \item The following properties of morphisms in $(\Sch_{S})_{\et}$ are stable: proper, separated, surjective, and quasi-compact. 
        \item The following properties of morphisms in $(\Sch_{S})_{\et}$ are stable and local on domain: locally of finite type, locally of finite presentation, flat, \'{e}tale, universally open, locally quasi-finite, and smooth. 
    \end{enumerate}
\end{proposition}
\subsection{Representability and Sheaves on the \'{E}tale Site}
We now introduce some language to describe morphisms of sheaves on the \'{e}tale site. 
\begin{definition}[Representable by Schemes]\label{def: morphism of etale sheaves representable by schemes}
    Let $S$ be a scheme and let $\Fsf,\Gsf$ be sheaves on the site $(\Sch_{S})_{\et}$ and $f:\Fsf\to\Gsf$ a morphism of sheaves. The morphism $f$ is representable by schemes if for every $S$-scheme $T$ and morphism $T\to \Gsf$ the fibered product functor $\Fsf\times_{\Gsf}T$ is representable by a scheme. 
\end{definition}
\begin{definition}[Properties of Morphisms of \'{E}tale Sheaves]
    Let $S$ be a scheme and let $\Fsf,\Gsf$ be sheaves on the site $(\Sch_{S})_{\et}$, $f:\Fsf\to\Gsf$ a morphism of sheaves, and $\Psf$ a stable property of morphisms of schemes. If $f$ is representable by schemes, then $f$ has property $\Psf$ if for every $S$-scheme $T$ the morphism $\pr_{2}:\Fsf\times_{\Gsf}T\to T$ has $\Psf$. 
\end{definition}
\begin{example}
    Let $\Fsf,\Gsf$ be representable sheaves, say by $X,Y$, respectively. Any morphism $f:\Fsf\to\Gsf$ is representable by sheaves, induced by the corresponding morphism of schemes $X\to Y$. For any $S$-scheme $T$ with a map $T\to Y$ we have that $\Fsf\times_{\Gsf}T$ is the scheme $T\times_{Y}X$. 
\end{example}
\begin{lemma}\label{lem: morphisms of schemes and etale sheaves}
    Let $S$ be a scheme and $f:X\to Y$ be a $S$-morphism of schemes and $\Psf$ a stable property of morphisms of schemes. The morphism $f$ has property $\Psf$ if and only if the morphism of representable sheaves $h_{f}:h_{X}\to Y$ has property $\Psf$. 
\end{lemma}
\begin{proof}
    Let $T$ be an $S$-scheme and $g:T\to Y$ with $h_{g}:h_{T}\to h_{Y}$ where we have a natural isomorphism $h_{T}\times_{h_{Y}}h_{X}\Longrightarrow h_{X\times_{Y}T}$.
    $$% https://q.uiver.app/#q=WzAsOCxbMiwxLCJZIl0sWzIsMCwiVCJdLFswLDEsIlgiXSxbMCwwLCJYXFx0aW1lc197WX1UIl0sWzQsMCwiaF97WH1cXHRpbWVzX3toX3tZfX1oX3tUfVxcUmlnaHRhcnJvdyBoX3tYXFx0aW1lc197WX1UfSJdLFs0LDEsImhfe1h9Il0sWzYsMCwiaF97VH0iXSxbNiwxLCJoX3tZfSJdLFsyLDBdLFszLDFdLFsxLDBdLFszLDJdLFs1LDddLFs0LDZdLFs0LDVdLFs2LDddXQ==
    \begin{tikzcd}
        {X\times_{Y}T} && T && {h_{T}\times_{h_{Y}}h_{X}\Rightarrow h_{X\times_{Y}T}} && {h_{T}} \\
        X && Y && {h_{X}} && {h_{Y}}
        \arrow[from=2-1, to=2-3]
        \arrow[from=1-1, to=1-3]
        \arrow[from=1-3, to=2-3]
        \arrow[from=1-1, to=2-1]
        \arrow[from=2-5, to=2-7]
        \arrow[from=1-5, to=1-7]
        \arrow[from=1-5, to=2-5]
        \arrow[from=1-7, to=2-7]
    \end{tikzcd}$$
    We have that $f$ is $\Psf$ if and only if $X\times_{Y}T\to T$ is $\Psf$ if and only if $h_{X\times_{Y}T}\to h_{T}$ is $\Psf$ if and only if $h_{f}$ is $\Psf$ upon repeated application of the stable property of morphisms. 
\end{proof}
We prove an additional lemma before defining algebraic spaces.
\begin{lemma}[Representable Diagonal Implies Representability]\label{lem: rep diagonal then rep}
    Let $S$ be a scheme and $\Fsf$ a sheaf on $(\Sch_{S})_{\et}$. If $\delta_{\Fsf}:\Fsf\to\Fsf\times\Fsf$ is representable by schemes and $T$ any $S$-scheme then the morphism of sheaves $f:T\to\Fsf$ is representable by schemes. 
\end{lemma} 
\begin{proof}
    This follows from general abstract nonsense. See \cite[\href{https://stacks.math.columbia.edu/tag/0022}{Lemma 0022}]{stacks-project} and \cite[\href{https://stacks.math.columbia.edu/tag/0024}{Lemma 0024}]{stacks-project}. 
\end{proof}
\newpage
\subsection{Algebraic Spaces}
We are now able to define algebraic spaces, an important, slightly more restricted notion of an algebraic stack. 
\begin{definition}[Algebraic Space]\label{def: algebraic space}
    Let $S$ be a scheme. An algebraic space over $S$ is a functor $X:\Sch_{S}^{\Opp}\to\Sets$ such that the following hold:
    \begin{enumerate}[label=(\alph*)]
        \item $X$ is a sheaf on the big \'{e}tale site $(\Sch_{S})_{\et}$. 
        \item $\delta_{\Fsf}:\Fsf\to\Fsf\times_{S}\Fsf$ is representable by schemes. 
        \item There is an $S$-scheme $U$ and a surjective \'{e}tale morphism $U\to X$. 
    \end{enumerate}
\end{definition}
\begin{remark}
    Let $S$ be a scheme. Objects of the category of $S$-schemes $\Sch_{S}$ are tautologically algebraic spaces. 
\end{remark}
Given a base scheme $S$, we can consider the category of algebraic spaces $\Spaces_{S}$ with objects sheaves of sets on the big \'{e}tale site $(\Sch_{S})_{\et}$ and morphisms base-preserving 2-functors, or morphisms of sheaves on the site.  

\subsection{Sheaf Quotients}
We begin with the definition of an \'{e}tale equivalence relation. 
\begin{definition}[\'{E}tale Equivalence Relation]\label{def: etale equiv relation}
    Let $S$ be a scheme. An \'{e}tale equivalence relation on an $S$-scheme $X$ is a monomorphism of schemes $R\hookrightarrow X\times_{S}X$ such that 
    \begin{enumerate}[label=(\alph*)]
        \item For all $S$-schemes $T$, the $T$-points $R(T)\subseteq X(T)\times X(T)$ is an equivalence relation on $X(T)$. 
        \item The maps $s,t:R\to X$ induced by the projections from $X\times_{S}X$ are \'{e}tale morphisms.
        $$% https://q.uiver.app/#q=WzAsNCxbMCwwLCJSIl0sWzEsMSwiWFxcdGltZXNfe1N9WCJdLFsxLDIsIlgiXSxbMywxLCJYIl0sWzEsM10sWzEsMl0sWzAsMiwidCIsMix7ImN1cnZlIjoyfV0sWzAsMywicyIsMCx7ImN1cnZlIjotMn1dLFswLDEsIiIsMSx7InN0eWxlIjp7InRhaWwiOnsibmFtZSI6Imhvb2siLCJzaWRlIjoidG9wIn19fV1d
        \begin{tikzcd}
            R \\
            & {X\times_{S}X} && X \\
            & X
            \arrow[from=2-2, to=2-4]
            \arrow[from=2-2, to=3-2]
            \arrow["t"', curve={height=12pt}, from=1-1, to=3-2]
            \arrow["s", curve={height=-12pt}, from=1-1, to=2-4]
            \arrow[hook, from=1-1, to=2-2]
        \end{tikzcd}$$
    \end{enumerate}
\end{definition}
\begin{remark}
    If $S=\spec\ZZ$ then \Cref{def: etale equiv relation} defines an ``absolute'' \'{e}tale equivalence relation, as opposed to the $S$-relative notion we just discussed. We omit these technicalities in our discussion going forward. 
\end{remark}
Taking the quotient by $R$, we have a functor
$$(\Sch_{S})^{\Opp}\longrightarrow\Sets$$
by
$$T\mapsto X(T)/R(T).$$
We denote the \'{e}tale sheaf $X/R$ which is in fact an algebraic space. 
\begin{proposition}\label{prop: X mod R is an algebraic space}
    \begin{enumerate}[label=(\alph*)]
        \item Let $S$ be a scheme. If $R$ is an \'{e}tale equivalence relation on an $S$-scheme $X$, then  $X/R$ is an algebraic space. 
        \item If further $Y$ is an $S$-algebraic space and $X\to Y$ an \'{e}tale surjective morphism, then $R$ is an \'{e}tale equivalence relation and $X/R\to Y$ is an isomorphism. 
    \end{enumerate}
\end{proposition}
\subsection{Properties of Algebraic Spaces}
We consider various properties of algebraic spaces, and how they relate to to properties of schemes. 
\begin{definition}[Property of Algebraic Space]\label{def: property of algebraic space}
    Let $\Psf$ be a property of schemes stable on the \'{e}tale site. An algebraic space $X$ has the property $\Psf$ if there exists a scheme $U$ with property $\Psf$ and an \'{e}tale surjection $U\to X$.
\end{definition}
One can define a property of a morphism of algebraic spaces in the following way. 
\begin{definition}[Properties of Morphisms of Algebraic Spaces]\label{def: property of morphisms of spaces}
    Let $f:X\to Y$ be a morphism of $S$-algebraic spaces and $\Psf$ a property of morphisms stable on the \'{e}tale site. The morphism $f$ has the property $\Psf$ if there exists a scheme $V$ and an \'{e}tale surjection $V\to Y$ such that 
    $$% https://q.uiver.app/#q=WzAsNCxbMCwwLCJWXFx0aW1lc197WX1YIl0sWzAsMSwiWCJdLFsyLDEsIlkiXSxbMiwwLCJWIl0sWzMsMl0sWzEsMl0sWzAsM10sWzAsMV1d
    \begin{tikzcd}
        {V\times_{Y}X} && V \\
        X && Y
        \arrow[from=1-3, to=2-3]
        \arrow[from=2-1, to=2-3]
        \arrow[from=1-1, to=1-3]
        \arrow[from=1-1, to=2-1]
    \end{tikzcd}$$
    the morphism $V\times_{Y}X\to V$ has $\Psf$. 
\end{definition}
Using the definition of a property of a morphism of algebraic spaces representable by schemes, we have the following. 
\begin{definition}[Embedding of Algebraic Spaces]\label{def: embedding of algebraic spaces}
    Let $f:X\to Y$ be a morphism of $S$-algebraic spaces. $f$ is an open embedding if $f$ is representable by schemes and there exists a scheme $V$ and an \'{e}tale surjection $V\to Y$ such that the morphism $V\times_{Y}X\to V$ is an embedding. 
\end{definition}
\begin{definition}[Open Embedding of Algebraic Spaces]\label{def: open embedding of algebraic spaces}
    Let $f:X\to Y$ be a morphism of $S$-algebraic spaces. $f$ is an open embedding if $f$ is representable by schemes and there exists a scheme $V$ and an \'{e}tale surjection $V\to Y$ such that the morphism $V\times_{Y}X\to V$ is an open embedding. 
\end{definition}
\begin{definition}[Closed Embedding of Algebraic Spaces]\label{def: closed embedding of algebraic spaces}
    Let $f:X\to Y$ be a morphism of $S$-algebraic spaces. $f$ is an open embedding if $f$ is representable by schemes and there exists a scheme $V$ and an \'{e}tale surjection $V\to Y$ such that the morphism $V\times_{Y}X\to V$ is a closed embedding.
\end{definition}
\begin{remark}
    \Cref{def: embedding of algebraic spaces,def: open embedding of algebraic spaces,def: closed embedding of algebraic spaces} use the fibered product 
    $$% https://q.uiver.app/#q=WzAsNCxbMCwwLCJWXFx0aW1lc197WX1YIl0sWzAsMSwiWCJdLFsyLDEsIlkiXSxbMiwwLCJWIl0sWzMsMl0sWzEsMl0sWzAsM10sWzAsMV1d
    \begin{tikzcd}
        {V\times_{Y}X} && V \\
        X && Y
        \arrow[from=1-3, to=2-3]
        \arrow[from=2-1, to=2-3]
        \arrow[from=1-1, to=1-3]
        \arrow[from=1-1, to=2-1]
    \end{tikzcd}$$
    where $f$ is representable by schemes inducing an associated morphism of schemes $V\times_{Y}X\to V$ that is an embedding (resp. open embedding, resp. closed embedding). 
\end{remark}
One particularly nice property of the category of $S$-algebraic spaces is the following. 
\begin{proposition}\label{prop: spaces has all colimits}
    The category of $S$-algebraic spaces $\Spaces_{S}$ has all finite limits. 
\end{proposition}
Building on \Cref{def: property of algebraic space,def: property of morphisms of spaces}, we can define separatedness of morphisms of algebraic spaces and of algebraic spaces themselves as follows. 
\begin{definition}[Quasiseparated Morphism of Spaces]\label{def: quasiseparated morphism of spaces}
    Let $f:X\to Y$ be a morphism of $S$-algebraic spaces. $f$ is quasiseparated if the relative diagonal $\delta_{X/Y}:X\to X\times_{Y}X$ is quasicompact. 
\end{definition}
\begin{definition}[Locally Separated Morphism of Spaces]\label{def: locally separated morphism of spaces}
    Let $f:X\to Y$ be a morphism of $S$-algebraic spaces. $f$ is locally separated if the relative diagonal $\delta_{X/Y}:X\to X\times_{Y}X$ is an embedding.
\end{definition}
\begin{definition}[Separated Morphism of Spaces]\label{def: separated morphism of spaces}
    Let $f:X\to Y$ be a morphism of $S$-algebraic spaces. $f$ is separated if the relative diagonal $\delta_{X/Y}:X\to X\times_{Y}X$ is a closed embedding.
\end{definition}
The notions for schemes are defined by the structure morphism to $S$ having that particular property. More explicitly, we have the following. 
\begin{definition}[Quasiseparated Algebraic Space]\label{def: quasiseparated space}
    Let $X$ be an $S$ algebraic space. $X$ is quasiseparated if the morphism $X\to S$ is quasiseparated.
\end{definition}
\begin{definition}[Locally Separated Algebraic Space]\label{def: locally separated space}
    Let $X$ be an $S$ algebraic space. $X$ is locally separated if the morphism $X\to S$ is locally separated.
\end{definition}
\begin{definition}[Separated Algebraic Space]\label{def: separated space}
    Let $X$ be an $S$ algebraic space. $X$ is separated if the morphism $X\to S$ is separated. 
\end{definition}
\begin{remark}
    Note here that everything can be checked in the category of schemes since the diagonals are representable by schemes. 
\end{remark}
\begin{remark}
    This style of definition will recur in considering algebraic stacks, \Cref{def: quasiseparated stack morphism,def: separated stack morphism,def: quasiseparated stack,def: separated stack}.
\end{remark}
If we are considering a property morphisms of algebraic spaces not only stable on the \'{e}tale site, but also local on domain, we can refine \Cref{def: property of algebraic space} to a property simply about morphisms of schemes with \Cref{def: property of spaces via schemes}. 
\begin{definition}[Property of Morphisms of Algebraic Spaces via Schemes]\label{def: properties of morphisms of spaces via schemes}
    Let $\Psf$ be a property of morphisms stable on the \'{e}tale site and local on domain and $f:X\to Y$ a morphism of $S$-algebraic spaces. $f$ has $\Psf$ if there exist schemes $U,V$ and surjective \'{e}tale morphisms $U\to X, V\to Y$ such that the morphism $V\times_{Y}U\to V$ has $\Psf$. 
\end{definition}
\begin{remark}
    The morphism $V\times_{Y}U\to V$ in \Cref{def: properties of morphisms of spaces via schemes} above is obtained from the diagram 
    $$% https://q.uiver.app/#q=WzAsNixbMiwxLCJYIl0sWzQsMSwiWSJdLFs0LDAsIlYiXSxbMiwwLCJWXFx0aW1lc197WX1YIl0sWzAsMSwiVSJdLFswLDAsIlxcbGVmdChWXFx0aW1lc197WX1YXFxyaWdodClcXHRpbWVzX3tYfVUiXSxbNCwwXSxbMiwxXSxbMCwxXSxbMywwXSxbMywyXSxbNSw0XSxbNSwzXV0=
    \begin{tikzcd}
        {\left(V\times_{Y}X\right)\times_{X}U} && {V\times_{Y}X} && V \\
        U && X && Y
        \arrow[from=2-1, to=2-3]
        \arrow[from=1-5, to=2-5]
        \arrow[from=2-3, to=2-5]
        \arrow[from=1-3, to=2-3]
        \arrow[from=1-3, to=1-5]
        \arrow[from=1-1, to=2-1]
        \arrow[from=1-1, to=1-3]
    \end{tikzcd}$$
    where the two squares are Cartesian making the outer square Cartesian and giving a canonical isomorphism $U\times_{X}\left(V\times_{Y}X\right)\cong U\times_{Y}V$. 
    $$% https://q.uiver.app/#q=WzAsNyxbMiwyLCJYIl0sWzQsMiwiWSJdLFs0LDEsIlYiXSxbMiwxLCJWXFx0aW1lc197WX1YIl0sWzAsMiwiVSJdLFswLDEsIlxcbGVmdChWXFx0aW1lc197WX1YXFxyaWdodClcXHRpbWVzX3tYfVUiXSxbMCwwLCJWXFx0aW1lc197WX1VIl0sWzQsMF0sWzIsMV0sWzAsMV0sWzMsMF0sWzMsMl0sWzUsNF0sWzUsM10sWzYsNSwiXFxleGlzdHMhXFxjb25nIiwyLHsic3R5bGUiOnsiYm9keSI6eyJuYW1lIjoiZGFzaGVkIn19fV0sWzYsMl1d
    \begin{tikzcd}
        {V\times_{Y}U} \\
        {\left(V\times_{Y}X\right)\times_{X}U} && {V\times_{Y}X} && V \\
        U && X && Y
        \arrow[from=3-1, to=3-3]
        \arrow[from=2-5, to=3-5]
        \arrow[from=3-3, to=3-5]
        \arrow[from=2-3, to=3-3]
        \arrow[from=2-3, to=2-5]
        \arrow[from=2-1, to=3-1]
        \arrow[from=2-1, to=2-3]
        \arrow["{\exists!\cong}"', dashed, from=1-1, to=2-1]
        \arrow[from=1-1, to=2-5]
    \end{tikzcd}$$
\end{remark}
We now continue with several definitions of properties of algebraic spaces and their morphisms. 
\begin{definition}[Quasicompact Algebraic Space]\label{def: quasicompact algebraic space}
    An $S$-algebraic space $S$ is quasicompact if there exists a quasicompact scheme $U$ and a surjective \'{e}tale morphism $U\to X$. 
\end{definition}
\begin{definition}[Noetherian Algebraic Space]\label{def: Noetherian algebraic space}
    An $S$-algebraic space $X$ is Noetherian if it is there exists a Noetherian scheme $U$ and a surjective \'{e}tale morphism $U\to X$. 
\end{definition}
\begin{remark}
    These properties are in fact stable on the \'{e}tale site, so we apply \Cref{def: property of algebraic space}.
\end{remark}
\begin{definition}[Quasicompact Morphism of Algebraic Spaces]\label{def: quasicompact morphism of algebraic spaces}
    Let $f:X\to Y$ be a morphism of $S$-algebraic spaces. $f$ is quasicompact if for any quasicompact scheme $Z$ and morphism $Z\to Y$ the algebraic space $X\times_{Y}Z$ is quasicompact. 
\end{definition}
\subsection{Characterization as $\fppf$-Sheaves}
Another useful definition of morphisms of spaces is as follows. 
\begin{definition}[Properties of Morphisms of Algebraic Spaces]\label{def: property of spaces via schemes}
    Let $f:X\to Y$ be a morphism of $S$-algebraic spaces and $\Psf$ a property of morphisms stable on the \'{e}tale site and local on domain. The morphism $f$ has the property $\Psf$ if there are schemes $U,V$ and \'{e}tale maps $u:U\to X, v:V\to Y$ such that 
    $$% https://q.uiver.app/#q=WzAsNCxbMiwxLCJZIl0sWzIsMCwiViJdLFswLDEsIlUiXSxbMCwwLCJWXFx0aW1lc197WX1VIl0sWzIsMCwiZlxcY2lyYyB1IiwyXSxbMSwwLCJ2Il0sWzMsMl0sWzMsMV1d
    \begin{tikzcd}
        {V\times_{Y}U} && V \\
        U && Y
        \arrow["{f\circ u}"', from=2-1, to=2-3]
        \arrow["v", from=1-3, to=2-3]
        \arrow[from=1-1, to=2-1]
        \arrow[from=1-1, to=1-3]
    \end{tikzcd}$$
    the morphism of schemes $V\times_{Y}U\to V$ has $\Psf$.
\end{definition}
We can refine the topology under consideration and think of algebraic spaces as fppf sheaves as well. 
\begin{theorem}\label{thm: spaces are fppf sheaves}
    If $S$ is a scheme and $X$ an $S$-algebraic space with quasicompact diagonal then $X$ is an fppf sheaf on the site $(\Sch_{S})_{\fppf}$.
\end{theorem}
